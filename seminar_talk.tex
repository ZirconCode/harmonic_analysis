%%%%%%%%%%%%%%%%%%%%%%%%%%%%%%%%%%%%%%%%%%%%%%%%%%%%%%%%%%%%%%%%%%%%%%%%%%
%Author:																 %
%-------																 %
%Yannis Baehni at University of Zurich									 %
%baehni.yannis@uzh.ch													 %
%																		 %
%Version log:															 %
%------------															 %
%06/02/16 . Basic structure												 %
%04/08/16 . Layout changes including section, contents, abstract.		 %
%%%%%%%%%%%%%%%%%%%%%%%%%%%%%%%%%%%%%%%%%%%%%%%%%%%%%%%%%%%%%%%%%%%%%%%%%%

%Page Setup
\documentclass[
	11pt, 
	oneside, 
	a4paper,
	reqno,
	final
]{amsart}

\usepackage[
	left = 3cm, 
	right = 3cm, 
	top = 3cm, 
	bottom = 3cm
]{geometry}

%Headers and footers
\usepackage{fancyhdr}
	\pagestyle{fancy}
	%Clear fields
	\fancyhf{}
	%Header right
	\fancyhead[R]{
		\footnotesize
		Yannis B\"{a}hni\\
		\href{mailto:yannis.baehni@uzh.ch}{yannis.baehni@uzh.ch}
	}
	%Header left
	\fancyhead[L]{
		\footnotesize
		MAT694 Seminar: Introduction to Harmonic Analysis\\
		HS16
	}
	%Page numbering in footer
	\fancyfoot[C]{\thepage}
	%Separation line header and footer
	\renewcommand{\headrulewidth}{0.4pt}
	%\renewcommand{\footrulewidth}{0.4pt}
	
	\setlength{\headheight}{19pt} 

%Title
\usepackage[foot]{amsaddr}
\usepackage{xspace}
\makeatletter
\def\@textbottom{\vskip \z@ \@plus 1pt}
\let\@texttop\relax
\usepackage{etoolbox}
\patchcmd{\abstract}{\scshape\abstractname}{\textbf{\abstractname}}{}{}

%Switching commands for different section formats
%Mainsectionsytle
\newcommand{\mainsectionstyle}{%
  \renewcommand{\@secnumfont}{\bfseries}
  \renewcommand\section{\@startsection{section}{2}%
    \z@{.5\linespacing\@plus.7\linespacing}{-.5em}%
    {\normalfont\bfseries}}%
}
\newcommand{\originalsectionstyle}{%
\def\@secnumfont{\bfseries}%\mdseries
\def\section{\@startsection{section}{1}%
  \z@{.7\linespacing\@plus\linespacing}{.5\linespacing}%
  {\normalfont\bfseries\centering}}
}
%Formatting title of TOC
\renewcommand{\contentsnamefont}{\bfseries}
%Table of Contents
\setcounter{tocdepth}{3}

% Add bold to \section titles in ToC and remove . after numbers
\renewcommand{\tocsection}[3]{%
  \indentlabel{\@ifnotempty{#2}{\bfseries\ignorespaces#1 #2\quad}}\bfseries#3}
% Remove . after numbers in \subsection
\renewcommand{\tocsubsection}[3]{%
  \indentlabel{\@ifnotempty{#2}{\ignorespaces#1 #2\quad}}#3}
%\let\tocsubsubsection\tocsubsection% Update for \subsubsection
%...

\newcommand\@dotsep{4.5}
\def\@tocline#1#2#3#4#5#6#7{\relax
  \ifnum #1>\c@tocdepth % then omit
  \else
    \par \addpenalty\@secpenalty\addvspace{#2}%
    \begingroup \hyphenpenalty\@M
    \@ifempty{#4}{%
      \@tempdima\csname r@tocindent\number#1\endcsname\relax
    }{%
      \@tempdima#4\relax
    }%
    \parindent\z@ \leftskip#3\relax \advance\leftskip\@tempdima\relax
    \rightskip\@pnumwidth plus1em \parfillskip-\@pnumwidth
    #5\leavevmode\hskip-\@tempdima{#6}\nobreak
    \leaders\hbox{$\m@th\mkern \@dotsep mu\hbox{.}\mkern \@dotsep mu$}\hfill
    \nobreak
    \hbox to\@pnumwidth{\@tocpagenum{\ifnum#1=1\bfseries\fi#7}}\par% <-- \bfseries for \section page
    \nobreak
    \endgroup
  \fi}
\AtBeginDocument{%
\expandafter\renewcommand\csname r@tocindent0\endcsname{0pt}
}
\def\l@subsection{\@tocline{2}{0pt}{2.5pc}{5pc}{}}
\makeatother

\advance\footskip0.4cm
\textheight=54pc    %a4paper
\textheight=50.5pc %letterpaper
\advance\textheight-0.4cm
\calclayout

%Font settings
%\usepackage{anyfontsize}

%Footnote settings
\usepackage{footmisc}
%	\renewcommand*{\thefootnote}{\fnsymbol{footnote}}

%Further math environments
%Further math fonts (loads amsfonts implicitely)
\usepackage{amssymb}
%Redefinition of \text
%\usepackage{amstext}
\usepackage{upref}
%Graphics
%\usepackage{graphicx}
%\usepackage{caption}
%\usepackage{subcaption}
%Frames
\usepackage{mdframed}

\renewcommand{\Re}{\operatorname{Re}}
\renewcommand{\Im}{\operatorname{Im}}
\DeclareMathOperator\Log{Log}
\DeclareMathOperator\Arg{Arg}
\DeclareMathOperator\sech{sech}
%\usepackage{hhline}
%\usepackage{booktabs} 
%\usepackage{array}
%\usepackage{xfrac} 
%\everymath{\displaystyle}
%Enumerate
\usepackage{enumitem} 
%\renewcommand{\labelitemi}{$\bullet$}
%\renewcommand{\labelitemii}{$\ast$}
%\renewcommand{\labelitemiii}{$\cdot$}
%\renewcommand{\labelitemiv}{$\circ$}
%Colors
%\usepackage{color}
%\usepackage[cmtip, all]{xy}
%Theorems
\newtheoremstyle{bold}              	 %Name
  {}                                     %Space above
  {}                                     %Space below
  {\itshape}		                     %Body font
  {}                                     %Indent amount
  {\scshape}                             %Theorem head font
  {.}                                    %Punctuation after theorem head
  { }                                    %Space after theorem head, ' ', 
  										 %	or \newline
  {} 
\theoremstyle{bold}
\newtheorem{definition}{Definition}[section]
\newtheorem{lemma}{Lemma}[section]
\newtheorem{Proof}{Proof}[section]
\newtheorem{proposition}{Proposition}[section]
\newtheorem{properties}{Properties}[section]
\newtheorem{corollary}{Corollary}[section]
\newtheorem{theorem}{Theorem}[section]
\newtheorem{example}{Example}[section]
\newtheorem{remark}{Remark}[section]
%German non-ASCII-Characters
%Graphics-Tool
%\usepackage{tikz}
%\usepackage{tikzscale}
%\usepackage{bbm}
%\usepackage{bera}
%Listing-Setup
%Bibliographie
\usepackage[backend=bibtex, style=alphabetic]{biblatex}
%\usepackage[babel, german = swiss]{csquotes}
\bibliography{Bibliography}
%PDF-Linking
%\usepackage[hyphens]{url}
\usepackage[bookmarksopen=true,bookmarksnumbered=true]{hyperref}
%\PassOptionsToPackage{hyphens}{url}\usepackage{hyperref}
\hypersetup{
  colorlinks   = true, %Colours links instead of ugly boxes
  urlcolor     = blue, %Colour for external hyperlinks
  linkcolor    = blue, %Colour of internal links
  citecolor    = blue %Colour of citations
}
%Weierstrass-P symbol for power set
\newcommand{\powerset}{\raisebox{.15\baselineskip}{\Large\ensuremath{\wp}}}


\begin{document}
\title{Classical Fourier Analysis: Interpolation of $L^p$ Spaces}
\author{Yannis B\"{a}hni}
\address[Yannis B\"{a}hni]{University of Zurich, R\"{a}mistrasse 71, 8006 Zurich}
\email[Yannis B\"{a}hni]{\href{mailto:yannis.baehni@uzh.ch}{yannis.baehni@uzh.ch}}

\maketitle

\addtocounter{section}{1}

\begin{mdframed}
	\begin{definition}
		Let $(X,\mu)$ and $(Y,\nu)$ be measure spaces. Further let $T$ be an operator defined on a linear space of complex-valued measurable functions on $X$ and taking values in the set of all complex-valued, finite almost everywhere, measurable functions on $Y$. Then $T$ is called \emph{linear} if for all functions $f$ and $g$ in the domain of $T$ and all $z \in \mathbb{C}$ 

		\begin{equation*}
			T\left( f + g \right) = T(f) + T(g) \qquad T\left( zf \right) = zT(f)
		\end{equation*}

		holds and \emph{quasi-linear} if

		\begin{equation*}
			\left| T\left( f + g \right) \right| \leqslant K \left( \left| T(f)\right| + \left| T(g)\right| \right) \qquad \left| T(zf) \right| = \left| z\right| \left| T(f)\right|
		\end{equation*}

		holds for some constant $K > 0$. If $K = 1$, $T$ is called \emph{sublinear}.
	\end{definition}
\end{mdframed}

\vspace{2mm}

Suppose $\left(p_0,q_0\right),\left( p_1,q_1 \right) \in [1,\infty]\times [1,\infty]$ are two pairs of indices and assume that 

\begin{equation*}
	\left\|T(f)\right\|_{L^{q_0}} \leqslant M_0\left\|f\right\|_{L^{p_0}} \qquad \left\|T(f)\right\|_{L^{q_1}} \leqslant M_1\left\|f\right\|_{L^{p_1}}
\end{equation*}

where $T$ is a linear operator. Does this imply that

\begin{equation*}
	\left\| T(f)\right\|_{L^q} \leqslant M \left\| f\right\|_{L^p}
\end{equation*}

for other pairs $\left( p,q \right) \in [1,\infty]$? We shall investigate this question in the next theorem, but first we need to establish some terminology. For two measure spaces $\left( X,\mu \right)$, $\left( Y,\nu \right)$ let $\Sigma_X$ and $\Sigma_Y$ denote the set of all finitely simple functions on $X$, $Y$ respectively. 

\vspace{2mm}

\begin{mdframed}
	\begin{theorem}\emph{(Riesz-Thorin Interpolation Theorem)}
		Suppose that $(X,\mu)$, $(Y,\nu)$ are measure spaces and $1 \leqslant p_0,p_1,q_0,q_1 \leqslant \infty$. If $q_0 = q_1 = \infty$, suppose also that $\nu$ is semifinite. Let $T$ be a linear operator defined on $\Sigma_X$ and taking values in the set of measurable functions on $Y$, such that

		\begin{equation}
			\left\|T(f)\right\|_{L^{q_0}} \leqslant M_0\left\|f\right\|_{L^{p_0}} \qquad \left\|T(f)\right\|_{L^{q_1}} \leqslant M_1\left\|f\right\|_{L^{p_1}}
			\label{hyp:Lq0Lq1}
		\end{equation}

		for all $f \in \Sigma_X$ and $0 < M_0,M_1 < \infty$. Then for all $0 \leqslant \theta \leqslant 1$ we have

		\begin{equation}
			\left\|T(f)\right\|_{L^q} \leqslant M_0^{1 - \theta}M_1^\theta\left\|f\right\|_{L^p}
			\label{est:boundTf}
		\end{equation}

		for all $f \in \Sigma_X$, where

		\begin{equation*}
			\frac{1}{p} = \frac{1 - \theta}{p_0} + \frac{\theta}{p_1} \qquad \frac{1}{q} = \frac{1 - \theta}{q_0} + \frac{\theta}{q_1}.
		\end{equation*}
		\label{thm:Riesz_Thorin}
	\end{theorem}
\end{mdframed}

\vspace{2mm}

The proof of Hadamard's three lines lemma heavily relies on a restatement of the maximum modulus theorem (see \cite[212]{rudin:rc_analysis:1987}) found in \cite[253]{rudin:rc_analysis:1987}. To do so we have to first establish some common terminology. A complex-valued function $f$ is said to be \emph{holomorphic} in $\Omega \subseteq \mathbb{C}$ open, if $f'(z)$ exists for any $z \in \Omega$. By a region we shall mean a nonempty connected open subset of the complex plane. The restatement reads as follows.

\begin{theorem*}
	Let $\Omega \subseteq \mathbb{C}$ be a bounded region and $f$ be a continuous function on $\overline{\Omega}$ which is holomorphic in $\Omega$. Then 

	\begin{equation*}
		\left| f(z)\right| \leqslant \sup\left\{ \left|f(z) \right| : z \in \partial\Omega\right\}
	\end{equation*}

	for every $z \in \Omega$. If equality holds at one point $z \in \Omega$, then $f$ is constant.
\end{theorem*}

\begin{mdframed}
	\begin{lemma}\emph{(Hadamard's three lines lemma)}
		Let $F$ be a holomorphic function in the strip $S := \{z \in \mathbb{C}: 0 < \Re z < 1\}$, continuous and bounded on $\overline{S}$, such that $\left| F(z)\right| \leqslant B_0$ when $\Re z = 0$ and $\left| F(z) \right| \leqslant B_1$ when $\Re z = 1$, for some $0 < B_0,B_1 < \infty$. Then $\left| F(z) \right| \leqslant B_0^{1 - \theta}B_1^\theta$ when $\Re z = \theta$, for any $0 < \theta < 1$.
	\end{lemma}
\end{mdframed}

\begin{proof}
For $z \in \overline{S}$ define 

\begin{equation*}
	G(z) := \frac{F(z)}{B_0^{1 - z}B_1^z} \qquad G_n(z) := G(z) e^{\left(z^2 - 1\right)/n},~n \in\mathbb{N}_{>0}
\end{equation*}

$G(z)$ and $G_n(z)$ are holomorphic in $S$ by
	
\begin{equation*}
	G'(z) = \frac{F'(z) - F(z)\log\left( B_1/B_0 \right)}{B_0^{1 - z}B_1^z} \qquad G_n'(z) = G'(z)e^{\left( z^2 - 1 \right)/n} + \frac{2}{n}zG_n(z)	
\end{equation*}

and $e^z \neq 0$ for every $z \in \mathbb{C}$. Further, we have

\begin{equation*}
		\left| B_0^{1 - z}B_1^z \right| = \left(B_0^{1 - z}B_0^{1 - \overline{z}} B_1^z B_1^{\overline{z}}\right)^{1/2} =  B_0^{1 -\Re z}B_1^{\Re z}
\end{equation*}

Consider $0 \leqslant \Re z \leqslant 1$ and $B_0 \geqslant 1$. Then $B_0^{1 - \Re z} \geqslant 1$ and $B_0^{1 - \Re z } \geqslant B_0$ in the case $B_0 < 1$. Similarily, $B_1^{\Re z} \geqslant 1$ if $B_1 \geqslant 1$ and $B_1^{\Re z} \leqslant B_1$ if $B_1 < 1$. Hence 

\begin{equation}
	\left| B_0^{1 - z}B_1^z \right| \geqslant \min\left\{1,B_0\right\}\min\left\{1,B_1\right\} > 0
	\label{est:denom}
\end{equation}

for all $z \in \overline{S}$. Since $F$ is bounded on $\overline{S}$, we have $\left| F(z) \right| \leqslant L$ for some $L > 0$ and all $z \in \overline{S}$. Thus by (\ref{est:denom})

\begin{equation*}
	\left| G(z)\right| = \frac{\left| F(z)\right|}{\left| B_0^{1 - z}B_1^z \right|} \leqslant \frac{L}{\min\left\{1,B_0\right\}\min\left\{1,B_1\right\}} =: M
\end{equation*}

for every $z \in \overline{S}$. Fix $n \in \mathbb{N}_{>0}$ and write $z := x + iy \in \overline{S}$. Then

\begin{gather*}
	\left| G_n(z)\right| \leqslant M \left(e^{\left(x^2 + 2ixy -y^2 - 1\right)/n} e^{\left(x^2 - 2ixy -y^2 - 1\right)/n}\right)^{1/2}= M e^{-y^2/n}e^{\left(x^2 - 1\right)/n} \leqslant Me^{-y^2/n}
\end{gather*}

for $0 \leqslant x \leqslant 1$. Thus
	
\begin{equation*}
	\lim_{y \to \pm \infty}\sup\{\left| G_n(z)\right| : 0 \leqslant x \leqslant 1\} = 0
\end{equation*}

by the pinching-principle. Hence there exist $C_0,C_1 \in \mathbb{R}$, such that 

\begin{equation*}
	\sup\{\left| G_n(z)\right| : 0 \leqslant x \leqslant 1\} \leqslant 1
\end{equation*}

when $y > C_0$ or $y < C_1$. Letting

\begin{equation*}
	C(n) := \max\left\{ \left| C_0\right| + 1, \left| C_1 \right| + 1\right\}
\end{equation*}

we conclude $\left| G_n(z) \right| \leqslant 1$ for all $0 \leqslant x \leqslant 1$ when $\left| y \right| \geqslant C(n)$. Now consider the rectangle $R := \left(0,1\right) \times \left(-C(n),C(n)\right)$. We have $\left| G_n(z) \right| \leqslant 1$ on the lines $[0,1] \times \{\pm C(n)\}$. By

\begin{equation*}
	\left| G_n(iy)\right| = \frac{\left| F(iy)\right|}{\left| B_0^{1 - iy} B_1^{iy}\right|}e^{-\left( y^2 + 1 \right)/n} \leqslant 1 \qquad \left| G_n(1 + iy)\right| =	\frac{\left| F(1 + iy)\right|}{\left| B_0^{-iy}B_1^{1 + iy}\right|}e^{-y^2/n} \leqslant 1
\end{equation*}

we have $\left| G_n(z)\right| \leqslant 1$ on the lines $\{0\} \times [-C(n),C(n)]$, $\{1\} \times [-C(n),C(n)]$. Thus $\left| G_n(z) \right| \leqslant 1$ on $\partial R$. Since $\left| G_n(z)\right|$ is continuous on $\overline{R}$, holomorphic in $R$ and $R$ is a bounded region, the maximum modulus theorem implies

\begin{equation*}
	\left| G_n(z)\right| \leqslant \sup\left\{ \left|G_n(z) \right| : z \in \partial R \right\} \leqslant 1
\end{equation*}

for every $z \in R$. Therefore $\left| G_n(z) \right| \leqslant 1$ on $\overline{R}$ and so $\left| G_n(z) \right| \leqslant 1$ on $\overline{S}$. Since inequalities are preserved by limits and the modulus is a continuous function, we have that $\left| G(z) \right| = \lim_{n \to \infty} \left| G_n(z) \right| \leqslant 1$ for $z \in \overline{S}$. We conclude by 

\begin{equation*}
	\left| F(\theta + it) \right| = \left| G(\theta + it) \right| \left| B_0^{1 - \theta - it}B_1^{\theta + it}\right| \leqslant B_0^{1 - \theta} B_1^{\theta}
\end{equation*}

whenever $0 < \theta < 1$, $t \in \mathbb{R}$.
\end{proof}

%Remake Tikz-picture
%\tikzset{external/force remake}

\begin{figure}[h!tb]
	\centering
	\begin{tikzpicture}
		\draw (0,-5)--(0,5);
		\draw (4,-5)--(4,5);
		\draw (-1,0)--(5,0);
		\draw [pattern = adjusted lines, pattern color = black, line width = .4mm] (0,-4) rectangle (4,4);
		\node (0) at (-.25,-.25) {$0$};
		\node (1) at (4.25,-.25) {$1$};
		\node (R) at (.5,3.2) {$\overline{R}$};
		\node (Cn) at (-.5,4) {$C(n)$};
		\node (-Cn) at (-.65,-4) {$-C(n)$};
		\node (R2) at (4.5,4.5) {$\mathbb{R}^2$};
	\end{tikzpicture}	
	\caption{Sketch of the rectangle $\overline{R}$.}
	\label{fig:Hadamards_three_lines_lemma}
\end{figure}

\begin{proof}
	The idea is to bound the quantity (see \cite[189]{folland:real_analysis:1999})
	
	\begin{equation*}
		M_q\left( T(f) \right) = \sup\left\{ \left| \int_Y T(f)g d\nu\right| : g \in \Sigma_Y, \left\| g\right\|_{L^{q'}} = 1\right\} < \infty
	\end{equation*}

	appropriately.\\
	If either $\theta = 0$ or $\theta = 1$, the estimate (\ref{est:boundTf}) follows directly from the hypotheses (\ref{hyp:Lq0Lq1}) on $T$. Thus we may assume \underline{$0 < \theta < 1$}. Furthermore, if $f \in \Sigma_X$, $\left\| f \right\|_{L^p} = 0$, then $f = 0$ $\mu$-a.e. and either one of the hypotheses on $T$ in (\ref{hyp:Lq0Lq1}) implies $T(f) = 0$ $\mu$-a.e. and thus the estimate (\ref{est:boundTf}) holds trivially. Therefore we can assume \underline{$\left\| f\right\|_{L^p} \neq 0$}. Fix 
	
\begin{equation*}
	f :\equiv \sum_{j = 1}^n a_j e^{i\alpha_j}\chi_{A_j} \in \Sigma_X \qquad g :\equiv \sum_{k = 1}^m b_k e^{i\beta_k}\chi_{B_k} \in \Sigma_Y
\end{equation*}

where $a_j,b_k \neq 0$, $\alpha_j,\beta_k \in \mathbb{R}$ for any $j = 1,\dots,n$, $k = 1,\dots,m$, the sets $A_j$ and $B_k$ are each pairwise disjoint with $\mu\left( A_j \right),\nu\left( B_k \right) < \infty$ and so, that $\left\| g\right\|_{L^{q'}} \neq 0$ (recall $q' := q/\left( q - 1 \right)$). Define

\begin{equation*}
	P(z) := \frac{p}{p_0}(1 - z) + \frac{p}{p_1}z \qquad Q(z) := \frac{q'}{q'_0}(1 - z) + \frac{q'}{q'_1}z
\end{equation*}

for $z \in \mathbb{C}$ (since either $p = \infty$ implies $p_0 = p_1 = \infty$ or $q = 1$ implies $q_0 = q_1 = 1$, the functions $P$, $Q$ are well-defined). Further let
				
\begin{equation}
	f_z :\equiv \sum_{j = 1}^n a^{P(z)}_j e^{i\alpha_j}\chi_{A_j} \qquad g_z :\equiv  \sum_{k = 1}^m b^{Q(z)}_k e^{i\beta_k}\chi_{B_k}
	\label{def:fzgz}
\end{equation}
				
and 

\begin{equation}
	F(z) := \int_Y T(f_z)g_zd\nu
	\label{eq:def_F}
\end{equation}

By the linearity of the operator $T$ we have

\begin{gather*}
	F(z) = \sum_{j = 1}^n\sum_{k = 1}^m a^{P(z)}_j b_k^{Q(z)} e^{i\alpha_j} e^{i\beta_k} \int_YT(\chi_{A_j})\chi_{B_k}d\nu
\end{gather*}

and by H\"older's inequality

\begin{gather}
	\begin{aligned}
		\left| \int_YT(\chi_{A_j})\chi_{B_k}d\nu \right| &\leqslant \int_Y\left| T(\chi_{A_j})\chi_{B_k}\right|d\nu\\
		&= \left\|T(\chi_{A_j})\chi_{B_k}\right\|_{L^1}\\
		&\leqslant \left\|T(\chi_{A_j})\right\|_{L^{q_0}} \left\|\chi_{B_k}\right\|_{L^{q_0'}}\\
		&\leqslant M_0\left\|\chi_{A_j}\right\|_{L^{p_0}} \left\|\chi_{B_k}\right\|_{L^{q_0'}}\\
		&\leqslant M_0 \mu\left(A_j\right)^{1/p_0}\nu\left(B_k\right)^{1/q_0'}
	\end{aligned}
	\label{est:constant_F}
\end{gather}

for each $j = 1,\hdots,n$, $k = 1,\hdots,m$ (even in the cases where either $p_0 = \infty$ or $q_0' = \infty$, or both, by observing that $\left\| \chi_{A}\right\|_{L^\infty} \leqslant 1$ for any measurable set $A$). Thus the function $F$ is well-defined on $\mathbb{C}$. Let $t \in \mathbb{R}$. For $p,p_0 \neq \infty$

\begin{gather*}
	\begin{aligned}
		\left\|f_{it}\right\|_{L^{p_0}} &= \left(\sum_{j = 1}^n \int_{A_j} \left| f_{it} \right|^{p_0} d\mu + \int_{X \setminus \bigcup_{j = 1}^n A_j} \left| f_{it} \right|^{p_0} d\mu\right)^{1/p_0}\\
		&= \left(\sum_{j = 1}^n \left| a_j^{P(it)} e^{i\alpha_j}\right|^{p_0}\int_X \chi_{A_j} d\mu\right)^{1/p_0}\\
		&= \left(\sum_{j = 1}^n a_j^{p_0\Re P(it)}\mu\left(A_j\right)\right)^{1/p_0}\\
		&= \left(\sum_{j = 1}^n a_j^p\mu\left(A_j\right)\right)^{p/\left(p_0p\right)}\\
		&= \left\|f\right\|_{L^p}^{p/p_0} 
	\end{aligned}
\end{gather*}

holds. Let $p_0 = \infty$, $p \neq \infty$. Then $\left\|f_{it}\right\|_{L^{\infty}} = 1$ since $\left| a_j^{P(it)}\right| = a_j^{p/p_0} = 1$ and that there exists some index $j$, such that $\mu\left( A_j \right) \neq 0$. If $p = \infty$, then $p_0 = p_1 = \infty$ and thus $P(it) = 1$. By the same considerations we have $\|g_{it}\|_{L^{q_0'}} = \|g\|_{L^{q'}}^{q'/q'_0}$. Hence

\begin{gather*}
	\begin{aligned}
		\left| F(it) \right| &\leqslant \int_Y \left| T(f_{it})g_{it}\right| d\nu\\
		&= \left\|T(f_{it}) g_{it}\right\|_{L^1}\\
		&\leqslant \left\|T(f_{it})\right\|_{L^{q_0}}\left\|g_{it}\right\|_{L^{q_0'}}\\
		&\leqslant M_0 \left\|f_{it}\right\|_{L^{p_0}} \left\|g_{it}\right\|_{L^{q_0'}}\\
		&= M_0 \left\|f\right\|_{L^p}^{p/p_0} \left\|g\right\|_{L^{q'}}^{q'/q'_0}
	\end{aligned}
\end{gather*}

by H\"older's inequality. In an analogous manner we derive
				
\begin{equation*}
	\left\|f_{1 + it}\right\|_{L^{p_1}} = \left\|f\right\|_{L^p}^{p/p_1} \qquad \left\|g_{1 + it}\right\|_{L^{q_1'}} = \left\|g\right\|_{L^{q'}}^{q'/q_1'}
\end{equation*}

and thus 
				
\begin{equation*}
	\left| F(1 + it)\right| \leqslant M_1 \left\|f\right\|_{L^p}^{p/p_1}\left\|g\right\|_{L^{q'}}^{q'/q_1'}
\end{equation*}	

Further by estimate (\ref{est:constant_F}) 

\begin{equation*}
	\begin{aligned}
		\left| F(z)\right| &\leqslant \sum_{j = 1}^n\sum_{k = 1}^m \left| a_j^{P(z)}\right| \left| b_k^{Q(z)}\right| \left| \int_Y T(\chi_{A_j})\chi_{B_k} d\nu\right|\\
		&\leqslant M_0\sum_{j = 1}^n\sum_{k = 1}^m a_j^{\Re P(z)}b_k^{\Re Q(z)}\mu\left(A_j\right)^{1/p_0}\nu\left(B_k\right)^{1/q_0'}\\
		&\leqslant M_0\sum_{j = 1}^n\sum_{k = 1}^m \max\left\{ 1, a_j^{p/p_0 + p/p_1}\right\} \max\left\{1,b_k^{q'/q_0' + q'/q_1'}\right\}\mu\left(A_j\right)^{1/p_0}\nu\left(B_k\right)^{1/q_0'}
	\end{aligned}
\end{equation*}


Hence $F$ is bounded on $\overline{S}$ by some constant depending on $f$ and $g$ only. By 

\begin{multline*}
	F'(z) = \sum_{j = 1}^n\sum_{k = 1}^m a^{P(z)}_j\log \left( a_j \right) \left( \frac{p}{p_1} - \frac{p}{p_0} \right) b_k^{Q(z)}e^{i\alpha_j} e^{i\beta_k} \int_YT(\chi_{A_j})\chi_{B_k}d\nu \\
	+  \sum_{j = 1}^n\sum_{k = 1}^m a^{P(z)}_jb_k^{Q(z)}\log\left( b_k \right)\left( \frac{q'}{q'_1} - \frac{q'}{q'_0} \right)e^{i\alpha_j} e^{i\beta_k} \int_YT(\chi_{A_j})\chi_{B_k}d\nu 
\end{multline*}

	it is immediate, that $F$ is an entire function and thus holomorphic in $S$ and continuous on $\overline{S}$. Therefore, Hadamard's three lines lemma yields

\begin{gather*}
	\left| F(z) \right| \leqslant \left( M_0  \left\|f\right\|_{L^p}^{p/p_0} \left\|g\right\|_{L^{q'}}^{q'/q'_0} \right)^{1 - \theta}\left(  M_1 \left\|f\right\|_{L^p}^{p/p_1}\left\|g\right\|_{L^{q'}}^{q'/q_1'} \right)^\theta = M_0^{1 - \theta}M_1^\theta \left\|f\right\|_{L^p}\left\|g\right\|_{L^{q'}}
\end{gather*}

for $\Re z = \theta$, $0 < \theta < 1$. We have

\begin{equation*}
	\left\{ T(f) \neq 0\right\} = \bigcup_{n = 1}^\infty \left\{ \left| T(f)\right| > 1/n\right\}
\end{equation*}

and by Chebychev's inequality either

\begin{equation*}
	\nu\left( \left\{ \left| T(f)\right| > 1/n\right\} \right) \leqslant n^{q_0}\left\| T(f)\right\|_{L^{q_0}}^{q_0}
\end{equation*}

or

\begin{equation*}
	\nu\left( \left\{ \left| T(f)\right| > 1/n\right\} \right) \leqslant n^{q_1}\left\| T(f)\right\|_{L^{q_1}}^{q_1}
\end{equation*}

whenever $q_0 \neq \infty$ or $q_1 \neq \infty$. Therefore, the set $\left\{ T(f) \neq 0\right\}$ is $\sigma$-finite unless $q_0 = q_1 = \infty$. Further we have $P(\theta) = Q(\theta) = 1$. Thus by

\begin{gather*}
	\begin{aligned}
		M_q\left( T(f) \right) &= \sup\left\{\left| \int_Y T(f)gd\nu\right| : g \in \Sigma_Y, \left\|g\right\|_{L^{q'}} = 1\right\}\\
		&=  \sup\left\{\left| F(\theta)\right| : g \in \Sigma_Y, \left\|g\right\|_{L^{q'}} = 1\right\}\\
		&\leqslant M_0^{1 - \theta}M_1^\theta \left\|f\right\|_{L^p}\\
	\end{aligned}
\end{gather*}

we conclude 
	
\begin{equation*}
	\left\| T(f)\right\|_{L^q} = M_q\left( T(f) \right) \leqslant M_0^{1 - \theta}M_1^\theta \left\|f\right\|_{L^p}
\end{equation*}
	
for any $f \in \Sigma_X$.
\end{proof}

\begin{remark*}
	A more general version of the Riesz-Thorin interpolation theorem can be found in \textup{\cite[200--202]{folland:real_analysis:1999}}. There, a linear map 

	\begin{equation*}
		T: L^{p_0} + L^{p_1} \rightarrow L^{q_0} + L^{q_1}
	\end{equation*}

	is considered. This follows using a density argument from the current version of the theorem and will not be prooven here.
	\label{rem:extension}
\end{remark*}

\begin{remark*}
	Using the previous remark, a standard application of the Riesz-Thorin interpolation theorem is to proove Young's inequality \textup{\cite[22--23]{grafakos:fourier:2014}}.
\end{remark*}

\begin{mdframed}
	\begin{definition}\emph{(Analytic family, admissible growth)}
		Let $(X,\mu)$, $(Y,\nu)$ be two $\sigma$-finite measure spaces and for every $z \in \overline{S}$ we have an associated linear operator $T_z$ which is defined on $\Sigma_X$ and taking values in the space of all measurable functions on $Y$ such that

		\begin{equation*}
			\int_Y \left| T_z(\chi_A)\chi_B \right| d\nu < \infty
		\end{equation*}

		whenever $\mu(A),\nu(B) < \infty$. The family $\left( T_z \right)_{z \in \overline{S}}$ is said to be \emph{analytic} if for all $f \in \Sigma_X$, $g \in \Sigma_Y$ we have that

		\begin{equation*}
			z \mapsto \int_Y T_z(f)gd\nu
		\end{equation*}

		is analytic in $S$ and continuous on $\overline{S}$. Further, an analytic family $\left( T_z \right)_{z \in \overline{S}}$ is called of \emph{admissible growth}, if there is a constant $\tau_0 \in (0,\pi)$, such that for all $f \in \Sigma_X$, $g \in \Sigma_Y$ a constant $0 < C(f,g) < \infty$ exists with

			\begin{equation*}
				\log\left| \int_Y T_z(f) g d\nu\right| \leqslant C(f,g)e^{\tau_0\left| \Im z\right|}
			\end{equation*}

			for all $z \in \overline{S}$.
	\end{definition}
\end{mdframed}

\vspace{2mm}

\begin{mdframed}
	\begin{theorem}\emph{(Stein's Theorem on Interpolation of Analytic Families of Operators)}
		Let $\left( T_z \right)_{z \in \overline{S}}$ be an analytic family of admissible growth, $1 \leqslant p_0,p_1,q_0,q_1 \leqslant \infty$ and suppose that $M_0$, $M_1$ are positive functions on the real line such that for some $\tau_1 \in (0,\pi)$

			\begin{equation}
				\sup\left\{e^{-\tau_1 \vert y \vert} \log M_0(y) : y \in \mathbb{R}\right\} < \infty \qquad \sup\left\{e^{-\tau_1 \vert y \vert} \log M_1(y) : y \in \mathbb{R}\right\} < \infty
			\end{equation}

			Fix $0 < \theta < 1$ and define

			\begin{equation}
				\frac{1}{p} := \frac{1 - \theta}{p_0} + \frac{\theta}{p_1} \qquad \frac{1}{q} := \frac{1 - \theta}{q_0} + \frac{\theta}{q_1}
			\end{equation}

			Further suppose that for all $f \in \Sigma_X$ and $y \in \mathbb{R}$ we have

			\begin{equation}
				\|T_{iy}(f)\|_{L^{q_0}} \leqslant M_0(y)\|f\|_{L^{p_0}} \qquad \|T_{1 + iy}(f)\|_{L^{q_1}} \leqslant M_1(y)\|f\|_{L^{p_1}} 
			\end{equation}

			Then for all $f \in \Sigma_X$ we have

			\begin{equation*}
				\|T_\theta(f)\|_{L^q} \leqslant M(\theta)\|f\|_{L^p}
			\end{equation*}

			where for $0 < x < 1$

			\begin{equation*}
				M(x) = \exp\left( \frac{\sin(\pi x)}{2} \int_{-\infty}^\infty \left[ \frac{\log M_0(t)}{\cosh(\pi t) - \cos(\pi x)} + \frac{\log M_1(t)}{\cosh(\pi t) + \cos(\pi x)}\right] d\lambda(t) \right)
			\end{equation*}
	\end{theorem}
\end{mdframed}

\vspace{2mm}

\begin{mdframed}
	\begin{lemma}\emph{(Hadamard's three lines lemma, extension)}
		Let $F$ be a holomorphic function in the strip $S := \{z \in \mathbb{C}: 0 < \mathrm{Re}z < 1\}$ and continuous on $\overline{S}$, such that for some $0 < A < \infty$ and $\tau_0 \in (0,\pi)$ we have $\log \vert F(z)\vert \leqslant A e^{\tau_0 \vert \Im z \vert}$ for every $z \in \overline{S}$. Then

			\begin{equation*}
				\vert F(z) \vert \leqslant \exp\left( \frac{\sin(\pi x)}{2} \int_{-\infty}^\infty \left[ \frac{\log \vert F(it + iy)\vert}{\cosh(\pi t) - \cos(\pi x)} + \frac{\log \vert F(1 + it + iy)\vert}{\cosh(\pi t) + \cos(\pi x)} \right] d\lambda(t)\right)
			\end{equation*}

			whenever $z := x + iy \in S$.
			\label{lem:EHTL}
	\end{lemma}
\end{mdframed}

\begin{proof}
	As mentioned in \href{https://terrytao.wordpress.com/2011/05/03/steins-interpolation-theorem/}{Terence Tao's blog}, Fefferman once noted, that this proof can be obtained from that of the Riesz-Thorin theorem \ref{thm:Riesz_Thorin} simply by adding a single letter of the alphabet. Indeed, this is truly the case, since all hypotheses made in the theorem incorporate the same proof as in the Riesz-Thorin theorem. The only heavy and technical part is the proof of the extension of Hadamard's three lines lemma \ref{lem:EHTL}.
\end{proof}

\begin{mdframed}
	\begin{theorem}\emph{(The Marcinkiewicz Interpolation Theorem)}
		Let $(X,\mu)$ be a $\sigma$-finite measure space, $(Y,\nu)$ another measure space and $0 < p_0 < p_1 \leqslant \infty$. Further let $T$ be a sublinear operator defined on
		
		\begin{equation*}
			L^{p_0} + L^{p_1} := \left\{ f_0 + f_1 : f_0 \in L^{p_0}(X,\mu), f_1 \in L^{p_1}(X,\mu) \right\}
		\end{equation*}
		
		and taking values in the space of measurable functions on $Y$. Assume that there exist $0 < A_0,A_1 < \infty$ such that

		\begin{align}
			&\forall f \in L^{p_0}(X,\mu)~\|T(f)\|_{L^{p_0,\infty}} \leqslant A_0 \|f\|_{L^{p_0}}\label{hyp:fp_0}\\
			&\forall f \in L^{p_1}(X,\mu)~\|T(f)\|_{L^{p_1,\infty}} \leqslant A_1 \|f\|_{L^{p_1}}\label{hyp:fp_1}
		\end{align}

		Then for all $p_0 < p < p_1$ and for all $f \in L^p(X,\mu)$ we have the estimate

		\begin{equation}
			\left\|T(f)\right\|_{L^p} \leqslant A \left\|f\right\|_{L^p}
		\end{equation}

		where

		\begin{equation}
			A := 2\left( \frac{p}{p - p_0} + \frac{p}{p_1 - p} \right)^{1/p}A_0^{\frac{\frac{1}{p} - \frac{1}{p_1}}{\frac{1}{p_0}-\frac{1}{p_1}}}A_1^{\frac{\frac{1}{p_0}-\frac{1}{p}}{\frac{1}{p_0}-\frac{1}{p_1}}}
			\label{eq:constant}
		\end{equation}
	\end{theorem}
\end{mdframed}
\end{document}

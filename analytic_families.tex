\section{Interpolation of Analytic Families of Operators}
This generalization of the classical Riesz-Thorin theorem is due to \href{http://www.ams.org/journals/tran/1956-083-02/S0002-9947-1956-0082586-0/S0002-9947-1956-0082586-0.pdf}{Elias M. Stein}. Crucial for its proof is again a complex-analytic theorem which can be extended on the basis of Hadamard's three lines lemma.

\subsection{Extension of Hadamard's Three Lines Lemma}
This theorem is analogous to the one originally used by Stein itself and formulated by \href{http://download.springer.com/static/pdf/285/art\%253A10.1007\%252FBF02825637.pdf?originUrl=http\%3A\%2F\%2Flink.springer.com\%2Farticle\%2F10.1007\%2FBF02825637\&token2=exp=1470939579~acl=\%2Fstatic\%2Fpdf\%2F285\%2Fart\%25253A10.1007\%25252FBF02825637.pdf\%3ForiginUrl\%3Dhttp\%253A\%252F\%252Flink.springer.com\%252Farticle\%252F10.1007\%252FBF02825637*~hmac=d88bfe05b2cc8b0deed0f4781b8dfdd3701969606a6033727bfaf0c034cbd876}{ I. I. Hirschman, Jr.}

\vspace{2mm}

\begin{mdframed}
	\begin{lemma}\emph{(Hadamard's three lines lemma, extension)}
		Let $F$ be an analytic function in the strip $S := \{z \in \mathbb{C}: 0 < \mathrm{Re}z < 1\}$ and continuous on $\overline{S}$, such that for every $z \in \overline{S}$ we have $\log \vert F(z)\vert \leqslant A e^{\tau \vert \mathrm{Im}z \vert}$ for some $A < \infty$ and $\tau \in [0,\pi[$. Then

			\begin{equation*}
				\vert F(z) \vert \leqslant \exp\left( \frac{\sin(\pi x)}{2} \int_{-\infty}^\infty \left[ \frac{\log \vert F(it + iy)\vert}{\cosh(\pi t) - \cos(\pi x)} + \frac{\log \vert F(1 + it + iy)\vert}{\cosh(\pi t) + \cos(\pi x)} \right] d\lambda(t)\right)
			\end{equation*}

			whenever $z := x + iy \in S$.
	\end{lemma}
\end{mdframed}

\begin{proof}
	Since the proof is rather long we divide it as follows.

	\begin{enumerate}[label = \textbf{(\roman*.)}]
		\item Construct a holomorphic function $h$ defined on the open unit disc $D := \{z \in \mathbb{C} : \vert z \vert < 1\}$ with range $S$.
		\item Using the Poisson integral formula and the maximum principle for subharmonic functions find an upper bound for $\log\vert F \circ h\vert$ for $D$.
	\end{enumerate}

	\begin{enumerate}[label = \textbf{(\roman*.)}]
		\item Assume $F$ not identically zero (the case where $F$ is identically zero is trivial). Consider the function 

	\begin{equation}
		h(z) := \frac{1}{\pi i}\Log\left( i\frac{1 + z}{1 - z} \right)
	\end{equation}

	on $D$. Define $\psi(z) := i(1 + z)/(1 - z)$. If we write $z := x + iy \in D$, we have 

	\begin{equation}
		\psi(z) = \frac{-2y}{(1 - x)^2 + y^2} + i \frac{1 - x^2 - y^2}{(1 - x)^2 + y^2}
	\end{equation}

	Hence $\Im \psi(z) > 0$. Stipulating $x := 1 - y$ for $y$ satisfying $y^2 < y$, we get

	\begin{equation}
		\lim\limits_{y^2 < y, y \rightarrow 0^+} \Im \psi(z) = \lim\limits_{y^2 < y, y \rightarrow 0^+} \left( \frac{1}{y} - 1 \right) = \infty
	\end{equation}

	using the same definition of $x$ we get

	\begin{equation}
		\lim\limits_{y^2 < y, y \rightarrow 0^+} \Re \psi(z) = -\lim\limits_{y^2 < y, y \rightarrow 0^+} \frac{1}{y} = -\infty
	\end{equation}

	and by stipulating $x := 1 + y$

	\begin{equation}
		\lim\limits_{y^2 < -y, y \rightarrow 0^-} \Re \psi(z) = -\lim\limits_{y^2 < -y, y \rightarrow 0^-} \frac{1}{y} = \infty	
	\end{equation}


	Since $2i \neq 0$, $\psi$ is a linear fractional transformation (see \cite[279]{rudin:rc_analysis:1987}) with

	\begin{equation}
		\psi^{-1}(z) = \frac{z - i}{z + i}
	\end{equation}

	Therefore $\psi$ maps the unit circle $D$ onto the upper half plane. The principal value of $\log z$ denoted by $\Log z$ is defined by 

	\begin{equation}
		\Log z := \log \vert z \vert + i \Arg z \qquad z \neq 0
	\end{equation}

where $-\pi < \mathrm{Arg} z \leqslant \pi$ is the principal value of the argument of $z \neq 0$.  We see that $\pi i h(z)$ maps the upper half plane onto the strip $\mathbb{R} \times ]0,\pi[$. Thus $h(z)$ maps the unit circle $D$ onto the strip $]0,1[ \times \mathbb{R}$. By

\begin{equation}
	h'(z) = \frac{2}{\pi i} \frac{1}{z - 1}
\end{equation}

we see that $h$ is a holomorphic function on $D$. By composition, $F \circ h$ is holomorphic on $D$ and thus by \cite[336]{rudin:rc_analysis:1987} $\log\vert F \circ h \vert$ is subharmonic on $D$. It is easy to verify, that 

\begin{equation}
	h^{-1}(z) = \frac{e^{\pi i z} - i}{e^{\pi i z} + i}
\end{equation}
 on the unit strip $S$.\\

 \item Fix some $0 \leqslant R < 1$. Then $\log\vert F \circ h \vert$ is continuous (as the sum, product, quotient, composition of continuous functions) for $\vert z \vert = R$. Define 

\begin{gather*}
	\begin{aligned}
		H(re^{i\theta}):= \begin{cases}
			\displaystyle
			\log \vert F(h(Re^{i\theta}))\vert & r = R,\\
			\displaystyle
			\frac{1}{2\pi} \int_{-\pi}^\pi \log\vert F(h(Re^{it}))\vert \frac{R^2 - r^2}{R^2 - 2Rr\cos(\theta - t) + r^2} d\lambda(t) & 0 \leqslant r < R
	\end{cases}
	\end{aligned}
\end{gather*}

Then $H$ is continuous for $\vert z \vert \leqslant R$ and harmonic for $\vert z \vert < R$ (see \cite[234--235]{rudin:rc_analysis:1987}). Since $\log\vert F(h(Re^{i\theta}))\vert = H(Re^{i\theta})$, by \cite[336]{rudin:rc_analysis:1987} we have

\begin{equation}
	\log\vert F(h(re^{i\theta})) \leqslant \frac{1}{2\pi} \int_{-\pi}^\pi \log\vert F(h(Re^{it}))\vert \frac{R^2 - r^2}{R^2 - 2Rr\cos(\theta - t) + r^2} d\lambda(t) 
\end{equation}

Consider $e^{i\theta}$ where $\Arg e^{i\theta} \neq 0,\pi$, we have $\Im \psi(e^{i\theta}) = 0$ and hence $\psi(e^{i\theta}) \in \mathbb{R}$. But then either $\Re h(e^{i\theta}) = 0$, $\psi(e^{i\theta}) > 0$ or $\Re h(e^{i\theta}) = 1$, $\psi(e^{i\theta}) < 0$. Hence the growth property of the hypothesis implies

\begin{gather}
	\begin{aligned}
		\log \vert F(h(e^{i\theta})) \vert &\leqslant Ae^{\tau\vert \Im h(e^{i\theta})\vert}\\
		&= Ae^{\tau/\pi\vert \log\vert (1 + e^{i\theta})(1 - e^{i\theta})\vert\vert}\\
		&= A \left\vert \frac{1 + e^{i\theta}}{1 - e^{i\theta}} \right\vert^{\tau/\pi}
	\end{aligned}
\end{gather}

\item Fix some $re^{i\theta}$, $r < R$ and stipulate $x := h(re^{i\theta})$. Then we obtain

	\begin{gather}
		\begin{aligned}
			re^{i\theta} =& h^{-1}(x)\\
			=& \frac{e^{\pi i x}- i}{e^{\pi i x} + i}\\
			=& \frac{\cos(\pi x) + i\sin(\pi x) - i}{\cos(\pi x) + i\sin(\pi x) + i}\\
			=& \frac{\cos(\pi x) + i\sin(\pi x) - i}{\cos(\pi x) + i\sin(\pi x) + i}\frac{\cos(\pi x) - i\sin(\pi x) - i}{\cos(\pi x) - i\sin(\pi x) + i}\\
			=& -i \frac{\cos(\pi x)}{1 + \sin(\pi x)}
		\end{aligned}
	\end{gather}

	by

	\begin{gather*}
		\begin{aligned}
			\left(\cos(\pi x) + i\sin(\pi x) - i\right)\left(\cos(\pi x) - i\sin(\pi x) - i\right) =& \cos^2(\pi x) - i\sin(\pi x)\cos(\pi x)\\	
			& -i\cos(\pi x) + i\sin(\pi x)\cos(\pi x)\\
			& + \sin^2(\pi x) + \sin(\pi x)\\
			& -i \cos(\pi x) - \sin(\pi x) -1\\
			=& -2i \cos(\pi x)  
		\end{aligned}
	\end{gather*}

	and

	\begin{gather*}
		\begin{aligned}
			\left( \cos(\pi x) + i\sin(\pi x) + i \right)\left( \cos(\pi x) - i\sin(\pi x) - i \right) =& \cos^2(\pi x) - i\sin(\pi x)\cos(\pi x)\\
			& -i\cos(\pi x) + i\sin(\pi x)\cos(\pi x)\\
			& + \sin^2(\pi x) + \sin(\pi x)\\
			& + i \cos(\pi x) + \sin(\pi x) + 1\\
			=& 2 + 2\sin(\pi x)
		\end{aligned}
	\end{gather*}


	\end{enumerate}
\end{proof}

\subsection{Stein's Theorem on Interpolation of Analytic Families of Operators}

\begin{mdframed}
	\begin{definition}\emph{(Analytic family, admissible growth)}
		Let $(X,\mu)$, $(Y,\nu)$ be measure spaces and $\left( T_z \right)_{z \in \overline{S}}$, where $T_z$ is defined on the space of all finitely simple functions on $X$ and taking values in the space of all measurable functions on $Y$ such that

		\begin{equation}
			\int_Y \vert T_z(\chi_A)\chi_B \vert d\nu
		\end{equation}

		whenever $\mu(A),\nu(B) < \infty$. The family $\left( T_z \right)_{z \in \overline{S}}$ is said to be \emph{analytic} if for all $f$, $g$ finitely simple we have that

		\begin{equation}
			z \mapsto \int_Y T_z(f)gd\nu
		\end{equation}

		is analytic on $S$ and continuous on $\overline{S}$. Further, an analytic family $\left( T_z \right)_{z \in \overline{S}}$ is called of \emph{admissible growth}, if there is a constant $\tau \in [0,\pi[$, such that for all finitely simple functions $f$, $g$ a constant $C(f,g)$ exists with

			\begin{equation}
				\log\left\vert \int_Y T_z(f) g d\nu\right\vert \leqslant C(f,g)e^{\tau\vert \mathrm{Im}z\vert}
			\end{equation}

			for all $z \in \overline{S}$.
	\end{definition}
\end{mdframed}

\vspace{2mm}

\begin{mdframed}
	\begin{theorem}\emph{(Riesz-Thorin interpolation theorem, extension)}
		Let $\left( T_z \right)_{z \in \overline{S}}$ be an analytic family of admissible growth, $1 \leqslant p_0,p_1,q_0,q_1 \leqslant \infty$ and suppose that $M_0$, $M_1$ are positive functions on the real line such that for some $\tau \in [0,\pi[$

			\begin{equation}
				\sup\left\{e^{-\tau \vert y \vert} \log M_0(y) : y \in \mathbb{R}\right\} < \infty \qquad \sup\left\{e^{-\tau \vert y \vert} \log M_1(y) : y \in \mathbb{R}\right\} < \infty
			\end{equation}

			Fix $0 < \theta < 1$ and define

			\begin{equation}
				\frac{1}{p} := \frac{1 - \theta}{p_0} + \frac{\theta}{p_1} \qquad \frac{1}{q} := \frac{1 - \theta}{q_0} + \frac{\theta}{q_1}
			\end{equation}

			Further suppose that for all finitely simple functions $f$ on $X$ and $y \in \mathbb{R}$ we have

			\begin{equation}
				\|T_{iy}(y)\|_{L^{q_0}} \leqslant M_0(y)\|f\|_{L^{p_0}} \qquad \|T_{1 + iy}(y)\|_{L^{q_1}} \leqslant M_1(y)\|f\|_{L^{p_1}} 
			\end{equation}

			Then for all finitely simple functions $f$ on $X$ we have

			\begin{equation*}
				\|T_\theta(f)\|_{L^q} \leqslant M(\theta)\|f\|_{L^p}
			\end{equation*}

			where for $0 < x < 1$

			\begin{equation*}
				M(x) = \exp\left( \frac{\sin(\pi x)}{2} \int_{-\infty}^\infty \left[ \frac{\log M_0(t)}{\cosh(\pi t) - \cos(\pi x)} + \frac{\log M_1(t)}{\cosh(\pi t) + \cos(\pi x)}\right] d\lambda(t) \right)
			\end{equation*}
	\end{theorem}
\end{mdframed}

\begin{proof}
	Fix $0 < \theta < 1$ and finitely simple functions $f$, $g$ on $X$, $Y$ respectively with $\|f\|_{L^p} = \|g\|_{L^{q'}} = 1$. Define $f_z$, $g_z$ as in (\ref{def:fzgz}) and for $z \in \overline{S}$

	\begin{equation}
		F(z) := \int_Y T_z(f_z)g_z d\nu	
	\end{equation}

	Observe, that $\vert a^{P(z)}_j\vert \leqslant a_j^{p/p_0 + p/p_1}$ and $\vert b^{Q(z)}_k\vert \leqslant b_k^{q'/q'_0 + q'/q'_1}$ for $z \in \overline{S}$. Hence

	\begin{gather}
		\begin{aligned}
			\log \vert F(z) \vert &= \log \left\vert \sum_{j = 1}^n\sum_{k = 1}^m a^{P(z)}_j b_j^{Q(z)} e^{i\alpha_j} e^{i\beta_k} \int_YT_z(\chi_{X_j})(y)\chi_{Y_k}(y)d\nu(y)\right\vert\\
			&\leqslant \log \left( \sum_{j = 1}^n\sum_{k = 1}^m \vert a^{P(z)}_j\vert \vert b_j^{Q(z)}\vert \int_Y\vert T_z(\chi_{X_j})(y)\vert \chi_{Y_k}(y)d\nu(y)\right)\\
			&\leqslant  \log \left( \sum_{j = 1}^n\sum_{k = 1}^m a_j^{p/p_0 + p/p_1} b_k^{q'/q'_0 + q'/q'_1} \int_{Y_k}\vert T_z(\chi_{X_j})\vert d\nu\right)\\
		\end{aligned}
	\end{gather}
\end{proof}

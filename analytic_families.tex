\section{Interpolation of Analytic Families of Operators}
This generalization of the classical Riesz-Thorin theorem is due to \href{http://www.ams.org/journals/tran/1956-083-02/S0002-9947-1956-0082586-0/S0002-9947-1956-0082586-0.pdf}{Elias M. Stein}. Crucial for its proof is again a complex-analytic theorem which can be extended on the basis of Hadamard's three lines lemma.

\subsection{Extension of Hadamard's Three Lines Lemma}
This theorem is analogous to the one originally used by Stein itself and formulated by \href{http://download.springer.com/static/pdf/285/art\%253A10.1007\%252FBF02825637.pdf?originUrl=http\%3A\%2F\%2Flink.springer.com\%2Farticle\%2F10.1007\%2FBF02825637\&token2=exp=1470939579~acl=\%2Fstatic\%2Fpdf\%2F285\%2Fart\%25253A10.1007\%25252FBF02825637.pdf\%3ForiginUrl\%3Dhttp\%253A\%252F\%252Flink.springer.com\%252Farticle\%252F10.1007\%252FBF02825637*~hmac=d88bfe05b2cc8b0deed0f4781b8dfdd3701969606a6033727bfaf0c034cbd876}{ I. I. Hirschman, Jr.}

\subsubsection{Auxiliary Lemmata} To shorten the proof of the extension of Hadamard's three lines lemma, I will summarize the most important facts used during the proof.

\begin{lemma}
	Let $D := \{z \in \mathbb{C} : \vert z \vert < 1\}$ be the open unit disc and 
	
	\begin{equation}
		h(z) := \frac{1}{\pi i}\log\left( i\frac{1 + z}{1 - z} \right)
	\end{equation}

	for $z \in D$ where we shall interpret $\log z := \log \vert z \vert + i \arg z$ as the principal value, this means $-\pi < \arg z \leqslant \pi$. Then $h$ is a holomorphic function which maps $D$ bijectively onto the strip $S := \{z \in \mathbb{C} : 0 < \Re z < 1\}$.

	\label{lem:h}
\end{lemma}

\begin{proof}
	Define $\displaystyle f(z) := i\frac{1 + z}{1 - z}$. If we write $z := x + iy \in D$, we have 

	\begin{equation}
		f(z) = \frac{-2y}{(1 - x)^2 + y^2} + i \frac{1 - x^2 - y^2}{(1 - x)^2 + y^2}
		\label{eq:rep_lft}
	\end{equation}

	Hence $\Im f(z) > 0$ on $D$. Stipulating $x := 1 - y$ for $y$ satisfying $y^2 < y$, we get

	\begin{equation}
		\lim\limits_{y^2 < y, y \rightarrow 0^+} \Im f(z) = \lim\limits_{y^2 < y, y \rightarrow 0^+} \left( \frac{1}{y} - 1 \right) = \infty
	\end{equation}

	using the same definition of $x$ we get

	\begin{equation}
		\lim\limits_{y^2 < y, y \rightarrow 0^+} \Re f(z) = -\lim\limits_{y^2 < y, y \rightarrow 0^+} \frac{1}{y} = -\infty
	\end{equation}

	and by stipulating $x := 1 + y$

	\begin{equation}
		\lim\limits_{y^2 < -y, y \rightarrow 0^-} \Re f(z) = -\lim\limits_{y^2 < -y, y \rightarrow 0^-} \frac{1}{y} = \infty	
	\end{equation}


	Since $2i \neq 0$, $f$ is a linear fractional transformation (see \cite[279]{rudin:rc_analysis:1987}) with

	\begin{equation}
		f^{-1}(z) = \frac{z - i}{z + i}
	\end{equation}

	Therefore $f$ maps the unit circle $D$ onto the upper half plane $\{z \in \mathbb{C} : \Im z > 0\}$. The preceeding logarithm maps the upper half plane onto the strip $\{z \in \mathbb{C} : 0 < \Im z < \pi \}$. Thus $h(z)$ maps the unit circle $D$ onto the strip $S$. By

\begin{equation}
	h'(z) = \frac{2}{\pi i} \frac{1}{1 - z}
	\label{eq:h_diff}
\end{equation}

	we see that $h$ is a holomorphic function in $D$. 
\end{proof}

\begin{lemma}
	The mapping $\Phi: \mathbb{R} \rightarrow (-\pi,0)$ defined by $\Phi(t) := -i\log\left( h^{-1}(it) \right)$ is a $C^1$-Diffeomorphism with $\vert D\Phi(t)\vert = \pi \sech(\pi t)$. In an analogous manner we have that $\Psi: \mathbb{R} \rightarrow (0,\pi)$, $\Psi(t) := -i\log\left( h^{-1}(1 + it) \right)$ is a $C^1$-Diffeomorphism with $\vert D\Psi(t)\vert = \pi \sech(\pi t)$.
	\label{lem:change_of_variables}
\end{lemma}

\begin{proof}
	It is easier to consider $\Phi^{-1}(\varphi) = -i h(e^{i\varphi})$ and $\Psi^{-1}(\varphi) = -i\left( h(e^{i\varphi}) - 1 \right)$ (this already shows that $\Phi$ is a bijective mapping). Since $\left| e^{i\varphi}\right| = 1$ it is immediate by the representation (\ref{eq:rep_lft}) and $y < 0$ that $\Im \Phi(\varphi) = 0$. Furthermore, $\lim_{\varphi \rightarrow -\pi} \Phi(\varphi) = \infty$ and $\lim_{\varphi \rightarrow 0} \Phi(\varphi) = -\infty$. By (\ref{eq:h_diff}) $\Phi$ is clearly continuously differentiable. Using
	
	\begin{equation*}
		h^{-1}(it) = \frac{e^{-\pi t} - i}{e^{-\pi t} + i}
	\end{equation*}
	
	we get

	\begin{gather*}
		\left| D\Phi(t)\right| = \pi\left| \frac{e^{-\pi t}}{e^{\pi t} - i} - \frac{e^{-\pi t}}{e^{-\pi t} + i}\right| = \pi\left| \frac{2e^{-\pi t}}{e^{-2\pi t} + 1}\right| \pi \left| \frac{2}{e^{-\pi t} + e^{\pi t}}\right|= \pi \sech(\pi t)
	\end{gather*}
\end{proof}

\subsubsection{The Lemma}
Now we are able to proove the main result in prooving Stein's interpolation theorem.

\vspace{2mm}

\begin{mdframed}
	\begin{lemma}\emph{(Hadamard's three lines lemma, extension)}
		Let $F$ be a holomorphic function in the strip $S := \{z \in \mathbb{C}: 0 < \mathrm{Re}z < 1\}$ and continuous on $\overline{S}$, such that for some $A < \infty$ and $\tau \in [0,\pi[$ we have $\log \vert F(z)\vert \leqslant A e^{\tau \vert \Im z \vert}$ for every $z \in \overline{S}$. Then

			\begin{equation*}
				\vert F(z) \vert \leqslant \exp\left( \frac{\sin(\pi x)}{2} \int_{-\infty}^\infty \left[ \frac{\log \vert F(it + iy)\vert}{\cosh(\pi t) - \cos(\pi x)} + \frac{\log \vert F(1 + it + iy)\vert}{\cosh(\pi t) + \cos(\pi x)} \right] d\lambda(t)\right)
			\end{equation*}

			whenever $z := x + iy \in S$.
			\label{lem:EHTL}
	\end{lemma}
\end{mdframed}

\begin{proof}
	We will first proove the case \underline{$y = 0$.} Assume $F$ to be not identically zero (the case where $F$ is identically zero is trivial). Consider the function 

	on $D$.By composition, $F \circ h$ is holomorphic on $D$ and thus by \cite[336]{rudin:rc_analysis:1987} $\log\vert F \circ h \vert$ is subharmonic on $D$. It is easy to verify, that 

\begin{equation}
	h^{-1}(z) = \frac{e^{\pi i z} - i}{e^{\pi i z} + i}
	\label{eq:h_inv}
\end{equation}
 on the unit strip $S$.\\

Fix some $0 < R < 1$. Then $\log\left| F \circ h \right|$ is continuous for $\left| z \right| = R$ and subharmonic in by \cite[336]{rudin:rc_analysis:1987}. Define
	 
	\begin{gather*}
		\begin{aligned}
			H(re^{i\theta}):= \begin{cases}
				\log \vert F(h(Re^{i\theta}))\vert & r = R,\\
				\displaystyle \frac{1}{2\pi} \int_{-\pi}^\pi \log\vert F(h(Re^{it}))\vert \frac{R^2 - r^2}{R^2 - 2Rr\cos(\theta - t) + r^2} d\lambda(t) & 0 \leqslant r < R
		\end{cases}
		\end{aligned}
	\end{gather*}

Then $H$ is continuous for $\vert z \vert \leqslant R$ and harmonic for $\vert z \vert < R$ (see \cite[234--235]{rudin:rc_analysis:1987}). Since $\log\vert F(h(Re^{i\theta}))\vert = H(Re^{i\theta})$, by \cite[336]{rudin:rc_analysis:1987} we have

\begin{equation*}
	\log\vert F(h(re^{i\theta})) \leqslant \frac{1}{2\pi} \int_{-\pi}^\pi \log\vert F(h(Re^{it}))\vert \frac{R^2 - r^2}{R^2 - 2Rr\cos(\theta - t) + r^2} d\lambda(t) 
\end{equation*}

for $0 \leqslant r < R$. Now fix $r < R$, $-\pi < \theta \leqslant \pi$ and let $R := 1 - \frac{1}{n}$ for $n \in \mathbb{N}$ such that $r < R$ holds. Thus

\begin{equation}
	\log\vert F(h(re^{i\theta})) \leqslant \frac{1}{2\pi} \int_{-\pi}^\pi f_n(t) d\lambda(t)
\end{equation}

Consider $e^{i\theta}$ where $\Arg e^{i\theta} \neq 0,\pi$, we have $\Im \psi(e^{i\theta}) = 0$ and hence $\psi(e^{i\theta}) \in \mathbb{R}$. But then either $\Re h(e^{i\theta}) = 0$, $\psi(e^{i\theta}) > 0$ or $\Re h(e^{i\theta}) = 1$, $\psi(e^{i\theta}) < 0$. Hence the growth property of the hypothesis implies

\begin{gather*}
		\log \vert F(h(e^{i\theta})) \vert \leqslant Ae^{\tau\vert \Im h(e^{i\theta})\vert}
		= Ae^{\tau/\pi\vert \log\vert (1 + e^{i\theta})(1 - e^{i\theta})^{-1}\vert\vert}
		= A \left\vert \frac{1 + e^{i\theta}}{1 - e^{i\theta}} \right\vert^{\tau/\pi}
\end{gather*}

\item Fix some $re^{i\theta}$, $r < R$ and stipulate $x := h(re^{i\theta})$. Then we obtain \footnote{
		Recall, that for $z \in \mathbb{C}$ the trigonometric functions are defined by $\sin(z) := \frac{e^{iz} - e^{-iz}}{2i}$ and $\cos(z) := \frac{e^{iz} + e^{-iz}}{2}$. Hence the identities $e^{iz} = \cos(z) + i\sin(z)$ and $\cos^2(z) + \sin^2(z) = 1$ holds for any $z \in \mathbb{C}$ (see \cite[42--44]{ahlfors:complex_analysis:1979}).	
	}

	\begin{gather}
		\begin{aligned}
			re^{i\theta} &= h^{-1}(x) = \frac{e^{\pi i x}- i}{e^{\pi i x} + i} = \frac{\cos(\pi x) + i\sin(\pi x) - i}{\cos(\pi x) + i\sin(\pi x) + i}\\
			&= \frac{\cos(\pi x) + i\sin(\pi x) - i}{\cos(\pi x) + i\sin(\pi x) + i}\frac{\cos(\pi x) - i\sin(\pi x) - i}{\cos(\pi x) - i\sin(\pi x) - i} = -i \frac{\cos(\pi x)}{1 + \sin(\pi x)}\\
			&= \left( \frac{\cos(\pi x)}{1 + \sin(\pi x)} \right)e^{-i\pi/2}
		\end{aligned}
		\label{eq:radius_angle}
	\end{gather}

	by

		\begin{multline*}
			\left(\cos(\pi x) + i\sin(\pi x) - i\right)\left(\cos(\pi x) - i\sin(\pi x) - i\right)\\ = \cos^2(\pi x) - i\sin(\pi x)\cos(\pi x)	
			 -i\cos(\pi x) + i\sin(\pi x)\cos(\pi x)\\
			 + \sin^2(\pi x) + \sin(\pi x) - i \cos(\pi x) - \sin(\pi x) - 1 = -2i \cos(\pi x)  
		\end{multline*}

	and
	
		\begin{multline*}
			\left( \cos(\pi x) + i\sin(\pi x) + i \right)\left( \cos(\pi x) - i\sin(\pi x) - i \right)\\ = \cos^2(\pi x) - i\sin(\pi x)\cos(\pi x) - i\cos(\pi x) + i\sin(\pi x)\cos(\pi x)\\
		 	+ \sin^2(\pi x) + \sin(\pi x)+ i \cos(\pi x) + \sin(\pi x) + 1 = 2 + 2\sin(\pi x)
		\end{multline*}

	From equality (\ref{eq:radius_angle}) we deduce $r = \frac{\cos(\pi x)}{1 + \sin(\pi x)}$, $\theta = \frac{\pi}{2}$ if $0 < x \leqslant \frac{1}{2}$ and $r = -\frac{\cos(\pi x)}{1 + \sin(\pi x)}$, $\theta = \frac{\pi}{2}$ if $\frac{1}{2} \leqslant x < 1$. Let $0 < x \leqslant \frac{1}{2}$. Then we have

	\begin{multline*}
		\frac{1 - r^2}{1 - 2r\cos(\theta - \varphi) + r^2} \\= \frac{1 + 2\sin(\pi x) + \sin^2(\pi x) - \cos^2(\pi x)}{1 + 2\sin(\pi x) + \sin^2(\pi x) + 2\cos(\pi x)\sin(\varphi)(1 + \sin(\pi x)) + \cos^2(\pi x)}\\
			= \frac{\sin(\pi x) + \sin^2(\pi x)}{1 + \sin(\pi x) + \cos(\pi x)\sin(\varphi)(1 + \sin(\pi x))}
			= \frac{\sin(\pi x)}{1 + \cos(\pi x)\sin(\varphi)}
	\end{multline*}

	since $\cos\left( -\pi/2 - \varphi \right) = -\sin(\varphi)$. That the case $\frac{1}{2} \leqslant x < 1$ yields the same result is due to $\cos(\pi/2 - \varphi) = \sin(\varphi)$.\\
	Now we have to reformulate 

	\begin{equation}
		\frac{1}{2\pi} \int_{-\pi}^\pi \frac{\sin(\pi x)}{1 + \cos(\pi x)\sin(\varphi)} \log \vert F(h(e^{i\varphi}))\vert d\lambda(\varphi)
		\label{eq:int}
	\end{equation}


Let $\Phi$ and $\Psi$ be defined as in lemma (\ref{lem:change_of_variables}). We have 
				
\begin{gather*}
	\begin{aligned}
		e^{i\Phi(t)} &= h^{-1}(it) = \frac{e^{-\pi t} - i}{e^{-\pi t} + i} \frac{e^{-\pi t} - i}{e^{-\pi t} - i} = \frac{e^{-2\pi t} - 2ie^{-\pi t} - 1}{e^{-2\pi t} + 1} = \frac{e^{-2\pi t} - 1}{e^{-2\pi t} + 1} - \frac{2ie^{-\pi t}}{e^{-2\pi t} + 1}\\ 
		&= \frac{e^{-2\pi t} - 1}{e^{-2\pi t} + 1} - \frac{2i}{e^{-\pi t} + e^{\pi t}} = \frac{1 - e^{2\pi t}}{1 + e^{2\pi t}} - \frac{2i}{e^{-\pi t} + e^{\pi t}} = -\tanh (\pi t) - i \sech(\pi t)
	\end{aligned}
\end{gather*}
	
and thus

\begin{gather*}
	\begin{aligned}
		\sin\left( \Phi(t) \right)\cosh(\pi t) &= \sin\left( -i\log\left( -\tanh (\pi t) - i \sech(\pi t) \right)\right) \cosh(\pi t)\\
		&= \frac{1}{2i} \left[ -\tanh(\pi t) - i\sech(\pi t) + \frac{1}{\tanh(\pi t) + i\sech(\pi t) }\right]\cosh(\pi t)\\
		&= \frac{1}{2i} \left[ \frac{\cosh(\pi t) - \tanh(\pi t) \sinh(\pi t) - 2i\tanh(\pi t) + \sech(\pi t)}{\tanh(\pi t) + i\sech(\pi t)}\right]\\
		&= \frac{1}{2i} \left[ \frac{\cosh^2(\pi t) - \sinh^2(\pi t) - 2i \sinh(\pi t) + 1}{\sinh(\pi t) + i}\right]\\
		&= \frac{1 - i\sinh(\pi t)}{i\sinh(\pi t) - 1}\\
		&= -1
	\end{aligned}
\end{gather*}

Therefore the transformation formula yields

\begin{multline}
	\frac{1}{2\pi} \int_{-\pi}^0 \frac{\sin(\pi x)}{1 + \cos(\pi x)\sin(\varphi)} \log \vert F(h(e^{i\varphi}))\vert d\lambda(\varphi)\\ = \frac{1}{2}\int_{-\infty}^\infty\frac{\sin(\pi x)}{\cosh(\pi t) - \cos(\pi x)} \log\vert F(it) \vert d\lambda(t)
\end{multline}
				
and in a similar manner
		
\begin{multline}
	\frac{1}{2\pi} \int_0^\pi \frac{\sin(\pi x)}{1 + \cos(\pi x)\sin(\varphi)} \log \vert F(h(e^{i\varphi}))\vert d\lambda(\varphi)\\ = \frac{1}{2}\int_{-\infty}^\infty\frac{\sin(\pi x)}{\cosh(\pi t) + \cos(\pi x)} \log\vert F(1 + it) \vert d\lambda(t)
\end{multline}

holds since

\begin{gather*}
	\begin{aligned}
		\sin\left(\Psi(t)\right)\cosh(\pi t) &= \sin\left( -i\log\left( -\tanh (\pi t) + i \sech(\pi t) \right)\right) \cosh(\pi t)\\
		&= \frac{1}{2i} \left[ -\tanh(\pi t) + i\sech(\pi t) - \frac{1}{-\tanh(\pi t) + i\sech(\pi t) }\right]\cosh(\pi t)\\
		&= \frac{1}{2i} \left[ \frac{-\cosh(\pi t) + \tanh(\pi t) \sinh(\pi t) - 2i\tanh(\pi t) - \sech(\pi t)}{-\tanh(\pi t) + i\sech(\pi t)}\right]\\
		&= \frac{1}{2i} \left[ \frac{- \cosh^2(\pi t) + \sinh^2(\pi t) - 2i \sinh(\pi t) - 1}{i - \sinh(\pi t)}\right]\\
		&= \frac{1 + i\sinh(\pi t)}{1 + i\sinh(\pi t)}\\
		&= 1
	\end{aligned}
\end{gather*}

Thus the case $y = 0$ is prooven.\\
The case \underline{$y \neq 0$.} follows easily from the previous one. Fix $y \neq 0$ and define $G(z) := F(z + iy)$ for $z \in \overline{S}$. Then $G$ is a holomorphic function in $S$ and continuous on $\overline{S}$ as a composition of continuous and holomorphic functions. Moreover, the hypothesis on $F$ yields

		\begin{equation}
			\log \vert G(z) \vert = \log \vert F(z + iy) \vert \leqslant Ae^{\tau \vert \Im z + y\vert} \leqslant Ae^{\tau \vert \Im z \vert}e^{\tau \vert y \vert}
		\end{equation}

		for all $z \in \overline{S}$. The previous case yields for $G$ with $A$ replaced by $Ae^{\tau\vert y \vert}$

		\begin{equation}
			\vert G(x) \vert \leqslant \exp\left( \frac{\sin(\pi x)}{2} \int_{-\infty}^\infty \left[ \frac{\log \vert G(it)\vert}{\cosh(\pi t) - \cos(\pi x)} + \frac{\log \vert G(1 + it)\vert}{\cosh(\pi t) + \cos(\pi x)} \right] d\lambda(t)\right)
		\end{equation}

		Now, observing $G(x) = F(x + iy)$, $G(it) = F(it + iy)$ and $G(1 + it) = F(1 + it + iy)$ yields the desired result.
\end{proof}

\subsection{Stein's Theorem on Interpolation of Analytic Families of Operators}
Because of the complex nature of the proof of the Riesz-Thorin Interpolation Theorem (\ref{thm:Riesz_Thorin}), Elias M. Stein realized quickly, that the restritction to consider only one linear operator $T$ could easily be omited and instead, an analytic family of operators $T_z$ depending on some complex parameter $z$ could be considered.

\vspace{2mm}

\begin{mdframed}
	\begin{definition}\emph{(Analytic family, admissible growth)}
		Let $(X,\mu)$ be a measure space, $(Y,\nu)$ be a $\sigma$-finite measure spaces and $\left( T_z \right)_{z \in \overline{S}}$, where $T_z$ is defined on the space of all finitely simple functions on $X$ and taking values in the space of all measurable functions on $Y$ such that

		\begin{equation}
			\int_Y \vert T_z(\chi_A)\chi_B \vert d\nu
		\end{equation}

		whenever $\mu(A),\nu(B) < \infty$. The family $\left( T_z \right)_{z \in \overline{S}}$ is said to be \emph{analytic} if for all $f$, $g$ finitely simple we have that

		\begin{equation}
			z \mapsto \int_Y T_z(f)gd\nu
		\end{equation}

		is analytic on $S$ and continuous on $\overline{S}$. Further, an analytic family $\left( T_z \right)_{z \in \overline{S}}$ is called of \emph{admissible growth}, if there is a constant $\tau \in [0,\pi[$, such that for all finitely simple functions $f$, $g$ a constant $C(f,g)$ exists with

			\begin{equation}
				\log\left\vert \int_Y T_z(f) g d\nu\right\vert \leqslant C(f,g)e^{\tau\vert \mathrm{Im}z\vert}
			\end{equation}

			for all $z \in \overline{S}$.
	\end{definition}
\end{mdframed}

\vspace{2mm}

Now we are able to write down the theorem.

\vspace{2mm}

\begin{mdframed}
	\begin{theorem}\emph{(Stein's Theorem on Interpolation of Analytic Families of Operators)}
		Let $\left( T_z \right)_{z \in \overline{S}}$ be an analytic family of admissible growth, $1 \leqslant p_0,p_1,q_0,q_1 \leqslant \infty$ and suppose that $M_0$, $M_1$ are positive functions on the real line such that for some $\tau \in [0,\pi)$

			\begin{equation}
				\sup\left\{e^{-\tau \vert y \vert} \log M_0(y) : y \in \mathbb{R}\right\} < \infty \qquad \sup\left\{e^{-\tau \vert y \vert} \log M_1(y) : y \in \mathbb{R}\right\} < \infty
			\end{equation}

			Fix $0 < \theta < 1$ and define

			\begin{equation}
				\frac{1}{p} := \frac{1 - \theta}{p_0} + \frac{\theta}{p_1} \qquad \frac{1}{q} := \frac{1 - \theta}{q_0} + \frac{\theta}{q_1}
			\end{equation}

			Further suppose that for all finitely simple functions $f$ on $X$ and $y \in \mathbb{R}$ we have

			\begin{equation}
				\|T_{iy}(y)\|_{L^{q_0}} \leqslant M_0(y)\|f\|_{L^{p_0}} \qquad \|T_{1 + iy}(y)\|_{L^{q_1}} \leqslant M_1(y)\|f\|_{L^{p_1}} 
			\end{equation}

			Then for all finitely simple functions $f$ on $X$ we have

			\begin{equation*}
				\|T_\theta(f)\|_{L^q} \leqslant M(\theta)\|f\|_{L^p}
			\end{equation*}

			where for $0 < x < 1$

			\begin{equation*}
				M(x) = \exp\left( \frac{\sin(\pi x)}{2} \int_{-\infty}^\infty \left[ \frac{\log M_0(t)}{\cosh(\pi t) - \cos(\pi x)} + \frac{\log M_1(t)}{\cosh(\pi t) + \cos(\pi x)}\right] d\lambda(t) \right)
			\end{equation*}
	\end{theorem}
\end{mdframed}

\begin{proof}
	Fix $0 < \theta < 1$ and finitely simple functions $f$, $g$ on $X$, $Y$ respectively with $\|f\|_{L^p} = \|g\|_{L^{q'}} = 1$. Define $f_z$, $g_z$ as in (\ref{def:fzgz}) and for $z \in \overline{S}$

	\begin{equation}
		F(z) := \int_Y T_z(f_z)g_z d\nu	
	\end{equation}

	Observe, that $\left| a^{P(z)}_j\right| \leqslant a_j^{p/p_0 + p/p_1}$ and $\left| b^{Q(z)}_k\right| \leqslant b_k^{q'/q'_0 + q'/q'_1}$ for $z \in \overline{S}$. Hence

	\begin{gather*}
		\begin{aligned}
			\log \left| F(z) \right| &= \log \left| \sum_{j = 1}^n\sum_{k = 1}^m a^{P(z)}_j b_j^{Q(z)} e^{i\alpha_j} e^{i\beta_k} \int_YT_z(\chi_{X_j})(y)\chi_{Y_k}(y)d\nu(y)\right|\\
			&\leqslant  \log \left( \sum_{j = 1}^n\sum_{k = 1}^m a_j^{p/p_0 + p/p_1} b_k^{q'/q'_0 + q'/q'_1} \left|\int_{Y_k} T_z(\chi_{X_j}) d\nu\right|\right)\\
		\end{aligned}
	\end{gather*}

	The same calculations as in the proof of the Riesz-Thorin interpolation theorem (\ref{thm:Riesz_Thorin}) yields for $y \in \mathbb{R}$

	\begin{equation*}
		\left\| f_{iy}\right\|_{L^{p_0}} = \left\| f\right\|^{p/p_0}_{L^p} = 1 = \left\| g\right\|_{L^{q'}}^{q'/q'_0} = \left\| g_{iy}\right\|_{L^{q_0'}}
	\end{equation*}

	and

	\begin{equation*}
		\left\| f_{1 + iy}\right\|_{L^{p_1}} = \left\| f\right\|^{p/p_1}_{L^p} = 1 = \left\| g\right\|_{L^{q'}}^{q'/q'_1} = \left\| g_{1 + iy}\right\|_{L^{q_1'}}
	\end{equation*}

	Further

	\begin{equation*}
		\left| F(iy)\right| \leqslant \left\| T_{iy}(f_{iy})\right\|_{L^{q_0}} \left\| g_{iy}\right\|_{L^{q'_0}} \leqslant M_0(y) \left\|f_{iy}\right\|_{L^{p_0}}\left\| g_{iy}\right\|_{L^{q_0'}} = M_0(y)
	\end{equation*}

	and

	\begin{equation*}
		\left| F(1 + iy)\right| \leqslant \left\| T_{1 + iy}(f_{1 + iy})\right\|_{L^{q_1}} \left\| g_{1 + iy}\right\|_{L^{q'_1}} \leqslant M_1(y) \left\|f_{1 + iy}\right\|_{L^{p_1}}\left\| g_{1 + iy}\right\|_{L^{q_1'}} = M_1(y)
	\end{equation*}

	by H\"older's inequality and the hypotheses on the analytic family $(T_z)_{z \in \overline{S}}$. Therefore the extension of Hadamard's three lines lemma (\ref{lem:EHTL}) yields

	\begin{equation*}
		\left| F(x) \right| \leqslant \exp\left( \frac{\sin(\pi x)}{2} \int_{-\infty}^\infty \left[ \frac{\log M_0(t)}{\cosh(\pi t) - \cos(\pi x)} + \frac{\log M_1(t)}{\cosh(\pi t) + \cos(\pi x)}\right] d\lambda(t) \right) = M(x)
	\end{equation*}

	for every $0 < x < 1$. Furthermore observe that

	\begin{equation*}
		F(\theta) = \int_Y T_\theta(f)gd\nu
	\end{equation*}

	and thus by \cite[189]{folland:real_analysis:1999} ($\Sigma$ denotes the set of all finitely simple functions on the $\sigma$-finite space $Y$)

	\begin{gather*}
		\begin{aligned}
			M_q\left(T_\theta(f)\right) &= \sup\left\{ \left| \int_Y T_\theta(f) g\right| : g \in \Sigma, \left\|g \right\|_{L^{q'}}\right\}\\
			&= \sup\left\{ \left| F(\theta)\right| : g \in \Sigma, \left\| g \right\|_{L^{q'}}\right\}\\
			&\leqslant M(\theta)
		\end{aligned}
	\end{gather*}

	Since $M(\theta)$ is an absolutely convergent integral for any $0 < \theta < 1$, $M_q\left(T_\theta(f)\right) < \infty$ and thus $M_q\left(T_\theta(f)\right) = \left\| T_\theta(f)\right\|_{L^q}$. The general statement follows by replacing $f$ with $f/\left\| f\right\|_{L^p}$ when $\left\|f \right\|_{L^p} \neq 0$. The theorem is trivially true when $\left\| f\right\|_{L^p} = 0$.
\end{proof}

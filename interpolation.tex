%%%%%%%%%%%%%%%%%%%%%%%%%%%%%%%%%%%%%%%%%%%%%%%%%%%%%%%%%%%%%%%%%%%%%%%%%%
%Author:																 %
%-------																 %
%Yannis Baehni at University of Zurich									 %
%baehni.yannis@uzh.ch													 %
%																		 %
%Version log:															 %
%------------															 %
%06/02/16 . Basic structure												 %
%04/08/16 . Layout changes including section, contents, abstract.		 %
%%%%%%%%%%%%%%%%%%%%%%%%%%%%%%%%%%%%%%%%%%%%%%%%%%%%%%%%%%%%%%%%%%%%%%%%%%

%Page Setup
\documentclass[
	11pt, 
	oneside, 
	a4paper,
	reqno,
	final
]{amsart}

\usepackage[
	left = 3cm, 
	right = 3cm, 
	top = 3cm, 
	bottom = 3cm
]{geometry}

%Headers and footers
\usepackage{fancyhdr}
	\pagestyle{fancy}
	%Clear fields
	\fancyhf{}
	%Header right
	\fancyhead[R]{
		\footnotesize
		Yannis B\"{a}hni\\
		\href{mailto:yannis.baehni@uzh.ch}{yannis.baehni@uzh.ch}
	}
	%Header left
	\fancyhead[L]{
		\footnotesize
		MAT694 Seminar: Introduction to Harmonic Analysis\\
		HS16
	}
	%Page numbering in footer
	\fancyfoot[C]{\thepage}
	%Separation line header and footer
	\renewcommand{\headrulewidth}{0.4pt}
	%\renewcommand{\footrulewidth}{0.4pt}
	
	\setlength{\headheight}{19pt} 

%Title
\usepackage[foot]{amsaddr}
\usepackage{xspace}
\makeatletter
\def\@textbottom{\vskip \z@ \@plus 1pt}
\let\@texttop\relax
\usepackage{etoolbox}
\patchcmd{\abstract}{\scshape\abstractname}{\textbf{\abstractname}}{}{}

%Switching commands for different section formats
%Mainsectionsytle
\newcommand{\mainsectionstyle}{%
  \renewcommand{\@secnumfont}{\bfseries}
  \renewcommand\section{\@startsection{section}{2}%
    \z@{.5\linespacing\@plus.7\linespacing}{-.5em}%
    {\normalfont\bfseries}}%
}
\newcommand{\originalsectionstyle}{%
\def\@secnumfont{\bfseries}%\mdseries
\def\section{\@startsection{section}{1}%
  \z@{.7\linespacing\@plus\linespacing}{.5\linespacing}%
  {\normalfont\bfseries\centering}}
}
%Formatting title of TOC
\renewcommand{\contentsnamefont}{\bfseries}
%Table of Contents
\setcounter{tocdepth}{3}

% Add bold to \section titles in ToC and remove . after numbers
\renewcommand{\tocsection}[3]{%
  \indentlabel{\@ifnotempty{#2}{\bfseries\ignorespaces#1 #2\quad}}\bfseries#3}
% Remove . after numbers in \subsection
\renewcommand{\tocsubsection}[3]{%
  \indentlabel{\@ifnotempty{#2}{\ignorespaces#1 #2\quad}}#3}
%\let\tocsubsubsection\tocsubsection% Update for \subsubsection
%...

\newcommand\@dotsep{4.5}
\def\@tocline#1#2#3#4#5#6#7{\relax
  \ifnum #1>\c@tocdepth % then omit
  \else
    \par \addpenalty\@secpenalty\addvspace{#2}%
    \begingroup \hyphenpenalty\@M
    \@ifempty{#4}{%
      \@tempdima\csname r@tocindent\number#1\endcsname\relax
    }{%
      \@tempdima#4\relax
    }%
    \parindent\z@ \leftskip#3\relax \advance\leftskip\@tempdima\relax
    \rightskip\@pnumwidth plus1em \parfillskip-\@pnumwidth
    #5\leavevmode\hskip-\@tempdima{#6}\nobreak
    \leaders\hbox{$\m@th\mkern \@dotsep mu\hbox{.}\mkern \@dotsep mu$}\hfill
    \nobreak
    \hbox to\@pnumwidth{\@tocpagenum{\ifnum#1=1\bfseries\fi#7}}\par% <-- \bfseries for \section page
    \nobreak
    \endgroup
  \fi}
\AtBeginDocument{%
\expandafter\renewcommand\csname r@tocindent0\endcsname{0pt}
}
\def\l@subsection{\@tocline{2}{0pt}{2.5pc}{5pc}{}}
\makeatother

\advance\footskip0.4cm
\textheight=54pc    %a4paper
\textheight=50.5pc %letterpaper
\advance\textheight-0.4cm
\calclayout

%Font settings
%\usepackage{anyfontsize}

%Footnote settings
\usepackage{footmisc}
%	\renewcommand*{\thefootnote}{\fnsymbol{footnote}}

%Further math environments
%Further math fonts (loads amsfonts implicitely)
\usepackage{amssymb}
%Redefinition of \text
%\usepackage{amstext}
\usepackage{upref}
%Graphics
%\usepackage{graphicx}
%\usepackage{caption}
%\usepackage{subcaption}
%Frames
\usepackage{mdframed}

\renewcommand{\Re}{\operatorname{Re}}
\renewcommand{\Im}{\operatorname{Im}}
\DeclareMathOperator\Log{Log}
\DeclareMathOperator\Arg{Arg}
\DeclareMathOperator\sech{sech}
%\usepackage{hhline}
%\usepackage{booktabs} 
%\usepackage{array}
%\usepackage{xfrac} 
%\everymath{\displaystyle}
%Enumerate
\usepackage{enumitem} 
%\renewcommand{\labelitemi}{$\bullet$}
%\renewcommand{\labelitemii}{$\ast$}
%\renewcommand{\labelitemiii}{$\cdot$}
%\renewcommand{\labelitemiv}{$\circ$}
%Colors
%\usepackage{color}
%\usepackage[cmtip, all]{xy}
%Theorems
\newtheoremstyle{bold}              	 %Name
  {}                                     %Space above
  {}                                     %Space below
  {\itshape}		                     %Body font
  {}                                     %Indent amount
  {\scshape}                             %Theorem head font
  {.}                                    %Punctuation after theorem head
  { }                                    %Space after theorem head, ' ', 
  										 %	or \newline
  {} 
\theoremstyle{bold}
\newtheorem{definition}{Definition}[section]
\newtheorem{lemma}{Lemma}[section]
\newtheorem{Proof}{Proof}[section]
\newtheorem{proposition}{Proposition}[section]
\newtheorem{properties}{Properties}[section]
\newtheorem{corollary}{Corollary}[section]
\newtheorem{theorem}{Theorem}[section]
\newtheorem{example}{Example}[section]
\newtheorem{remark}{Remark}[section]
%German non-ASCII-Characters
%Graphics-Tool
%\usepackage{tikz}
%\usepackage{tikzscale}
%\usepackage{bbm}
%\usepackage{bera}
%Listing-Setup
%Bibliographie
\usepackage[backend=bibtex, style=alphabetic]{biblatex}
%\usepackage[babel, german = swiss]{csquotes}
\bibliography{Bibliography}
%PDF-Linking
%\usepackage[hyphens]{url}
\usepackage[bookmarksopen=true,bookmarksnumbered=true]{hyperref}
%\PassOptionsToPackage{hyphens}{url}\usepackage{hyperref}
\hypersetup{
  colorlinks   = true, %Colours links instead of ugly boxes
  urlcolor     = blue, %Colour for external hyperlinks
  linkcolor    = blue, %Colour of internal links
  citecolor    = blue %Colour of citations
}
%Weierstrass-P symbol for power set
\newcommand{\powerset}{\raisebox{.15\baselineskip}{\Large\ensuremath{\wp}}}


\begin{document}
\begin{abstract}
	In this written seminar work I will basically follow \cite[33--48]{grafakos:fourier:2014}. I will review three basic but important theorems on interpolation of operators on $L^p$ spaces, namely the \emph{Marcinkiewicz Interpolation Theorem}, the \emph{Riesz-Thorin Interpolation Theorem} and finally an extension of the Riesz-Thorin Interpolation Theorem to analytic families of operators (the so-called \emph{Stein's theorem on interpolation of analytic families of operators}). We are mainly concerned with the notion of linear operators as well as slight generalizations of them. 
\end{abstract}

\maketitle

\tableofcontents

\mainsectionstyle

\section{Introduction and Basic Definitions}
\subsection{Linear Operators}
First we need to have a precise and suitable idea of \emph{linear operators} in the generalized setting of measure spaces.
\vspace{2mm}
\begin{mdframed}
	\begin{definition}
		Let $(X,\mu)$ and $(Y,\nu)$ be measure spaces. Further let $T$ be an operator defined on a linear space of complex-valued measurable functions on $X$ and taking values in the set of all complex-valued, finite almost everywhere, measurable functions on $Y$. Then $T$ is called \emph{linear} if for all functions $f$ and $g$ in the domain of $T$ and all $z \in \mathbb{C}$ holds

		\begin{equation}
			T\left( f + g \right) = T(f) + T(g) \qquad T\left( zf \right) = zT(f)
			\label{eq:linear}
		\end{equation}

		and \emph{quasi-linear} if

		\begin{equation}
			\vert T\left( f + g \right) \vert \leqslant K \left( \vert T(f)\vert + \vert T(g)\vert \right) \qquad \vert T(zf) \vert = \vert z\vert \vert T(f)\vert
			\label{eq:quasilinear}
		\end{equation}

		holds for some real constant $K > 0$. If $K = 1$, $T$ is called \emph{sublinear}.
	\end{definition}
\end{mdframed}

%Real method
\section{The Real Method}
\subsection{The Marcinkiewicz Interpolation Theorem}
The name originates from the real variables technique used for prooving the theorem.
\vspace{2mm}
\begin{mdframed}
	\begin{theorem}\emph{(The Marcinkiewicz Interpolation Theorem)}
		Let $(X,\mu)$ be a $\sigma$-finite measure space, $(Y,\nu)$ another measure space and $0 < p_0 < p_1 \leqslant \infty$. Further let $T$ be a sublinear operator defined on
		
		\begin{equation*}
			L^{p_0} + L^{p_1} := \left\{ f_0 + f_1 : f_0 \in L^{p_0}(X,\mu), f_1 \in L^{p_1}(X,\mu) \right\}
		\end{equation*}
		
		and taking values in the space of measurable functions on $Y$. Assume that there exist $A_0,A_1 < \infty$ such that

		\begin{align}
			&\forall f \in L^{p_0}(X,\mu)~\|T(f)\|_{L^{p_0,\infty}} \leqslant A_0 \|f\|_{L^{p_0}}\label{hyp:fp_0}\\
			&\forall f \in L^{p_1}(X,\mu)~\|T(f)\|_{L^{p_1,\infty}} \leqslant A_1 \|f\|_{L^{p_1}}\label{hyp:fp_1}
		\end{align}

		Then for all $p_0 < p < p_1$ and for all $f \in L^p(X,\mu)$ we have the estimate

		\begin{equation}
			\|T(f)\|_{L^p} \leqslant A \|f\|_{L^p}
		\end{equation}

		where

		\begin{equation}
			A := 2\left( \frac{p}{p - p_0} + \frac{p}{p_1 - p} \right)^{1/p}A_0^{\frac{\frac{1}{p} - \frac{1}{p_1}}{\frac{1}{p_0}-\frac{1}{p_1}}}A_1^{\frac{\frac{1}{p_0}-\frac{1}{p}}{\frac{1}{p_0}-\frac{1}{p_1}}}
			\label{eq:constant}
		\end{equation}
	\end{theorem}
\end{mdframed}

\begin{proof}
	The proof is subdivided into two main parts, which are further subdivided. In detail, we have the following partitioning:

	\begin{enumerate}[label = \textbf{(\roman*.)}]
		\item \underline{$p_1 < \infty$.}
			\begin{enumerate}[label = \textbf{\alph*.}]
				\item Split $f$ using cut-off functions.
				\item Estimate the distribution function $d_{T(f)}$.
				\item Estimate $\|T(f)\|_{L^p}^p$.
			\end{enumerate}
		\item \underline{$p_1 = \infty$.}
			\begin{enumerate}[label = \textbf{\alph*.}]
				\item Show that $\mu(\{\vert T(f_1)\vert > \alpha/2\}) = 0$.
				\item Estimate the distribution function $d_{T(f_0)}$.
				\item Estimate $\|T(f)\|_{L^p}^p$.
			\end{enumerate}
	\end{enumerate}

	\begin{enumerate}[label = \textbf{(\roman*.)}]
		\item 
			\begin{enumerate}[label = \textbf{\alph*.}]
				\item Let us first consider the case \underline{$p_1 < \infty$}. Fix $f \in L^p(X,\mu)$, $\alpha > 0$ and $\delta > 0$ ($\delta$ will be determined later). We split $f$ using so-called \emph{cut-off} functions, by stipulating $f \equiv f_0(\cdot;\alpha,\delta) + f_1(\cdot;\alpha,\delta)$, where $f_0(\cdot;\alpha,\delta)$ is the \emph{unbounded part of $f$} and $f_1(\cdot;\alpha,\delta)$ is the \emph{bounded part of $f$}, defined by

	\begin{gather}
		\begin{aligned}
			f_0(x;\alpha,\delta) &:= \begin{cases}
				f(x), & \vert f(x) \vert > \delta \alpha,\\
				0, & \vert f(x)\vert \leqslant \delta \alpha.
			\end{cases}\\
			f_1(x;\alpha,\delta) &:= \begin{cases}
				f(x), & \vert f(x) \vert \leqslant \delta \alpha,\\
				0, & \vert f(x)\vert > \delta \alpha.
			\end{cases}
		\end{aligned}
		\label{eq:cut_off}
	\end{gather}

	for $x \in X$. To facilitate reading I will omit the dependency of $f_0(\cdot;\alpha,\delta)$ and $f_1(\cdot;\alpha,\delta)$ upon the parameters $\alpha$ and $\delta$ in what follows and simply write $f_0$, $f_1$ respectively. Since $p_0 < p$ we have 

	\begin{gather}
		\begin{aligned}
			\|f_0\|^{p_0}_{L^{p_0}} &= \int_{X} \vert f_0\vert^{p_0} d\mu =\int_{X} \vert f \vert^{p_0} \cdot \chi_{\left\{\vert f\vert > \delta\alpha \right\}} d\mu \overset{(\dagger)}{=} \int_{\left\{\vert f \vert > \delta\alpha \right\}} \vert f \vert^{p_0}d\mu\\ 
			&= \int_{\left\{\vert f\vert > \delta\alpha \right\}} \vert f \vert^p \vert f \vert^{p_0 - p} d\mu = \int_{\left\{\vert f\vert > \delta\alpha \right\}} \frac{\vert f \vert^p}{\vert f \vert^{p - p_0}} d\mu\\
			&\leqslant \frac{1}{(\delta\alpha)^{p - p_0}} \int_{\left\{\vert f\vert > \delta\alpha \right\}} \vert f \vert^p d\mu = (\delta\alpha)^{p_0 - p} \int_{X} \vert f \vert^p \cdot \chi_{\left\{\vert f\vert > \delta\alpha \right\}} d\mu\\
			& \leqslant (\delta\alpha)^{p_0 - p} \int_{X} \vert f \vert^p d\mu = (\delta\alpha)^{p_0 - p} \|f\|^p_{L^p} < \infty
		\end{aligned}
		\label{est:f0}
	\end{gather}

	Thus $f_0 \in L^{p_0}(X,\mu)$. Analogously it can be checked, that $f_1 \in L^{p_1}(X,\mu)$ by the estimate $\|f_1\|^{p_1}_{L^{p_1}} \leqslant (\delta\alpha)^{p_1 - p}\|f\|_{L^p}^p$. Therefore $f \equiv f_0 + f_1 \in L^{p_0} + L^{p_1}$.\\
	
	\emph{Proof of the equality $(\dagger)$.} Assume $\mu$ is defined on the $\sigma$-algebra $\mathcal{A}$. We have to proove that $\{\vert f \vert > \delta\alpha\} \in \mathcal{A}$\footnote{
		For $Y \in \mathcal{A}$ the $\mu$-integral of $f: X \rightarrow \mathbb{C}$ over $Y$ is defined to be $\displaystyle \int_Y fd\mu := \int_X f \cdot \chi_Y d\mu$. For more details see \cite[135--136]{elstrodt:mass:2011}.}.
		Since $f$ is complex-valued, we may write $f \equiv \mathrm{Re} f + i\mathrm{Im}f$ and thus $\vert f\vert^2 \equiv \mathrm{Re}^2 f + \mathrm{Im}^2f$. Since $f$ is measurable by hypothesis this implies that $ \mathrm{Re} f$ and $\mathrm{Im}f$ are measurable\footnote{For a proof see \cite[106]{elstrodt:mass:2011}}. Further for measurable real-valued functions $f,g: (X,\mathcal{A}) \rightarrow (\overline{\mathbb{R}},\overline{\mathfrak{B}})$\footnote{$\overline{\mathfrak{B}} := \sigma(\overline{\mathbb{R}})$ and $\overline{\mathfrak{B}} = \{B \cup E : B \in \mathfrak{B}, E \subseteq \{\pm \infty\}\}$.} 
		the functions $f + g$ and $f \cdot g$ are measurable\footnote{For a proof see \cite[107]{elstrodt:mass:2011}.}
		and thus $\vert f \vert^2$ is measurable. Hence $\{ \mathrm{Re}^2 f + \mathrm{Im}^2f > \lambda\} \in \mathcal{A}$\footnote{For a proof see \cite[105--106]{elstrodt:mass:2011}} for any $\lambda \in \mathbb{R}$. So especially for $\lambda := (\delta\alpha)^2$ we have $\{\vert f \vert > \delta\alpha\} \in \mathcal{A}$\footnote{This follows from the fact that $x < y$ if and only if $x^n < y^n$ for $n \in \mathbb{N}_{>0}$ and some real numbers $x,y > 0$ (see \cite[119]{zorich:analysis_I:2004}).}.
	In a similar manner it can also be prooven that $\{\vert f\vert \leqslant \delta \alpha\} \in \mathcal{A}$. Let us next proove a useful lemma.

	\begin{lemma}
		Let $A \in \powerset(X)$ and $\chi_A: (X,\mathcal{A}) \rightarrow (\mathbb{C},\mathfrak{B}^2)$ be the \emph{characteristic function of the set $A$}. Then $\chi_A$ is measurable if and only if $A$ is measurable.
			\label{lem:charfun}
	\end{lemma}
		
	\begin{proof}
		Assume $\chi_A$ is measurable. Then $\mathrm{Re}\chi_A$ and $\mathrm{Im}\chi_A$ are measurable. Especially for $0 < \lambda < 1$ we have that $\{\mathrm{Re}\chi_A > \lambda\} = A \in \mathcal{A}$. Conversly, assume $A$ is measurable. For $\lambda < 0$ we have $\{\mathrm{Re}\chi_A > \lambda\} = X \in \mathcal{A}$, $\lambda \in [0,1[$, $\{\mathrm{Re}\chi_A > \lambda\} = A \in \mathcal{A}$ and $\{\mathrm{Re}\chi_A > \lambda\} = \emptyset \in \mathcal{A}$ for $\lambda \geqslant 1$. Since $\mathrm{Im}\chi_A \equiv 0$ we have $\{\mathrm{Im}\chi_A > \lambda \} = X \in \mathcal{A}$ if $\lambda < 0$ and $\{\mathrm{Im}\chi_A > \lambda\} = \emptyset \in \mathcal{A}$ if $\lambda \geqslant 0$.   
	\end{proof}

	By Lemma \ref{lem:charfun} and the fact that $f \cdot g$ is measurable for two measurable functions $f,g: (X,\mathcal{A}) \rightarrow (\mathbb{C},\mathfrak{B}^2)$\footcite[107]{elstrodt:mass:2011}, $f_0$ and $f_1$ are measurable since $f_0 \equiv f \cdot \chi_{\{\vert f \vert > \delta \alpha\}}$ and $f_1 \equiv f \cdot \chi_{\{\vert f \vert \leqslant \delta \alpha\}}$.\\

	One subtility is left to clear: the $\mu$-integrability of either $\vert f_1\vert^{p_0}$ or $\vert f_1 \vert^{p_1}$ requires that $\vert f_0 \vert^{p_0}$ and $\vert f_1 \vert^{p_1}$ are measurable functions. By the fact that any continuous map $g: (X,d_X) \rightarrow (Y,d_Y)$ between metric spaces is Borel-measurable (see \cite[86]{elstrodt:mass:2011}) and that the composition of measurable functions is again measurable (see \cite[87]{elstrodt:mass:2011}), the measurability of either $f_0$ or $f_1$ follows by $\vert f_0 \vert^{p_0} \equiv \cdot^{p_0} \circ \vert f \cdot \chi_{\{\vert f\vert > \delta\alpha\}}\vert$ and $\vert f_1 \vert^{p_1} \equiv \cdot^{p_1} \circ \vert f \cdot \chi_{\{\vert f \vert \leqslant \delta \alpha\}}\vert$ by stipulating $\cdot^{p}: (\mathbb{R}_{\geqslant 0},\vert \cdot \vert) \rightarrow (\mathbb{C},\vert \cdot \vert)$, $x^{p} := \exp(p \log(x))$ for $p > 0$ and $x \in \mathbb{R}_{> 0}$ and $x^p := 0$ if $x = 0$.

	\item Since $T$ is a sublinear operator we have $\vert T(f) \vert = \vert T(f_0 + f_1) \vert \leqslant \vert T(f_0) \vert + \vert T(f_1)\vert$. Thus for any $y \in Y$ with $\vert T(f)(y) \vert > \alpha$ we therefore have either $\vert T(f_0)(y) \vert > \alpha/2$ or $\vert T(f_1)(y) \vert > \alpha/2$ 
		\footnote{Without loss of generality assume $\vert T(f_0)(y) \vert \leqslant \vert T(f_1)(y) \vert $. Then we have $\alpha < \vert T(f)(y)\vert \leqslant \vert T(f_0)(y) \vert + \vert T(f_1)(y)\vert \leqslant 2\vert T(f_1)(y)\vert$ (this is possible since $\mathbb{R}$ is an ordered field).}
		. Hence

\begin{equation*}
	\{\vert T(f)\vert > \alpha \} \subseteq \{\vert T(f_0) \vert > \alpha/2 \} \cup \{\vert T(f_1) \vert > \alpha/2 \}
\end{equation*}

and so by the monotonicity and subadditivity property of the measure $\mu$ we have

\begin{gather}
	\begin{aligned}
	d_{T(f)}(\alpha) &= \mu(\{\vert T(f)\vert > \alpha\})\\
	&\leqslant \mu(\{\vert T(f_0)\vert > \alpha/2 \} \cup \{\vert T(f_1)\vert > \alpha/2 \})\\
	&\leqslant \mu(\{\vert T(f_0) \vert > \alpha/2 \}) + \mu(\{\vert T(f_1)\vert > \alpha/2 \})\\
	&= d_{T(f_0)}(\alpha/2) + d_{T(f_1)}(\alpha/2)
	\label{est:T}
	\end{aligned}
\end{gather}

Now by hypothesis (\ref{hyp:fp_0}) we can estimate $d_{T(f_0)}(\alpha/2)$ as follows

\begin{gather}
	\begin{aligned}
		d_{T(f_0)}(\alpha/2) &= \left(\frac{\alpha/2}{\alpha/2}\right)^{p_0} d_{T(f_0)}(\alpha/2)\\
		&\leqslant \left(\frac{1}{\alpha/2}\right)^{p_0} \left[\sup\left\{ \gamma d_{T(f_0)}(\gamma)^{1/p_0}: \gamma > 0\right\}\right]^{p_0}\\
	 & = \left(\frac{1}{\alpha/2}\right)^{p_0} \|T(f_0)\|^{p_0}_{L^{p_0,\infty}}\\
	 & \leqslant \left(\frac{A_0}{\alpha/2}\right)^{p_0} \|f_0\|^{p_0}_{L^{p_0}}
	\label{est:p_0}
	\end{aligned}
\end{gather}

Analogously, we get  $d_{T(f_1)}(\alpha/2) \leqslant \left(\frac{A_1}{\alpha/2}\right)^{p_1} \|f_1\|^{p_1}_{L^{p_1}}\label{est:p_1}$ by hypothesis (\ref{hyp:fp_1}).

	\item By

		\begin{gather}
			\begin{aligned}
				\int_0^{\frac{1}{\delta}\vert f\vert}\alpha^{p-p_0-1} d\lambda = 
				\begin{cases}
					\frac{1}{p-p_0}\frac{1}{\delta^{p-p_0}}\vert f \vert^{p - p_0}, & p \geqslant p_0 + 1\\
					\lim\limits_{\omega \rightarrow 0^+} \int_\omega^{\frac{1}{\delta}\vert f\vert}\alpha^{p-p_0-1} d\lambda\\
					= \lim\limits_{\omega \rightarrow 0^+}\left[\frac{1}{p-p_0}\alpha^{p - p_0}\big\vert_\omega^{\frac{1}{\delta}\vert f\vert}\right]\\
					= \frac{1}{p-p_0}\left[\frac{1}{\delta^{p-p_0}}\vert f \vert^{p - p_0} - \lim\limits_{\omega \rightarrow 0^+} \omega^{p-p_0}\right]\\
					= \frac{1}{p-p_0}\frac{1}{\delta^{p-p_0}} \vert f\vert^{p - p_0}, & p_0 < p < p_0 + 1
				\end{cases}
			\end{aligned}
		\end{gather}

		and

		\begin{gather}
			\begin{aligned}
				\int_{\frac{1}{\delta}\vert f\vert}^{\infty}\alpha^{p-p_1-1} d\lambda &= \lim_{\omega \rightarrow \infty} \left[ \frac{1}{p - p_1} \alpha^{p - p_1}\right]^\omega_{\frac{1}{\delta}\vert f\vert}\\
				&= \frac{1}{p - p_1} \left[  \lim_{\omega \rightarrow \infty} \omega^{p - p_1} - \frac{1}{\delta^{p - p_1}} \vert f\vert^{p - p_1}\right]\\
				&= \frac{1}{p_1 - p}\frac{1}{\delta^{p-p_1}} \vert f \vert^{p - p_1}
			\end{aligned}
		\end{gather}

		and the representation $\displaystyle \|f\|^p_{L^p} = p \int_0^{\infty} \alpha^{p-1}d_f(\alpha) d\lambda$ for $ 0 < p < \infty$ we get

		\begin{gather}
			\begin{aligned}
				\|T(f)\|^p_{L^p} = & p\int_0^{\infty}\alpha^{p-1}d_{T(f)} d\lambda\\
				\leqslant & p(2A_0)^{p_0}\int_0^{\infty}\alpha^{p-p_0-1} \int_{\{\vert f \vert > \delta \alpha\}} \vert f\vert^{p_0}d\mu d\lambda\\
				& + p(2A_1)^{p_1}\int_0^{\infty}\alpha^{p-p_1-1} \int_{\{\vert f \vert \leqslant \delta \alpha\}} \vert f \vert^{p_1}d\mu d\lambda\\
				= & p(2A_0)^{p_0}\int_{\{\vert f \vert > 0\}} \vert f \vert^{p_0} \int_0^{\frac{1}{\delta}\vert f\vert}\alpha^{p-p_0-1} d\lambda d\mu\\
				& + p(2A_0)^{p_0}\int_{\{\vert f \vert = 0\}} \vert f \vert^{p_0} \int_0^{\frac{1}{\delta}\vert f\vert}\alpha^{p-p_0-1} d\lambda d\mu\\
				& + p(2A_1)^{p_1}\int_X \vert f\vert^{p_1} \int_{\frac{1}{\delta}\vert f\vert}^{\infty}\alpha^{p - p_1 - 1} d\lambda d\mu\\
				= & p(2A_0)^{p_0}\int_X \vert f \vert^{p_0} \int_0^{\frac{1}{\delta}\vert f\vert}\alpha^{p-p_0-1} d\lambda d\mu\\
				& + p(2A_1)^{p_1}\int_X \vert f\vert^{p_1} \int_{\frac{1}{\delta}\vert f\vert}^{\infty}\alpha^{p - p_1 - 1} d\lambda d\mu\\
				= & \frac{p(2A_0)^{p_0}}{p-p_0}\frac{1}{\delta^{p-p_0}}\int_X \vert f \vert^{p_0}\vert f \vert^{p-p_0} d\mu\\
				& + \frac{p(2A_1)^{p_1}}{p_1-p}\frac{1}{\delta^{p-p_1}}\int_X \vert f \vert^{p_1} \vert f\vert^{p-p_1}d\mu\\
				= & p\left( \frac{(2A_0)^{p_0}}{p - p_0}\frac{1}{\delta^{p - p_0}} + \frac{(2A_1)^{p_1}}{p_1 - p}\delta^{p_1 - p} \right)\|f\|_{L^p}^p
			\end{aligned}
			\label{est:Tfp}
		\end{gather}

		We pick $\delta > 0$ such that $(2A_0)^{p_0}\delta^{p_0 - p} = (2A_1)^{p_1}\delta^{p_1 - p}$. Solving for $\delta$ yields 

		\begin{equation}
			\delta = \frac{1}{2} \left( \frac{A_0}{A_1}\right)^{p_1/(p_1 - p_0)}
		\end{equation}

		Substituting this in estimate (\ref{est:Tfp}) leads to

		\begin{gather}
			\begin{aligned}
				\|T(f)\|_{L^p}^p &\leqslant p\left( \frac{(2A_0)^{p_0}}{p - p_0}\frac{2^{p - p_0}A_1^\frac{p_1(p-p_0)}{p_1-p_0}}{A_0^\frac{p_0(p-p_0)}{p_1 - p_0}} + \frac{(2A_1)^{p_1}}{p_1 - p} \frac{A_0^\frac{p_0(p_1 - p)}{p_1 - p_0}}{2^{p_1 - p}A_1^\frac{p_1(p_1 - p)}{p_1 - p_0}} \right)\|f\|_{L^p}^p\\
				&=  2^pp\left( \frac{A_0^\frac{p_0(p_1 - p)}{p_1 - p_0}A_1^\frac{p_1(p-p_0)}{p_1-p_0}}{p - p_0} + \frac{A_0^\frac{p_0(p_1 - p)}{p_1 - p_0}A_1^\frac{p_1(p - p_0)}{p_1 - p_0}}{p_1- p} \right)\|f\|_{L^p}^p\\
			\end{aligned}
		\end{gather}

		And taking the $p$-th power further

		\begin{gather}
			\begin{aligned}
				\|T(f)\|_{L^p} &\leqslant 2\left( \frac{p}{p - p_0} + \frac{p}{p_1- p} \right)^{1/p} A_0^\frac{p_0(p_1 - p)}{p(p_1 - p_0)}A_1^\frac{p_1(p - p_0)}{p(p_1 - p_0)}\|f\|_{L^p}\\
				&= 2\left( \frac{p}{p - p_0} + \frac{p}{p_1- p} \right)^{1/p} A_0^{\frac{p_0(p_1 - p)}{p(p_1 - p_0)}\frac{p_1}{p_1}}A_1^{\frac{p_1(p - p_0)}{p(p_1 - p_0)}\frac{p_0}{p_0}}\|f\|_{L^p}\\
				&= 2\left( \frac{p}{p - p_0} + \frac{p}{p_1- p} \right)^{1/p} A_0^{\frac{\frac{p_1 - p}{pp_1}}{\frac{p_1 - p_0}{p_0p_1}}}A_1^{\frac{\frac{p - p_0}{p_0p}}{\frac{p_1 - p_0}{p_0p_1}}}\|f\|_{L^p}\\
				&= 2\left( \frac{p}{p - p_0} + \frac{p}{p_1- p} \right)^{1/p} A_0^{\frac{\frac{1}{p} - \frac{1}{p_1}}{\frac{1}{p_0} - \frac{1}{p_1}}}A_1^{\frac{\frac{1}{p_0} - \frac{1}{p}}{\frac{1}{p_0} - \frac{1}{p_1}}}\|f\|_{L^p}
			\end{aligned}
		\end{gather}
	\end{enumerate}
	
\item
	\begin{enumerate}[label = \textbf{\alph*.}]
		\item Assume \underline{$p_1 = \infty$}. We again use the cut-off functions defined in (\ref{eq:cut_off}) to decompose $f$.  Since $\{\vert f_1\vert > \delta\alpha \} = \emptyset$, we have 

\begin{equation*}
	\|T(f_1)\|_{L^\infty} \leqslant A_1 \|f_1\|_{L^\infty} = A_1 \inf \left\{B > 0: \mu(\{\vert f_1 \vert > B\}) = 0 \right\} \leqslant A_1\delta\alpha = \alpha/2
\end{equation*}

Provided we stipulate $\delta := 1/(2A_1)$. Therefore the set $\{\vert T(f_1) \vert > \alpha/2\}$ has measure zero (this is immediate since $\|T(f_1)\|_{L^\infty} =  \inf \left\{B > 0: \mu(\{\vert T(f_1) \vert > B\}) = 0 \right\} \leqslant \alpha/2 $ and any subset of a set with measure zero has itself measure zero). Thus similar to part \textbf{b.} of \textbf{(i.)} we get $d_{T(f)}(\alpha) \leqslant d_{T(f_0)}(\alpha/2)$.

	\item Hypothesis (\ref{hyp:fp_0}) yields the estimate $\displaystyle d_{T(f_0)}(\alpha/2) \leqslant \left(\frac{A_0}{\alpha/2}\right)^{p_0} \int_{\{2A_1\vert f \vert > \alpha\}} \vert f \vert^{p_0}d\mu$.

	\item Thus by \textbf{a.} and \textbf{b.}

	\begin{gather}
		\begin{aligned}
			\|T(f)\|_{L^p}^p &= p \int_0^{\infty} \alpha^{p-1}d_{T(f)} d\lambda\\
			&\leqslant p (2A_0)^{p_0} \int_0^{\infty} \alpha^{p-p_0-1} \int_{\{2A_1\vert f \vert > \alpha\}} \vert f \vert^{p_0}d\mu d\lambda\\
			&= p(2A_0)^{p_0} \int_X \vert f\vert^{p_0} \int_0^{2A_1\vert f \vert} \alpha^{p - p_0 - 1}d\lambda d\mu\\
			&= \frac{2^ppA_0^{p_0}A_1^{p - p_0}}{p - p_0} \int_X \vert f\vert^{p} d\mu\\
			&= \frac{2^ppA_0^{p_0}A_1^{p - p_0}}{p - p_0} \|f\|_{L^p}^p
			\label{est:p_1_infty}
		\end{aligned}
	\end{gather}

	That the constant $2^ppA_0^{p_0}A_1^{p - p_0}/(p - p_0)$ found in (\ref{est:p_1_infty}) is the $p$-th power of the one stated in the theorem can be seen by passing the constant (\ref{eq:constant}) to the limit $p_1 \rightarrow \infty$:

	\begin{gather*}
		\begin{aligned}
			\lim\limits_{p_1 \rightarrow \infty} A =& \lim\limits_{p_1 \rightarrow \infty}	\left[2\left( \frac{p}{p - p_0} + \frac{p}{p_1 - p} \right)^{1/p}A_0^{\frac{\frac{1}{p} - \frac{1}{p_1}}{\frac{1}{p_0}-\frac{1}{p_1}}}A_1^{\frac{\frac{1}{p_0}-\frac{1}{p}}{\frac{1}{p_0}-\frac{1}{p_1}}}\right]\\
			=& 2\exp\left[ \frac{1}{p} \log\left(\frac{p}{p - p_0} + \lim\limits_{p_1 \rightarrow \infty} \frac{1}{p_1}\frac{p}{1 - p\lim\limits_{p_1 \rightarrow \infty}\frac{1}{p_1}}\right)\right]\\
			& \cdot \lim\limits_{p_1 \rightarrow \infty}A_0^{\frac{\frac{1}{p} - \frac{1}{p_1}}{\frac{1}{p_0}-\frac{1}{p_1}}}\cdot\lim\limits_{p_1 \rightarrow \infty}A_1^{\frac{\frac{1}{p_0}-\frac{1}{p}}{\frac{1}{p_0}-\frac{1}{p_1}}}\\
		=& 2\left( \frac{p}{p - p_0} \right)^{1/p} \exp\left[\displaystyle\frac{\frac{1}{p} - \lim\limits_{p_1 \rightarrow + \infty}\frac{1}{p_1}}{\frac{1}{p_0}-\lim\limits_{p_1 \rightarrow \infty}\frac{1}{p_1}}\log(A_0)\right]\\
		& \cdot \exp\left[\frac{\frac{1}{p_0}-\frac{1}{p}}{\frac{1}{p_0}-\lim\limits_{p_1 \rightarrow \infty}\frac{1}{p_1}}\log(A_1)\right]\\
		=& 2\left( \frac{p}{p - p_0} \right)^{1/p} A_0^{\frac{p_0}{p}} A_1^{1 - \frac{p_0}{p}}
		\end{aligned}
	\end{gather*}
	\end{enumerate}

\end{enumerate}
\end{proof}

%Complex method
\section{The Complex Method}
This theorem will unfortunately only be applicable to linear operators but will yield a more natural bound of the operator on the intermediate space. The proof will make strong use of complex variables technique. A major tool will be an application of the maximum modulus principle, known as \emph{Hadamard's three lines lemma}.

\subsection{Hadamard's Three Lines Lemma}

\begin{mdframed}
	\begin{lemma}{Hadamard's three lines lemma)}
		Let $F$ be an analytic function on the strip $S := \{z \in \mathbb{C}: 0 < \mathrm{Re}z < 1\}$, continuous and bounded on $\overline{S}$, such that $\vert F(z)\vert \leqslant B_0$ when $\mathrm{Re}z = 0$ and $\vert F(z) \vert \leqslant B_1$ when $\mathrm{Re}z = 1$, for some $0 < B_0,B_1 < \infty$. Then $\vert F(z) \vert \leqslant B_0^{1 - \theta}B_1^\theta$ when $\mathrm{Re}z = \theta$, for any $0 \leqslant \theta \leqslant 1$.
		\label{lemma:HTL}
	\end{lemma}
\end{mdframed}
				
\vspace{2mm}

\begin{proof}
	For $z \in \overline{S}$ define 

\begin{equation}
	G(z) := \frac{F(z)}{B_0^{1 - z}B_1^z} \qquad \forall n \in \mathbb{N}_{>0}: G_n(z) := G(z) e^{(z^2 - 1)/n} 
\end{equation}

Obviously, $G(z)$ and $G_n(z)$ are analytic functions on $S$ for $n \in \mathbb{N}_{>0}$\footnote{
						Recall, that a function $f$ is called \emph{analytic on $U$}, $U \subseteq \mathbb{C}$ open, if $f$ is analytic at every $z_0 \in U$, that is, there exists a power series $\sum_{n \in \mathbb{N}} a_n (z - z_0)^n$ and some $r > 0$, such that the series converges absolutely for $\vert z - z_0 \vert < r$, and such that for such $z$, we have $f(z) = \sum_{n \in \mathbb{N}} a_n (z - z_0)^n$ (as defined in \cite[68--69]{lang:complex_analysis:1993}). If $f$ and $g$ are analytic on $U \subseteq \mathbb{C}$, so are $f + g$, $f \cdot g$. Also $f/g$ is analytic on the open subset of $z \in U$ such that $g(z) \neq 0$. If $g:U \rightarrow V$ and $f: V \rightarrow C$ are analytic so is $f \circ g$. Further, if $f(z) = \sum_{n \in \mathbb{N}}a_n z^n$ is a power series with radius of convergence $r$, $f$ is analytic on $B_r(0)$ (for a proof see \cite[69--70]{lang:complex_analysis:1993}).	
					}. Further, we have

					\begin{gather}
						\begin{aligned}
							\vert B_0^{1 - z}B_1^z \vert^2 &= \vert B_0^{1 - z}\vert^2 \vert B_1^z \vert^2 \overset{(\dagger)}{=} B_0^{1 - z}B_0^{1 - \overline{z}} B_1^z B_1^{\overline{z}} = \left( B_0^{1 -\mathrm{Re}z} \right)^2 \left( B_1^{\mathrm{Re}z} \right)^2 
						\end{aligned}
					\end{gather}

					Consider $0 \leqslant \mathrm{Re} z \leqslant 1$ and $B_0 \geqslant 1$. Then  $B_0^{1 - \mathrm{Re}z} = \exp\left((1 - \mathrm{Re}z ) \log B_0\right) \geqslant 1$ and $B_0^{1 - \mathrm{Re} z } \geqslant B_0$ in the case $B_0 < 1$. A similar estimation of $B_1^{\mathrm{Re}z}$ leads to 

					\begin{equation}
						\vert B_0^{1 - z}B_1^z \vert \geqslant \min\{1,B_0\}\min\{1,B_1\}
					\end{equation}

					for all $z \in \overline{S}$. By this, $G(z)$ is bounded on $\overline{S}$ (by the boundedness of $F$). Let $M > 0$, such that $\vert G(z) \vert \leqslant M$ for $z \in \overline{S}$. Fix $n \in \mathbb{N}_{>0}$ and write $z := x + iy \in \overline{S}$. Since

					\begin{gather}
						\begin{aligned}
						\vert G_n(z)\vert^2 &= \vert G(z) \vert^2 \vert e^{((x + iy)^2 - 1)/n} \vert^2\\
						& \leqslant M^2 e^{(x^2 + 2ixy -y^2 - 1)/n} e^{(x^2 - 2ixy -y^2 - 1)/n}\\
						&= M^2 \left( e^{-y^2/n} \right)^2 \left(e^{(x^2 - 1)/n}\right)^2\\
						&\leqslant M^2 \left(e^{-y^2/n}\right)^2\\
						&= M^2 \left( e^{-\vert y \vert^2/n} \right)^2
						\end{aligned}
					\end{gather}

					we have $\lim_{y \rightarrow \pm \infty}\sup\{\vert G_n(z)\vert : x \in [0,1]\} = 0$ by the pinching-principle. Hence there exists some $C(n) \in \mathbb{R}_{>0}$, such that $\vert G_n(z) \vert \leqslant 1$ for all $\vert y \vert \geqslant C(n)$ and all $x \in [0,1]$. Consider the rectangle $R := [0,1] \times [-C(n),C(n)]$. Now $\vert G_n(z) \vert \leqslant 1$ on the lines $[0,1] \times \{\pm C(n)\}$ and since $\vert G(z) \vert = \vert F(z)\vert/B_0 \leqslant 1$, $\vert G(z) \vert = \vert F(z) \vert/B_1 \leqslant 1$ on the line $\{0\} \times [-C(n),C(n)]$ and $\{1\} \times [-C(n),C(n)]$ respectively by assumption, we have $\vert G_n(z) \vert \leqslant 1$ on $\partial S$. By the maximum modulus principle \footnote{
						The theorem can be found in\cite[91--92]{lang:complex_analysis:1993}. I will reproduce it here.

						\begin{lemma}\emph{(Maximum Modulus Principle, global version)}
							Let $U \subseteq \mathbb{C}$ be a connected open set, and let $f$ be an analytic function on $U$. If $z_0 \in U$ is a maximum point for $\vert f \vert$, that is $\vert f(z_0) \vert \geqslant \vert f(z) \vert$ for all $z \in U$, then $f$ is constant on $U$.
						\end{lemma}

						For our purpose the following corollary is more appropriate.

						\begin{corollary}
							Let $U \subseteq \mathbb{C}$ be a connected open set and $f$ be a continuous function on $\overline{U}$, analytic and non-constant on $U$. If $z_0 \in \overline{U}$ is a maximum for $f$, that is $\vert f(z_0) \vert \geqslant \vert f(z) \vert$ for all $z \in \overline{U}$, then $z_0 \in \partial U$.
						\end{corollary}
					}
					we have $\vert G_n(z) \vert \leqslant 1$ on $R$ and thus $\vert G_n(z) \vert \leqslant 1$ on $\overline{S}$. Since inequalities are preserved by limits and the modulus is a continuous function, we have that $\vert G(z) \vert = \lim_{n \rightarrow \infty} \vert G_n(z) \vert \leqslant 1$ on $\overline{S}$. Taking $z := \theta + it$, where $0 \leqslant \theta \leqslant 1$ and $t \in \mathbb{R}$, we conclude $\vert F(z) \vert = \vert G(z) \vert \vert B_0^{1 - z}B_1^z\vert \leqslant B_0^{1 - \theta} B_1^{\theta}$, which completes the proof.\\
					
					\emph{Proof of the equality $(\dagger)$.} For any $\alpha \in \mathbb{R}_{>0}$ and $z \in \mathbb{C}$ we have $\alpha^z = \exp(z \log(\alpha))$. Since the exponential function is convergent on the whole complex plane, for fixed $\varepsilon > 0$ we find $C \in \mathbb{N}$ such that $\vert \sum_{k = 0}^N \frac{z^k}{k!} - \exp(z) \vert < \varepsilon$ whenever $N > C$. But by the properties of the complex conjugate we get $\vert \sum_{k = 0}^N \frac{\overline{z}^k}{k!} - \overline{\exp(z)} \vert = \vert \overline{\sum_{k = 0}^N \frac{z^k}{k!} - \exp(z)} \vert =\vert \sum_{k = 0}^N \frac{z^k}{k!} - \exp(z) \vert < \varepsilon $. Therefore $\overline{\exp(z)} = \sum_{k \in \mathbb{N}} \frac{\overline{z}^k}{k!} = \exp(\overline{z})$ and thus $\overline{\alpha^z} = \alpha^{\overline{z}}$.
		\end{proof}


		\begin{remark}
			To apply the maximum modulus principle it is mandatory for $G_n$ to be non-constant. That the constant case is obviously true can be seen as follows. Assume $G_n(z) \equiv w \in \mathbb{C}$ for $z \in S$. This immediately implies $F(z) = w B_0^{1 - z}B_1^z e^{(1 - z^2)/n}$. Hence $F(z) = F(z;n)$. Thus the only possible case left is $w = 0$ and so $F \equiv 0$. But then the lemma holds trivially.
		\end{remark}

		\subsection{The Riesz-Thorin Interpolation Theorem}
		Now we are able to proove the Riesz-Thorin Interpolation theorem without an interruption.

\vspace{2mm}

\begin{mdframed}
	\begin{theorem}\emph{(Riesz-Thorin Interpolation Theorem)}
		Let $(X,\mathcal{A},\mu)$ be a measure space, $(Y,\mathcal{B},\nu)$ a $\sigma$-finite measure space and $T$ be a linear operator defined on the set of all finitely simple functions on $X$ and taking values in the set of measurable functions on $Y$. Let $1 \leqslant p_0,p_1,q_0,q_1 \leqslant \infty$ and assume that

		\begin{equation}
			\|T(f)\|_{L^{q_0}(Y,\mathcal{B},\nu)} \leqslant M_0\|f\|_{L^{p_0}(X,\mathcal{A},\mu)} \qquad \|T(f)\|_{L^{q_1}(Y,\mathcal{B},\nu)} \leqslant M_1\|f\|_{L^{p_1}(X,\mathcal{A},\mu)}
		\end{equation}

		holds for all finitely simple functions $f$ on $X$ and $0 < M_0,M_1 < \infty$. Then for all $0 < \theta < 1$ we have

		\begin{equation}
			\|T(f)\|_{L^q(Y,\mathcal{B},\nu)} \leqslant M_0^{1 - \theta}M_1^\theta\|f\|_{L^p(X,\mathcal{A},\mu)}
		\end{equation}

		for all finitely simple functions $f$ on $X$, where

		\begin{equation}
			\frac{1}{p} = \frac{1 - \theta}{p_0} + \frac{\theta}{p_1} \qquad \frac{1}{q} = \frac{1 - \theta}{q_0} + \frac{\theta}{q_1}
		\end{equation}
	\end{theorem}
\end{mdframed}
:
\begin{proof}
	We will use the fact that the $L^p(Y,\mathcal{B},\nu)$ norm of a function can be obtained via duality for $1 < p \leqslant \infty$ (for $p = \infty$ the underlying space has to be $\sigma$-finite according to \cite[288--289]{elstrodt:mass:2011}) by 
	
	\begin{equation*}
		\|f\|_{L^p(Y,\mathcal{B},\nu)} = \sup \left\{ \left\vert \int_Y fgd\nu\right\vert : \|g\|_{L^{p'}(Y,\mathcal{B},\nu)} = 1\right\}
	\end{equation*}

with $p' := \frac{p}{p - 1}$ for $p \in ]1,\infty[$ and $p' := 1$ for $p = \infty$. Since we will also make use of 

	\begin{equation*}
		\|f\|_{L^p(Y,\mathcal{B},\nu)} = \sup \left\{ \left\vert \int_Y fgd\nu\right\vert : \|g\|_{L^{p'}(Y,\mathcal{B},\nu)} \leqslant 1\right\}
	\end{equation*}

	I will proove their equivalence. If we define $\varphi_f(g): L^{p'}(Y,\mathcal{B},\nu) \rightarrow \mathbb{C}$, $\displaystyle \varphi_f(g) := \int_Y fgd\mu$, $\varphi_f$ is clearly a linear functional (to be precise, a continuous linear functional by \cite[289]{elstrodt:mass:2011}). Hence let $(V,\|\cdot\|)$ and $(W,\|\cdot\|)$ be two normed vector spaces over $\mathbb{C}$ and $L \in \mathrm{Hom}_{\mathbb{C}}(V,W)$ continuous. Then we define $v_n := \left( 1 - \frac{1}{n} \right)v$ for $v \in V$ with $\|v\| = 1$ and $n \in \mathbb{N}_{>0}$. We have $\|v_n\| = 1 - \frac{1}{n} \leqslant 1$. Thus $\|L(v_n)\| \leqslant \sup\{\|L(v)\|:\|v\| \leqslant 1\}$ and so $\lim_{n \rightarrow \infty} \|L(v_n)\| = \|L(v)\| \leqslant \sup\{\|L(v)\|:\|v\| \leqslant 1\}$. On the other hand we have $\|L(v)\| \leqslant \frac{1}{\|v\|}\|L(v)\| = \left\| L\left( \frac{v}{\|v\|} \right)\right\| \leqslant \sup\{\|L(v)\|:\|v\| = 1\}$ for any $v \in V$ with $\|v\| \leqslant 1$.\\
	Define $\mathfrak{F} := \mathrm{span}_{\mathbb{C}}\{\chi_E: E \in \mathcal{B},\nu(E) < \infty\}$, the set of all finitely simple functions on $Y$\footnote{
		This is almost trivial. Consider $Y_1,Y_2 \in \mathcal{B}$ with $\nu(Y_1),\nu(Y_2) < \infty$ and $Y_1 \cap Y_2 \neq \emptyset$. Then $f \equiv z_1\chi_{Y_1} + z_2\chi_{Y_2} \in \mathfrak{F}$ for $z_1,z_2 \in \mathbb{C}$. We see, that $f \equiv z_1 \chi_{Y_1\setminus Y_2} + z_2 \chi_{Y_2\setminus Y_1} + (z_1 + z_2)\chi_{Y_1 \cap Y_2} \in \mathfrak{F}$ where the latter function is a finitely simple one since $\nu(Y_1 \cup Y_2) \leqslant \nu(Y_1) + \nu(Y_2) < \infty$ and $Y_1\setminus Y_2,Y_2 \setminus Y_1, Y_1 \cap Y_2 \subseteq Y_1 \cup Y_2$.	
	}. Since $\mathfrak{F}$ is dense in $L^p(Y,\mathcal{B},\nu)$ for every $0 < p < \infty$\footnote{
		In \cite[242]{elstrodt:mass:2011} a proof can be found, that $\mathfrak{F}$ is dense in $\mathcal{L}^p$ for $0< p < \infty$. Now the canonical map $\pi: \mathcal{L}^p \rightarrow L^p/\mathcal{N}$ is continuous. Hence we may use the following lemma.

		\begin{lemma}
			Let $X$ and $Y$ be topological spaces, $f: X \rightarrow Y$ and $A \subseteq X$ dense in $X$. Then $f(A)$ is dense in $Y$. 
		\end{lemma}

		\begin{proof}
			By \cite[104]{munkres:topology:2000} we have $Y = f(X) = f(\overline{A})  \subseteq \overline{f(A)} \subseteq Y$.
		\end{proof}
		
		
	}, we may use the corollary found in \cite[76]{bourbaki:general_topology:1995}
	
	\vspace{2mm}
	
	\begin{mdframed}
		\begin{corollary}\emph{(Principle of extension of identities)}
			Let $f,g$ be two continuous mappings of a topological space $X$ into a Hausdorff space $Y$. If $f(x) = g(x)$ at all points of a dense subset of $X$, then $f \equiv g$.
		\end{corollary}
	\end{mdframed}
	
	\vspace{2mm}

	to see, that also

	\begin{equation*}
		\|f\|_{L^p(Y,\mathcal{B},\nu)} = \sup \left\{ \left\vert \int_Y fgd\mu\right\vert : g \in \mathfrak{F},\|g\|_{L^{p'}(Y,\mathcal{B},\nu)} \leqslant 1\right\}
	\end{equation*}

	Assume \underline{$q > 1$}. Fix $f :\equiv \sum_{k = 1}^n a_k e^{i\alpha_k}\chi_{X_k}$, where $n \in \mathbb{N}_{>0}$,$a_k > 0$, $\alpha_k \in [0,2\pi[$, $X_i \cap X_j = \emptyset$ for $i,j = 1,\hdots,n$ and $\mu(X_k) < \infty$ for every $k = 1,\hdots,n$. Further let $g :\equiv \sum_{k = 1}^m b_k e^{i\beta_k}\chi_{Y_k} \in \mathfrak{F}$, where $m \in \mathbb{N}_{>0}$,$b_k > 0$ and $\beta_k \in [0,2\pi[$. Define

				\begin{equation*}
					P(z) := \frac{p}{p_0}(1 - z) + \frac{p}{p_1}z \qquad Q(z) := \frac{q'}{q'_0}(1 - z) + \frac{q'}{q'_1}z
				\end{equation*}

				for $z \in \overline{S}$ (in the case $p = \infty$ we get also $p_0 = p_1 = \infty$ and hence by stipulating $\frac{\infty}{\infty}:= 1$ the function $P$ is well-defined). Further let
				
				\begin{equation}
					f_z :\equiv \sum_{k = 1}^n a^{P(z)}_k e^{i\alpha_k}\chi_{X_k} \qquad g_z :\equiv  \sum_{k = 1}^m b^{Q(z)}_k e^{i\beta_k}\chi_{Y_k}
					\label{def:fzgz}
				\end{equation}
				
				and 

				\begin{equation}
					F(z) := \int_Y T(f_z)(y)g_z(y)d\nu(y)
				\end{equation}

				By the linearity of the operator $T$ we have

				\begin{gather}
					F(z) = \sum_{j = 1}^n\sum_{k = 1}^m a^{P(z)}_j b_j^{Q(z)} e^{i\alpha_j} e^{i\beta_k} \int_YT(\chi_{X_j})(y)\chi_{Y_k}(y)d\nu(y) 
				\end{gather}

				and by using H\"older's inequality \footnote{A proof can be found in \cite[223]{elstrodt:mass:2011}.}

				\begin{gather}
					\begin{aligned}
						\left\vert \int_YT(\chi_{X_j})(y)\chi_{Y_k}(y)d\nu(y) \right\vert &\leqslant \int_Y\vert T(\chi_{X_j})(y)\vert \chi_{Y_k}(y)d\nu(y)\\
						&= \|T(\chi_{X_j})\chi_{Y_k}\|_{L^1(Y,\mathcal{B},\nu)}\\
						&\leqslant \|T(\chi_{X_j})\|_{L^{q_0}(Y,\mathcal{B},\nu)} \|\chi_{Y_k}\|_{L^{q_0'}(Y,\mathcal{B},\nu)}\\
						&\leqslant M_0\|\chi_{X_j}\|_{L^{p_0}(X,\mathcal{A},\mu)} \|\chi_{Y_k}\|_{L^{q_0'}(Y,\mathcal{B},\nu)}\\
						&= M_0 \left(\int_X \vert \chi_{X_j}(x) \vert^{p_0}d\mu(x)\right)^{1/{p_0}} \left( \int_Y \vert\chi_{Y_k}(y)\vert^{q_0'}d\nu(y) \right)^{1/{q_0'}}\\
						&=  M_0 \left(\int_X \chi_{X_j}(x) d\mu(x)\right)^{1/{p_0}} \left( \int_Y \chi_{Y_k}(y)d\nu(y) \right)^{1/{q_0'}}\\
						&= M_0 \mu(X_j)^{p_0} \nu(Y_k)^{q_0'}\\
						&< \infty
					\end{aligned}
				\end{gather}

				for $p_0 < \infty$ and each $j = 1,\hdots,n$, $k = 1,\hdots,m$ we get that $F(z)$ is analytic on $S$. The case $p_0,q_0' = \infty$ is trivial since $\|\chi_{X_j}\|_{L^\infty(X,\mathcal{A},\mu)},\|\chi_{Y_k}\|_{L^\infty(Y,\mathcal{B},\nu)}  \leqslant 1$. Now

				\begin{gather}
					\begin{aligned}
						\|f_{it}\|_{L^{p_0}(X,\mathcal{A},\mu)} &= \left(\sum_{k = 1}^n \int_X \vert f_{it} \vert^{p_0} d\mu + \int_{X \setminus \bigcup_{k = 1}^n X_k} \vert f_{it} \vert^{p_0} d\mu\right)^{1/p_0}\\
						&= \left(\sum_{k = 1}^n \vert a_k^{P(it)} e^{i\alpha_k}\vert^{p_0}\int_X \chi_{X_k} d\mu\right)^{1/p_0}\\
						&= \left(\sum_{k = 1}^n a_k^{p_0\mathrm{Re}P(it)}\mu(X_k)\right)^{1/p_0}\\
						&= \left(\sum_{k = 1}^n a_{k}^p\mu(X_k)\right)^{p/p_0p}\\
						&= \|f\|_{L^p(X,\mathcal{A},\mu)}^{p/p_0} 
					\end{aligned}
				\end{gather}

				for $p_0 \neq \infty$ and $p < \infty$. Let us consider $p_0 = \infty$. Then either $\|f_{it}\|_{L^{\infty}(X,\mathcal{A},\mu)} = 0$ or $\|f_{it}\|_{L^{\infty}(X,\mathcal{A},\mu)} = 1$. Since $\|\cdot\|_{L^p(X,\mathcal{A},\mu)}$ is a norm for $1 \leqslant p \leqslant \infty$ (see \cite[231]{elstrodt:mass:2011}), we have $f = 0 + \mathcal{N}$ if $\|f_{it}\|_{L^{\infty}(X,\mathcal{A},\mu)} = 0$. Since $f \in \mathfrak{F}$, we may conclude $f \equiv \sum_{k = 1}^n a_k e^{i\alpha_k}\chi_{X_k}$, where $\mu(X_k) = 0$ for $k = 1,\hdots,n$. But then $\|f_{it}\|_{L^{\infty}(X,\mathcal{A},\mu)} = \inf\{B > 0 : \mu(\{\vert f_{it} \vert > B\}) = 0\} =  \inf\{B > 0 : \mu(\{1 > B\}) = 0\}= 0$ since $\vert a_k^{P(it)}\vert = \lim_{p_0 \rightarrow \infty} a_k^{p/p_0} = 1$. In the other case we simply have $\|f_{it}\|_{L^{\infty}(X,\mathcal{A},\mu)} = 1$ since there exists at least one subset $X_k$ such that $\mu(X_k) \neq 0$. Now consider $p = \infty$. Then $p_0 = p_1 = \infty$. Thus $P(it) = 1$ and so $f_z \equiv f$. By the same considerations we see that $\|g_{it}\|_{L^{q_0'}(Y,\mathcal{B},\nu)} = \|g\|_{L^{q'}(Y,\mathcal{B},\nu)}^{q'/q'_0}$ for $q_0 \in [1,\infty]$ (set $\infty' := 1 $). Hence

				\begin{gather}
					\begin{aligned}
						\vert F(it) \vert &\leqslant \int_Y \vert T(f_{it})(y)g_{it}(y)\vert d\nu(y)\\
						&= \|T(f_{it}) g_{it}\|_{L^1(Y,\mathcal{B},\nu)}\\
						&\leqslant \|T(f_{it})\|_{L^{q_0}(Y,\mathcal{B},\nu)}\|g_{it}\|_{L^{q_0'}(Y,\mathcal{B},\nu)}\\
						&\leqslant M_0 \|f_{it}\|_{L^{p_0}(X,\mathcal{A},\mu)} \|g_{it}\|_{L^{q_0'}(Y,\mathcal{B},\nu)}\\
						&= M_0  \|f\|_{L^p(X,\mathcal{A},\mu)}^{p/p_0} \|g\|_{L^{q'}(Y,\mathcal{B},\nu)}^{q'/q'_0}
					\end{aligned}
				\end{gather}

				by H\"older's inequality. By similar calculations we get 
				
				\begin{equation}
					\|f_{1 + it}\|_{L^{p_1}(X,\mathcal{A},\mu)} = \|f\|_{L^p(X,\mathcal{A},\mu)}^{p/p_1} \qquad \|g_{1 + it}\|_{L^{q_1'}(Y,\mathcal{B},\nu)} = \|g\|_{L^{q'}(Y,\mathcal{B},\nu)}^{q'/q_1'}
				\end{equation}

				and thus 
				
				\begin{equation}
					\vert F(1 + it)\vert \leqslant M_1 \|f\|_{L^p(X,\mathcal{A},\mu)}^{p/p_1}\|g\|_{L^{q'}(Y,\mathcal{B},\nu)}^{q'/q_1'}
				\end{equation}	

					Since $F$ is analytic on $S$ and continuous on $\overline{S}$ and further 
		
		\begin{gather}
			\begin{aligned}
				\vert F(z)\vert &\leqslant \int_Y\vert T(f_z)(y)g_z(y)\vert d\nu(y)\\
				&= \|T(f_z)g_z\|_{L^1(Y,\mathcal{B},\nu)}\\
				&\leqslant \|T(f_z)\|_{L^{q_0}(Y,\mathcal{B},\nu)} \|g_z\|_{L^{q'_0}(Y,\mathcal{B},\nu)}\\
				&\leqslant M_0 \|f_z\|_{L^{p_0}(X,\mathcal{A},\mu)} \|g_z\|_{L^{q'_0}(Y,\mathcal{B},\nu)}\\
				&= M_0 \left(\int_X \vert f_z \vert^{p_0} d\mu\right)^{1/p_0} \left(\int_Y \vert g_z \vert^{q'_0} d\nu\right)^{1/q'_0}\\
				&= M_0 \left( \sum\limits_{j = 1}^n a_j^{\mathrm{Re}P(z)}\mu(X_j) \right)^{1/p_0} \left( \sum\limits_{k = 1}^m b_k^{\mathrm{Re}Q(z)} \nu(Y_k) \right)^{1/q'_0}\\
				&\leqslant M_0 \left( \sum\limits_{j = 1}^n a_j^{p/p_0 + p/p_1}\mu(X_j) \right)^{1/p_0} \left( \sum\limits_{k = 1}^m b_k^{q'/q_0' + q'/q'_1} \nu(Y_k) \right)^{1/q'_0}
			\end{aligned}
		\end{gather}
		
		
		by H\"older's inequality $F$ is bounded on $\overline{S}$ we can apply Hadamard's three lines lemma to get

		\begin{gather}
			\begin{aligned}
			\vert F(z) \vert &\leqslant \left( M_0  \|f\|_{L^p(X,\mathcal{A},\mu)}^{p/p_0} \|g\|_{L^{q'}(Y,\mathcal{B},\nu)}^{q'/q'_0} \right)^{1 - \theta}\left(  M_1 \|f\|_{L^p(X,\mathcal{A},\mu)}^{p/p_1}\|g\|_{L^{q'}(Y,\mathcal{B},\nu)}^{q'/q_1'} \right)^\theta\\
			&= M_0^{1 - \theta}M_1^\theta \|f\|_{L^p(X,\mathcal{A},\mu)}\|g\|_{L^{q'}(Y,\mathcal{B},\nu)}
			\label{est:F}
		\end{aligned}
		\end{gather}

		for $\mathrm{Re}z = \theta$ where $0 \leqslant \theta \leqslant 1$. Further observe $P(\theta) = Q(\theta) = 1$ and thus 
		
		\begin{gather}
			\begin{aligned}
				\|T(f)\|_{L^q(Y,\mathcal{B},\nu)} &= \sup\left\{\left\vert \int_Y T(f)gd\nu\right\vert : g \in \mathfrak{F}, \|g\|_{L^{q'}(Y,\mathcal{B},\nu)}\leqslant 1\right\}\\
				&=  \sup\left\{\left\vert F(\theta)\right\vert : g \in \mathfrak{F}, \|g\|_{L^{q'}(Y,\mathcal{B},\nu)} \leqslant 1\right\}\\
				&\leqslant M_0^{1 - \theta}M_1^\theta \|f\|_{L^p(X,\mathcal{A},\mu)}
				\label{id:F}
			\end{aligned}
		\end{gather}

		Now assume \underline{$q = 1$}. Then $q_0 = q_1 = 1$ and so $Q(z) = 1$ which implies $g_z \equiv g$ for every $z \in \overline{S}$. Assume, that $\|g\|_{L^\infty(Y,\mathcal{B},\nu)} \leqslant 1$. Then the above proof is also valid, if we take the supremum over the simple functions, instead of finitely simple functions, since by \cite[100]{cohn:measure_theory:2013} the simple functions are dense in $L^{\infty}(Y,\mathcal{B},\nu)$.		
\end{proof}

%Analytic families
\section{Interpolation of Analytic Families of Operators}
This generalization of the classical Riesz-Thorin theorem is due to \href{http://www.ams.org/journals/tran/1956-083-02/S0002-9947-1956-0082586-0/S0002-9947-1956-0082586-0.pdf}{Elias M. Stein}. Crucial for its proof is again a complex-analytic theorem which can be extended on the basis of Hadamard's three lines lemma.

\subsection{Extension of Hadamard's Three Lines Lemma}
This theorem is analogous to the one originally used by Stein itself and formulated by \href{http://download.springer.com/static/pdf/285/art\%253A10.1007\%252FBF02825637.pdf?originUrl=http\%3A\%2F\%2Flink.springer.com\%2Farticle\%2F10.1007\%2FBF02825637\&token2=exp=1470939579~acl=\%2Fstatic\%2Fpdf\%2F285\%2Fart\%25253A10.1007\%25252FBF02825637.pdf\%3ForiginUrl\%3Dhttp\%253A\%252F\%252Flink.springer.com\%252Farticle\%252F10.1007\%252FBF02825637*~hmac=d88bfe05b2cc8b0deed0f4781b8dfdd3701969606a6033727bfaf0c034cbd876}{ I. I. Hirschman, Jr.}

\vspace{2mm}

\begin{mdframed}
	\begin{lemma}\emph{(Hadamard's three lines lemma, extension)}
		Let $F$ be an analytic function in the strip $S := \{z \in \mathbb{C}: 0 < \mathrm{Re}z < 1\}$ and continuous on $\overline{S}$, such that for every $z \in \overline{S}$ we have $\log \vert F(z)\vert \leqslant A e^{\tau \vert \mathrm{Im}z \vert}$ for some $A < \infty$ and $\tau \in [0,\pi[$. Then

			\begin{equation*}
				\vert F(z) \vert \leqslant \exp\left( \frac{\sin(\pi x)}{2} \int_{-\infty}^\infty \left[ \frac{\log \vert F(it + iy)\vert}{\cosh(\pi t) - \cos(\pi x)} + \frac{\log \vert F(1 + it + iy)\vert}{\cosh(\pi t) + \cos(\pi x)} \right] d\lambda(t)\right)
			\end{equation*}

			whenever $z := x + iy \in S$.
	\end{lemma}
\end{mdframed}

\begin{proof}
	Since the proof is rather long we divide it as follows.

	\begin{enumerate}[label = \textbf{(\roman*.)}]
		\item Construct a holomorphic function $h$ defined on the open unit disc $D := \{z \in \mathbb{C} : \vert z \vert < 1\}$ with range $S$.
		\item Using the Poisson integral formula and the maximum principle for subharmonic functions find an upper bound for $\log\vert F \circ h\vert$ for $D$.
	\end{enumerate}

	\begin{enumerate}[label = \textbf{(\roman*.)}]
		\item Assume $F$ not identically zero (the case where $F$ is identically zero is trivial). Consider the function 

	\begin{equation}
		h(z) := \frac{1}{\pi i}\Log\left( i\frac{1 + z}{1 - z} \right)
	\end{equation}

	on $D$. Define $\psi(z) := i(1 + z)/(1 - z)$. If we write $z := x + iy \in D$, we have 

	\begin{equation}
		\psi(z) = \frac{-2y}{(1 - x)^2 + y^2} + i \frac{1 - x^2 - y^2}{(1 - x)^2 + y^2}
	\end{equation}

	Hence $\Im \psi(z) > 0$. Stipulating $x := 1 - y$ for $y$ satisfying $y^2 < y$, we get

	\begin{equation}
		\lim\limits_{y^2 < y, y \rightarrow 0^+} \Im \psi(z) = \lim\limits_{y^2 < y, y \rightarrow 0^+} \left( \frac{1}{y} - 1 \right) = \infty
	\end{equation}

	using the same definition of $x$ we get

	\begin{equation}
		\lim\limits_{y^2 < y, y \rightarrow 0^+} \Re \psi(z) = -\lim\limits_{y^2 < y, y \rightarrow 0^+} \frac{1}{y} = -\infty
	\end{equation}

	and by stipulating $x := 1 + y$

	\begin{equation}
		\lim\limits_{y^2 < -y, y \rightarrow 0^-} \Re \psi(z) = -\lim\limits_{y^2 < -y, y \rightarrow 0^-} \frac{1}{y} = \infty	
	\end{equation}


	Since $2i \neq 0$, $\psi$ is a linear fractional transformation (see \cite[279]{rudin:rc_analysis:1987}) with

	\begin{equation}
		\psi^{-1}(z) = \frac{z - i}{z + i}
	\end{equation}

	Therefore $\psi$ maps the unit circle $D$ onto the upper half plane. The principal value of $\log z$ denoted by $\Log z$ is defined by 

	\begin{equation}
		\Log z := \log \vert z \vert + i \Arg z \qquad z \neq 0
	\end{equation}

where $-\pi < \mathrm{Arg} z \leqslant \pi$ is the principal value of the argument of $z \neq 0$.  We see that $\pi i h(z)$ maps the upper half plane onto the strip $\mathbb{R} \times ]0,\pi[$. Thus $h(z)$ maps the unit circle $D$ onto the strip $]0,1[ \times \mathbb{R}$. By

\begin{equation}
	h'(z) = \frac{2}{\pi i} \frac{1}{z - 1}
\end{equation}

we see that $h$ is a holomorphic function on $D$. By composition, $F \circ h$ is holomorphic on $D$ and thus by \cite[336]{rudin:rc_analysis:1987} $\log\vert F \circ h \vert$ is subharmonic on $D$. It is easy to verify, that 

\begin{equation}
	h^{-1}(z) = \frac{e^{\pi i z} - i}{e^{\pi i z} + i}
\end{equation}
 on the unit strip $S$.\\

 \item Fix some $0 \leqslant R < 1$. Then $\log\vert F \circ h \vert$ is continuous (as the sum, product, quotient, composition of continuous functions) for $\vert z \vert = R$. Define 

\begin{gather*}
	\begin{aligned}
		H(re^{i\theta}):= \begin{cases}
			\displaystyle
			\log \vert F(h(Re^{i\theta}))\vert & r = R,\\
			\displaystyle
			\frac{1}{2\pi} \int_{-\pi}^\pi \log\vert F(h(Re^{it}))\vert \frac{R^2 - r^2}{R^2 - 2Rr\cos(\theta - t) + r^2} d\lambda(t) & 0 \leqslant r < R
	\end{cases}
	\end{aligned}
\end{gather*}

Then $H$ is continuous for $\vert z \vert \leqslant R$ and harmonic for $\vert z \vert < R$ (see \cite[234--235]{rudin:rc_analysis:1987}). Since $\log\vert F(h(Re^{i\theta}))\vert = H(Re^{i\theta})$, by \cite[336]{rudin:rc_analysis:1987} we have

\begin{equation}
	\log\vert F(h(re^{i\theta})) \leqslant \frac{1}{2\pi} \int_{-\pi}^\pi \log\vert F(h(Re^{it}))\vert \frac{R^2 - r^2}{R^2 - 2Rr\cos(\theta - t) + r^2} d\lambda(t) 
\end{equation}

Consider $e^{i\theta}$ where $\Arg e^{i\theta} \neq 0,\pi$, we have $\Im \psi(e^{i\theta}) = 0$ and hence $\psi(e^{i\theta}) \in \mathbb{R}$. But then either $\Re h(e^{i\theta}) = 0$, $\psi(e^{i\theta}) > 0$ or $\Re h(e^{i\theta}) = 1$, $\psi(e^{i\theta}) < 0$. Hence the growth property of the hypothesis implies

\begin{gather}
	\begin{aligned}
		\log \vert F(h(e^{i\theta})) \vert &\leqslant Ae^{\tau\vert \Im h(e^{i\theta})\vert}\\
		&= Ae^{\tau/\pi\vert \log\vert (1 + e^{i\theta})(1 - e^{i\theta})\vert\vert}\\
		&= A \left\vert \frac{1 + e^{i\theta}}{1 - e^{i\theta}} \right\vert^{\tau/\pi}
	\end{aligned}
\end{gather}

\item Fix some $re^{i\theta}$, $r < R$ and stipulate $x := h(re^{i\theta})$. Then we obtain

	\begin{gather}
		\begin{aligned}
			re^{i\theta} =& h^{-1}(x)\\
			=& \frac{e^{\pi i x}- i}{e^{\pi i x} + i}\\
			=& \frac{\cos(\pi x) + i\sin(\pi x) - i}{\cos(\pi x) + i\sin(\pi x) + i}\\
			=& \frac{\cos(\pi x) + i\sin(\pi x) - i}{\cos(\pi x) + i\sin(\pi x) + i}\frac{\cos(\pi x) - i\sin(\pi x) - i}{\cos(\pi x) - i\sin(\pi x) + i}\\
			=& -i \frac{\cos(\pi x)}{1 + \sin(\pi x)}
		\end{aligned}
	\end{gather}

	by

	\begin{gather*}
		\begin{aligned}
			\left(\cos(\pi x) + i\sin(\pi x) - i\right)\left(\cos(\pi x) - i\sin(\pi x) - i\right) =& \cos^2(\pi x) - i\sin(\pi x)\cos(\pi x)\\	
			& -i\cos(\pi x) + i\sin(\pi x)\cos(\pi x)\\
			& + \sin^2(\pi x) + \sin(\pi x)\\
			& -i \cos(\pi x) - \sin(\pi x) -1\\
			=& -2i \cos(\pi x)  
		\end{aligned}
	\end{gather*}

	and

	\begin{gather*}
		\begin{aligned}
			\left( \cos(\pi x) + i\sin(\pi x) + i \right)\left( \cos(\pi x) - i\sin(\pi x) - i \right) =& \cos^2(\pi x) - i\sin(\pi x)\cos(\pi x)\\
			& -i\cos(\pi x) + i\sin(\pi x)\cos(\pi x)\\
			& + \sin^2(\pi x) + \sin(\pi x)\\
			& + i \cos(\pi x) + \sin(\pi x) + 1\\
			=& 2 + 2\sin(\pi x)
		\end{aligned}
	\end{gather*}


	\end{enumerate}
\end{proof}

\subsection{Stein's Theorem on Interpolation of Analytic Families of Operators}

\begin{mdframed}
	\begin{definition}\emph{(Analytic family, admissible growth)}
		Let $(X,\mu)$, $(Y,\nu)$ be measure spaces and $\left( T_z \right)_{z \in \overline{S}}$, where $T_z$ is defined on the space of all finitely simple functions on $X$ and taking values in the space of all measurable functions on $Y$ such that

		\begin{equation}
			\int_Y \vert T_z(\chi_A)\chi_B \vert d\nu
		\end{equation}

		whenever $\mu(A),\nu(B) < \infty$. The family $\left( T_z \right)_{z \in \overline{S}}$ is said to be \emph{analytic} if for all $f$, $g$ finitely simple we have that

		\begin{equation}
			z \mapsto \int_Y T_z(f)gd\nu
		\end{equation}

		is analytic on $S$ and continuous on $\overline{S}$. Further, an analytic family $\left( T_z \right)_{z \in \overline{S}}$ is called of \emph{admissible growth}, if there is a constant $\tau \in [0,\pi[$, such that for all finitely simple functions $f$, $g$ a constant $C(f,g)$ exists with

			\begin{equation}
				\log\left\vert \int_Y T_z(f) g d\nu\right\vert \leqslant C(f,g)e^{\tau\vert \mathrm{Im}z\vert}
			\end{equation}

			for all $z \in \overline{S}$.
	\end{definition}
\end{mdframed}

\vspace{2mm}

\begin{mdframed}
	\begin{theorem}\emph{(Riesz-Thorin interpolation theorem, extension)}
		Let $\left( T_z \right)_{z \in \overline{S}}$ be an analytic family of admissible growth, $1 \leqslant p_0,p_1,q_0,q_1 \leqslant \infty$ and suppose that $M_0$, $M_1$ are positive functions on the real line such that for some $\tau \in [0,\pi[$

			\begin{equation}
				\sup\left\{e^{-\tau \vert y \vert} \log M_0(y) : y \in \mathbb{R}\right\} < \infty \qquad \sup\left\{e^{-\tau \vert y \vert} \log M_1(y) : y \in \mathbb{R}\right\} < \infty
			\end{equation}

			Fix $0 < \theta < 1$ and define

			\begin{equation}
				\frac{1}{p} := \frac{1 - \theta}{p_0} + \frac{\theta}{p_1} \qquad \frac{1}{q} := \frac{1 - \theta}{q_0} + \frac{\theta}{q_1}
			\end{equation}

			Further suppose that for all finitely simple functions $f$ on $X$ and $y \in \mathbb{R}$ we have

			\begin{equation}
				\|T_{iy}(y)\|_{L^{q_0}} \leqslant M_0(y)\|f\|_{L^{p_0}} \qquad \|T_{1 + iy}(y)\|_{L^{q_1}} \leqslant M_1(y)\|f\|_{L^{p_1}} 
			\end{equation}

			Then for all finitely simple functions $f$ on $X$ we have

			\begin{equation*}
				\|T_\theta(f)\|_{L^q} \leqslant M(\theta)\|f\|_{L^p}
			\end{equation*}

			where for $0 < x < 1$

			\begin{equation*}
				M(x) = \exp\left( \frac{\sin(\pi x)}{2} \int_{-\infty}^\infty \left[ \frac{\log M_0(t)}{\cosh(\pi t) - \cos(\pi x)} + \frac{\log M_1(t)}{\cosh(\pi t) + \cos(\pi x)}\right] d\lambda(t) \right)
			\end{equation*}
	\end{theorem}
\end{mdframed}

\begin{proof}
	Fix $0 < \theta < 1$ and finitely simple functions $f$, $g$ on $X$, $Y$ respectively with $\|f\|_{L^p} = \|g\|_{L^{q'}} = 1$. Define $f_z$, $g_z$ as in (\ref{def:fzgz}) and for $z \in \overline{S}$

	\begin{equation}
		F(z) := \int_Y T_z(f_z)g_z d\nu	
	\end{equation}

	Observe, that $\vert a^{P(z)}_j\vert \leqslant a_j^{p/p_0 + p/p_1}$ and $\vert b^{Q(z)}_k\vert \leqslant b_k^{q'/q'_0 + q'/q'_1}$ for $z \in \overline{S}$. Hence

	\begin{gather}
		\begin{aligned}
			\log \vert F(z) \vert &= \log \left\vert \sum_{j = 1}^n\sum_{k = 1}^m a^{P(z)}_j b_j^{Q(z)} e^{i\alpha_j} e^{i\beta_k} \int_YT_z(\chi_{X_j})(y)\chi_{Y_k}(y)d\nu(y)\right\vert\\
			&\leqslant \log \left( \sum_{j = 1}^n\sum_{k = 1}^m \vert a^{P(z)}_j\vert \vert b_j^{Q(z)}\vert \int_Y\vert T_z(\chi_{X_j})(y)\vert \chi_{Y_k}(y)d\nu(y)\right)\\
			&\leqslant  \log \left( \sum_{j = 1}^n\sum_{k = 1}^m a_j^{p/p_0 + p/p_1} b_k^{q'/q'_0 + q'/q'_1} \int_{Y_k}\vert T_z(\chi_{X_j})\vert d\nu\right)\\
		\end{aligned}
	\end{gather}
\end{proof}


\originalsectionstyle

\appendix

\printbibliography

\printindex
\end{document}

\section{The Complex Method}
This theorem will unfortunately only be applicable to linear operators but will yield a more natural bound of the operator on the intermediate space. The proof will make strong use of complex variables technique. A major tool will be an application of the maximum modulus principle, known as \emph{Hadamard's three lines lemma}.

\subsection{Hadamard's Three Lines Lemma}
A complex-valued function $f$ is said to be \emph{holomorphic} in $\Omega \subseteq \mathbb{C}$ open, if $f'(z)$ exists for any $z \in \Omega$. By a region we shall mean a nonempty connected open subset of the complex plane. The proof of Hadamard's three lines lemma heavily relies on a restatement of the maximum modulus theorem (see \cite[212]{rudin:rc_analysis:1987}) found in \cite[253]{rudin:rc_analysis:1987}.

\begin{theorem}
	Let $\Omega \subseteq \mathbb{C}$ be a bounded region and $f$ be a continuous function on $\overline{\Omega}$ which is holomorphic in $\Omega$. Then 

	\begin{equation*}
		\left| f(z)\right| \leqslant \sup\left\{ \left|f(z) \right| : z \in \partial\Omega\right\}
	\end{equation*}

	for every $z \in \Omega$. If equality holds at one point $z \in \Omega$, then $f$ is constant.
\end{theorem}

\begin{mdframed}
	\begin{lemma}\emph{(Hadamard's three lines lemma)}
		Let $F$ be a holomorphic function in the strip $S := \{z \in \mathbb{C}: 0 < \Re z < 1\}$, continuous and bounded on $\overline{S}$, such that $\left| F(z)\right| \leqslant B_0$ when $\Re z = 0$ and $\left| F(z) \right| \leqslant B_1$ when $\Re z = 1$, for some $0 < B_0,B_1 < \infty$. Then $\left| F(z) \right| \leqslant B_0^{1 - \theta}B_1^\theta$ when $\Re z = \theta$, for any $0 \leqslant \theta \leqslant 1$.
	\end{lemma}
\end{mdframed}

\begin{proof}
For $z \in \overline{S}$ define 

\begin{equation*}
	G(z) := \frac{F(z)}{B_0^{1 - z}B_1^z} \qquad G_n(z) := G(z) e^{\left(z^2 - 1\right)/n},~n \in\mathbb{N}_{>0}
\end{equation*}

$G(z)$ and $G_n(z)$ are holomorphic in $S$ by
	
\begin{equation*}
	G'(z) = \frac{F'(z) - F(z)\log\left( B_1/B_0 \right)}{B_0^{1 - z}B_1^z} \qquad G_n'(z) = G'(z)e^{\left( z^2 - 1 \right)/n} + \frac{2}{n}zG_n(z)	
\end{equation*}

and $e^z \neq 0$ for every $z \in \mathbb{C}$. Further, we have

\begin{equation*}
		\left| B_0^{1 - z}B_1^z \right| = \left(B_0^{1 - z}B_0^{1 - \overline{z}} B_1^z B_1^{\overline{z}}\right)^{1/2} =  B_0^{1 -\Re z}B_1^{\Re z}
\end{equation*}

Consider $0 \leqslant \Re z \leqslant 1$ and $B_0 \geqslant 1$. Then $B_0^{1 - \Re z} \geqslant 1$ and $B_0^{1 - \Re z } \geqslant B_0$ in the case $B_0 < 1$. Similarily, $B_1^{\Re z} \geqslant 1$ if $B_1 \geqslant 1$ and $B_1^{\Re z} \leqslant B_1$ if $B_1 < 1$. Hence 

\begin{equation}
	\left| B_0^{1 - z}B_1^z \right| \geqslant \min\left\{1,B_0\right\}\min\left\{1,B_1\right\} > 0
	\label{est:denom}
\end{equation}

for all $z \in \overline{S}$. Since $F$ is bounded on $\overline{S}$, we have $\left| F(z) \right| \leqslant L$ for some $L > 0$ and all $z \in \overline{S}$. Thus by (\ref{est:denom})

\begin{equation*}
	\left| G(z)\right| = \frac{\left| F(z)\right|}{\left| B_0^{1 - z}B_1^z \right|} \leqslant \frac{L}{\min\left\{1,B_0\right\}\min\left\{1,B_1\right\}} =: M
\end{equation*}

for every $z \in \overline{S}$. Fix $n \in \mathbb{N}_{>0}$ and write $z := x + iy \in \overline{S}$. Then

\begin{gather*}
	\left| G_n(z)\right| \leqslant M \left(e^{\left(x^2 + 2ixy -y^2 - 1\right)/n} e^{\left(x^2 - 2ixy -y^2 - 1\right)/n}\right)^{1/2}= M e^{-y^2/n}e^{\left(x^2 - 1\right)/n} \leqslant Me^{-y^2/n}
\end{gather*}

for $0 \leqslant x \leqslant 1$. Thus
	
\begin{equation*}
	\lim_{y \to \pm \infty}\sup\{\left| G_n(z)\right| : 0 \leqslant x \leqslant 1\} = 0
\end{equation*}

by the pinching-principle. Hence there exist $C_0,C_1 \in \mathbb{R}$, such that 

\begin{equation*}
	\sup\{\left| G_n(z)\right| : 0 \leqslant x \leqslant 1\} \leqslant 1
\end{equation*}

when $y > C_0$ or $y < C_1$. Letting

\begin{equation*}
	C(n) := \max\left\{ \left| C_0\right| + 1, \left| C_1 \right| + 1\right\}
\end{equation*}

we conclude $\left| G_n(z) \right| \leqslant 1$ for all $0 \leqslant x \leqslant 1$ when $\left| y \right| \geqslant C(n)$. Now consider the rectangle $R := \left(0,1\right) \times \left(-C(n),C(n)\right)$. We have $\left| G_n(z) \right| \leqslant 1$ on the lines $[0,1] \times \{\pm C(n)\}$. By

\begin{equation*}
	\left| G_n(iy)\right| = \frac{\left| F(iy)\right|}{\left| B_0^{1 - iy} B_1^{iy}\right|}e^{-\left( y^2 + 1 \right)/n} \leqslant 1 \qquad \left| G_n(1 + iy)\right| =	\frac{\left| F(1 + iy)\right|}{\left| B_0^{-iy}B_1^{1 + iy}\right|}e^{-y^2/n} \leqslant 1
\end{equation*}

we have $\left| G_n(z)\right| \leqslant 1$ on the lines $\{0\} \times [-C(n),C(n)]$, $\{1\} \times [-C(n),C(n)]$. Thus $\left| G_n(z) \right| \leqslant 1$ on $\partial R$. Since $\left| G_n(z)\right|$ is continuous on $\overline{R}$, holomorphic in $R$ and $R$ is a bounded region, the maximum modulus theorem implies

\begin{equation*}
	\left| G_n(z)\right| \leqslant \sup\left\{ \left|G_n(z) \right| : z \in \partial R \right\} \leqslant 1
\end{equation*}

for every $z \in R$. Therefore $\left| G_n(z) \right| \leqslant 1$ on $\overline{R}$ and so $\left| G_n(z) \right| \leqslant 1$ on $\overline{S}$. Since inequalities are preserved by limits and the modulus is a continuous function, we have that $\left| G(z) \right| = \lim_{n \to \infty} \left| G_n(z) \right| \leqslant 1$ for $z \in \overline{S}$. We conclude by 

\begin{equation*}
	\left| F(\theta + it) \right| = \left| G(\theta + it) \right| \left| B_0^{1 - \theta - it}B_1^{\theta + it}\right| \leqslant B_0^{1 - \theta} B_1^{\theta}
\end{equation*}

whenever $0 \leqslant \theta \leqslant 1$, $t \in \mathbb{R}$.

\begin{figure}[h!tb]
	\centering
	\begin{tikzpicture}
		\draw (0,-5)--(0,5);
		\draw (4,-5)--(4,5);
		\draw (-1,0)--(5,0);
		\draw [pattern = adjusted lines, pattern color = black, line width = .4mm] (0,-4) rectangle (4,4);
		\node (0) at (-.25,-.25) {$0$};
		\node (1) at (4.25,-.25) {$1$};
		\node (R) at (.5,3.55) {$\overline{R}$};
		\node (Cn) at (-.5,4) {$C(n)$};
		\node (-Cn) at (-.65,-4) {$-C(n)$};
		\node (R2) at (4.5,4.5) {$\mathbb{R}^2$};
	\end{tikzpicture}	
	\caption[Hadamard's three lines lemma: sketch of the rectangle $\overline{R}$]{Sketch of the rectangle $\overline{R}$.}
	\label{fig:Hadamards_three_lines_lemma}
\end{figure}
\end{proof}

\subsection{The Riesz-Thorin Interpolation Theorem}
For two measure spaces $\left( X,\mu \right)$, $\left( Y,\nu \right)$ let $\Sigma_X$ and $\Sigma_Y$ denote the set of all finitely simple functions on $X$, $Y$ respectively.

\vspace{2mm}

\begin{mdframed}
	\begin{theorem}\emph{(Riesz-Thorin Interpolation Theorem)}
		Suppose that $(X,\mu)$, $(Y,\nu)$ are measure spaces and $1 \leqslant p_0,p_1,q_0,q_1 \leqslant \infty$. If $q_0 = q_1 = \infty$, suppose also that $\nu$ is semifinite. Let $T$ be a linear operator defined on $\Sigma_X$ and taking values in the set of measurable functions on $Y$, such that

		\begin{equation}
			\left\|T(f)\right\|_{L^{q_0}} \leqslant M_0\left\|f\right\|_{L^{p_0}} \qquad \left\|T(f)\right\|_{L^{q_1}} \leqslant M_1\left\|f\right\|_{L^{p_1}}
			\label{hyp:Lq0Lq1}
		\end{equation}

		for all $f \in \Sigma_X$ and $0 < M_0,M_1 < \infty$. Then for all $0 < \theta < 1$ we have

		\begin{equation}
			\left\|T(f)\right\|_{L^q} \leqslant M_0^{1 - \theta}M_1^\theta\left\|f\right\|_{L^p}
			\label{est:boundTf}
		\end{equation}

		for all $f \in \Sigma_X$, where

		\begin{equation*}
			\frac{1}{p} = \frac{1 - \theta}{p_0} + \frac{\theta}{p_1} \qquad \frac{1}{q} = \frac{1 - \theta}{q_0} + \frac{\theta}{q_1}
		\end{equation*}
		\label{thm:Riesz_Thorin}
	\end{theorem}
\end{mdframed}

\begin{proof}
	If $f \in \Sigma_X$, $\left\| f \right\|_{L^p} = 0$, then $f = 0$ $\mu$-a.e. and either one of the hypotheses on $T$ in (\ref{hyp:Lq0Lq1}) implies $T(f) = 0$ $\mu$-a.e. and thus the estimate (\ref{est:boundTf}) holds trivially. Therefore we can assume $\left\| f\right\|_{L^p} \neq 0$. Fix 
	
\begin{equation*}
	f :\equiv \sum_{j = 1}^n a_j e^{i\alpha_j}\chi_{A_j} \in \Sigma_X \qquad g :\equiv \sum_{k = 1}^m b_k e^{i\beta_k}\chi_{B_k} \in \Sigma_Y
\end{equation*}

where $a_j, b_k > 0$ and $\alpha_j, \beta_k \in \mathbb{R}$ for every $j = 1,\hdots,n$, $k = 1,\hdots,m$ such that $\left\| g\right\|_{L^{q'}} \neq 0$ (recall $q' := q/\left( q - 1 \right)$). Define

\begin{equation*}
	P(z) := \frac{p}{p_0}(1 - z) + \frac{p}{p_1}z \qquad Q(z) := \frac{q'}{q'_0}(1 - z) + \frac{q'}{q'_1}z
\end{equation*}

for $z \in \mathbb{C}$ (if $p,q' = \infty$ then also $p_0,p_1,q_0',q_1' = \infty $ and hence $P$, $Q$ are well-defined). Further let
				
\begin{equation}
	f_z :\equiv \sum_{j = 1}^n a^{P(z)}_j e^{i\alpha_j}\chi_{A_j} \qquad g_z :\equiv  \sum_{k = 1}^m b^{Q(z)}_k e^{i\beta_k}\chi_{B_k}
	\label{def:fzgz}
\end{equation}
				
and 

\begin{equation}
	F(z) := \int_Y T(f_z)g_zd\nu
	\label{eq:def_F}
\end{equation}

By the linearity of the operator $T$ we have

\begin{gather*}
	F(z) = \sum_{j = 1}^n\sum_{k = 1}^m a^{P(z)}_j b_k^{Q(z)} e^{i\alpha_j} e^{i\beta_k} \int_YT(\chi_{A_j})\chi_{B_k}d\nu
\end{gather*}

and by H\"older's inequality

\begin{gather}
	\begin{aligned}
		\left| \int_YT(\chi_{A_j})\chi_{B_k}d\nu \right| &\leqslant \int_Y\left| T(\chi_{A_j})\chi_{B_k}\right|d\nu\\
		&= \left\|T(\chi_{A_j})\chi_{B_k}\right\|_{L^1}\\
		&\leqslant \left\|T(\chi_{A_j})\right\|_{L^{q_0}} \left\|\chi_{B_k}\right\|_{L^{q_0'}}\\
		&\leqslant M_0\left\|\chi_{A_j}\right\|_{L^{p_0}} \left\|\chi_{B_k}\right\|_{L^{q_0'}}\\
		&\leqslant M_0 \mu\left(A_j\right)^{1/p_0}\nu\left(B_k\right)^{1/q_0'}
	\end{aligned}
	\label{est:constant_F}
\end{gather}

for each $j = 1,\hdots,n$, $k = 1,\hdots,m$ (even in the cases where either $p_0 = \infty$ or $q_0' = \infty$, or both, by observing that $\left\| \chi_{A}\right\|_{L^\infty} \leqslant 1$ for any measurable set $A$). Thus the function $F$ is well-defined on $\mathbb{C}$. Let $t \in \mathbb{R}$. For $p,p_0 \neq \infty$

\begin{gather*}
	\begin{aligned}
		\left\|f_{it}\right\|_{L^{p_0}} &= \left(\sum_{j = 1}^n \int_{A_j} \left| f_{it} \right|^{p_0} d\mu + \int_{X \setminus \bigcup_{j = 1}^n A_j} \left| f_{it} \right|^{p_0} d\mu\right)^{1/p_0}\\
		&= \left(\sum_{j = 1}^n \left| a_j^{P(it)} e^{i\alpha_j}\right|^{p_0}\int_X \chi_{A_j} d\mu\right)^{1/p_0}\\
		&= \left(\sum_{j = 1}^n a_j^{p_0\Re P(it)}\mu\left(A_j\right)\right)^{1/p_0}\\
		&= \left(\sum_{j = 1}^n a_j^p\mu\left(A_j\right)\right)^{p/\left(p_0p\right)}\\
		&= \left\|f\right\|_{L^p}^{p/p_0} 
	\end{aligned}
\end{gather*}

holds. Let $p_0 = \infty$, $p \neq \infty$. Then $\left\|f_{it}\right\|_{L^{\infty}} = 1$ since $\left| a_j^{P(it)}\right| = a_j^{p/p_0} = 1$ and that there exists some index $j$, such that $\mu\left( A_j \right) \neq 0$. If $p = \infty$, observe that $P(z) = 1$ and thus $\left\| f_{it}\right\|_{L^{\infty}} = \left\| f\right\|_{L^{\infty}}$. By the same considerations we have $\|g_{it}\|_{L^{q_0'}} = \|g\|_{L^{q'}}^{q'/q'_0}$. Hence

\begin{gather*}
	\begin{aligned}
		\left| F(it) \right| &\leqslant \int_Y \left| T(f_{it})g_{it}\right| d\nu\\
		&= \left\|T(f_{it}) g_{it}\right\|_{L^1}\\
		&\leqslant \left\|T(f_{it})\right\|_{L^{q_0}}\left\|g_{it}\right\|_{L^{q_0'}}\\
		&\leqslant M_0 \left\|f_{it}\right\|_{L^{p_0}} \left\|g_{it}\right\|_{L^{q_0'}}\\
		&= M_0 \left\|f\right\|_{L^p}^{p/p_0} \left\|g\right\|_{L^{q'}}^{q'/q'_0}
	\end{aligned}
\end{gather*}

by H\"older's inequality. In an analogous manner we derive
				
\begin{equation*}
	\left\|f_{1 + it}\right\|_{L^{p_1}} = \left\|f\right\|_{L^p}^{p/p_1} \qquad \left\|g_{1 + it}\right\|_{L^{q_1'}} = \left\|g\right\|_{L^{q'}}^{q'/q_1'}
\end{equation*}

and thus 
				
\begin{equation*}
	\left| F(1 + it)\right| \leqslant M_1 \left\|f\right\|_{L^p}^{p/p_1}\left\|g\right\|_{L^{q'}}^{q'/q_1'}
\end{equation*}	

Further by estimate (\ref{est:constant_F}) 

\begin{equation*}
	\begin{aligned}
		\left| F(z)\right| &\leqslant \sum_{j = 1}^n\sum_{k = 1}^m \left| a_j^{P(z)}\right| \left| b_k^{Q(z)}\right| \left| \int_Y T(\chi_{A_j})\chi_{B_k} d\nu\right|\\
		&\leqslant M_0\sum_{j = 1}^n\sum_{k = 1}^m a_j^{\Re P(z)}b_k^{\Re Q(z)}\mu\left(A_j\right)^{1/p_0}\nu\left(B_k\right)^{1/q_0'}\\
		&\leqslant M_0\sum_{j = 1}^n\sum_{k = 1}^m \max\left\{ 1, a_j^{p/p_0 + p/p_1}\right\} \max\left\{1,b_k^{q'/q_0' + q'/q_1'}\right\}\mu\left(A_j\right)^{1/p_0}\nu\left(B_k\right)^{1/q_0'}
	\end{aligned}
\end{equation*}


Hence $F$ is bounded on $\overline{S}$ by some constant depending on $f$ and $g$ only. By 

\begin{multline*}
	F'(z) = \sum_{j = 1}^n\sum_{k = 1}^m a^{P(z)}_j\log \left( a_j \right) \left( \frac{p}{p_1} - \frac{p}{p_0} \right) b_k^{Q(z)}e^{i\alpha_j} e^{i\beta_k} \int_YT(\chi_{A_j})\chi_{B_k}d\nu \\
	+  \sum_{j = 1}^n\sum_{k = 1}^m a^{P(z)}_jb_k^{Q(z)}\log\left( b_k \right)\left( \frac{q'}{q'_1} - \frac{q'}{q'_0} \right)e^{i\alpha_j} e^{i\beta_k} \int_YT(\chi_{A_j})\chi_{B_k}d\nu 
\end{multline*}

	it is immediate, that $F$ is an entire function and thus holomorphic in $S$ and continuous on $\overline{S}$. Therefore, Hadamard's three lines lemma yields

\begin{gather*}
	\left| F(z) \right| \leqslant \left( M_0  \left\|f\right\|_{L^p}^{p/p_0} \left\|g\right\|_{L^{q'}}^{q'/q'_0} \right)^{1 - \theta}\left(  M_1 \left\|f\right\|_{L^p}^{p/p_1}\left\|g\right\|_{L^{q'}}^{q'/q_1'} \right)^\theta = M_0^{1 - \theta}M_1^\theta \left\|f\right\|_{L^p}\left\|g\right\|_{L^{q'}}
\end{gather*}

for $\Re z = \theta$, $0 < \theta < 1$. We have

\begin{equation*}
	\left\{ T(f) \neq 0\right\} = \bigcup_{n = 1}^\infty \left\{ \left| T(f)\right| > 1/n\right\}
\end{equation*}

and by Chebychev's inequality either

\begin{equation*}
	\nu\left( \left\{ \left| T(f)\right| > 1/n\right\} \right) \leqslant n^{q_0}\left\| T(f)\right\|_{L^{q_0}}^{q_0}
\end{equation*}

or

\begin{equation*}
	\nu\left( \left\{ \left| T(f)\right| > 1/n\right\} \right) \leqslant n^{q_1}\left\| T(f)\right\|_{L^{q_1}}^{q_1}
\end{equation*}

whenever $q_0 \neq \infty$ or $q_1 \neq \infty$. Therefore, the set $\left\{ T(f) \neq 0\right\}$ is $\sigma$-finite unless $q_0 = q_1 = \infty$. Further we have $P(\theta) = Q(\theta) = 1$. Thus by

\begin{gather*}
	\begin{aligned}
		M_q\left( T(f) \right) &= \sup\left\{\left| \int_Y T(f)gd\nu\right| : g \in \Sigma_Y, \left\|g\right\|_{L^{q'}} = 1\right\}\\
		&=  \sup\left\{\left| F(\theta)\right| : g \in \Sigma_Y, \left\|g\right\|_{L^{q'}} = 1\right\}\\
		&\leqslant M_0^{1 - \theta}M_1^\theta \left\|f\right\|_{L^p}\\
	\end{aligned}
\end{gather*}

we conclude 
	
\begin{equation*}
	\left\| T(f)\right\|_{L^q} = M_q\left( T(f) \right) \leqslant M_0^{1 - \theta}M_1^\theta \left\|f\right\|_{L^p}
\end{equation*}
	
for any $f \in \Sigma_X$.
\end{proof}

\subsection{Young's inequality}
Using the Riesz-Thorin interpolation theorem, we can give an alternative proof of Young's inequality \cite[22--23]{grafakos:fourier:2014}.

\vspace{2mm}

\begin{mdframed}
	\begin{theorem}\emph{(Young's inequality)}
		Let $G$ be a locally compact group, which is a countable union of compact subsets, and let $\eta$ be a left invariant Haar measure. Let $1 \leqslant p,q,r \leqslant \infty$

		\begin{equation}
			\frac{1}{q} + 1 = \frac{1}{p} + \frac{1}{r}
		\end{equation}

		Then for all $f \in L^p(G,\eta)$ and all $g \in L^r(G,\eta)$ satisfying $\|g\|_{L^r} = \|\tilde{g}\|_{L^r}$ we have $f \ast g$ exists $\eta$-a.e. and satisfies

		\begin{equation}
			\|f \ast g\|_{L^q} \leqslant \|g\|_{L^r}\|f\|_{L^p}
		\end{equation}
	\end{theorem}
\end{mdframed}

\begin{proof}
	Fix $g \in L^r(G,\eta)$ and let $T(f) := f \ast g$ be defined on $L^1(G,\eta) + L^{r'}(G,\eta)$. Obviously, $T$ is a linear operator by the linearity of the integral. By Minkowski's integral inequality (see exercise \textbf{1.1.6} \cite[13]{grafakos:fourier:2014}) we get

	\begin{gather}
		\begin{aligned}
			\|T(f)\|_{L^r} &= \left(\int_G \left\vert \int_G f(y)g(y^{-1}x) d\eta(y)\right\vert^r d\eta(x)\right)^{1/r}\\
			&\leqslant \int_G \left( \int_G \vert f(y) \vert^r \vert g(y^{-1}x) \vert^r d\eta(x)\right)^{1/r} d\eta(y)\\
			&= \int_G \vert f(y) \vert \left( \int_G \vert g(y^{-1}x) \vert^r d\eta(y^{-1}x) \right)^{1/r} d\eta(y)\\
			&= \int_G \vert f(y) \vert \left( \int_G \vert g(z) \vert^r d\eta(z) \right)^{1/r} d\eta(y)\\
			&\leqslant \|f\|_{L^1} \|g\|_{L^r}
		\end{aligned} 
	\end{gather}

	for $f \in L^1(g,\mu)$ and $1 \leqslant p < \infty$ (since $(G,\eta)$ is $\sigma$-finite). The case $r = \infty$ follows from
	
	\begin{equation}
		\vert (f \ast g)(x) \vert = \left\vert \int_G f(y)g(y^{-1}x) d\eta(y)\right\vert \leqslant \int_G \vert f(y)\vert \vert g(y^{-1}x)\vert d\eta(y) \leqslant \|g\|_{L^\infty}\|f\|_{L^1}
	\end{equation}
	
	By stipulating $h(y) := g(y^{-1}x)$ we have 

	\begin{gather}
		\begin{aligned}
			\vert (f \ast g)(x) \vert &= \left\vert \int_G f(y)g(y^{-1}x) d\eta(y)\right\vert \leqslant \int_G \vert f(y)g(y^{-1}x)\vert d\eta(y)\\
			&= \|fh\|_{L^1} \leqslant \|f\|_{L^{r'}} \|h\|_{L^r} = \|f\|_{L^{r'}} \|\tilde{g}\|_{L^r} = \|g\|_{L^r} \|f\|_{L^{r'}}
		\end{aligned}
	\end{gather}

	for $r < \infty$ and $f \in L^{r'}(g,\eta)$, since

	\begin{equation*}
		\|h\|^r_{L^r} = \int_G \vert g(y^{-1}x)\vert^r d\eta(y) = \int_G \vert\tilde{g}(x^{-1}y)\vert d\eta(y) = \|\tilde{g}\|^r_{L^{r}}
	\end{equation*}

	The Riesz-Thorin interpolation theorem now yields for any $0 < \theta < 1$

	\begin{equation}
		\|f \ast g\|_{L^q} = \|T(f)\|_{L^q} \leqslant \|g\|_{L^r}^{1 - \theta}\|g\|_{L^r}^{\theta} \|f\|_{L^p} = \|g\|_{L^r}\|f\|_{L^p}
	\end{equation}

	where 

	\begin{equation*}
		\frac{1}{p} = \frac{1 - \theta}{1} + \frac{\theta}{r'} \qquad \frac{1}{q} = \frac{1 - \theta}{r} + \frac{\theta}{\infty}
	\end{equation*}

	and by 

	\begin{equation*}
		\frac{1}{p} = 1 - \frac{\theta}{r} \qquad \frac{1}{q} = \frac{1}{r} - \frac{\theta}{r}
	\end{equation*}

	we get 

	\begin{equation*}
			\frac{1}{q} + 1 = \frac{1}{p} + \frac{1}{r}		
	\end{equation*}
\end{proof}

\begin{remark}
	The proof would be much shorter if we just used Minkowski's inequality \textup{\cite[21--22]{grafakos:fourier:2014}} instead of Minkowski's integral inequality. However, the proof given here is an alternative version of the one given already for Minkowski's inequality.
\end{remark}

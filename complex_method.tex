\section{The Complex Method}
This theorem will unfortunately only be applicable to linear operators but will yield a more natural bound of the operator on the intermediate space. The proof will make strong use of complex variables technique. A major tool will be an application of the maximum modulus principle, known as \emph{Hadamard's three lines lemma}.

\subsection{Hadamard's Three Lines Lemma}
As the name already says, the lemma yields a natural bound of an analytic function defined on a vertical strip in the complex plane using the bounds of the function on the two parallel lines enclosing the strip.

\vspace{2mm}

\begin{mdframed}
	\begin{lemma}{Hadamard's three lines lemma)}
		Let $F$ be a holomorphic function in the strip $S := \{z \in \mathbb{C}: 0 < \Re z < 1\}$, continuous and bounded on $\overline{S}$, such that $\left| F(z)\right| \leqslant B_0$ when $\Re z = 0$ and $\left| F(z) \right| \leqslant B_1$ when $\Re z = 1$, for some $0 < B_0,B_1 < \infty$. Then $\left| F(z) \right| \leqslant B_0^{1 - \theta}B_1^\theta$ when $\Re z = \theta$, for any $0 \leqslant \theta \leqslant 1$.
		\label{lem:HTL}
	\end{lemma}
\end{mdframed}
				
\vspace{2mm}

\begin{proof}
	For $z \in \overline{S}$ define 

\begin{equation}
	G(z) := \frac{F(z)}{B_0^{1 - z}B_1^z} \qquad \forall n \in \mathbb{N}_{>0}: G_n(z) := G(z) e^{(z^2 - 1)/n} 
\end{equation}

	Obviously, $G(z)$ and $G_n(z)$ are holomorphic functions on $S$ for $n \in \mathbb{N}_{>0}$\footnote{
	I adapt here the terminology established in \cite[197]{rudin:rc_analysis:1987}. A complex-valued function $f$ is said to be \emph{holomorphic} (or \emph{analytic}) in $\Omega \subseteq \mathbb{C}$ open, if $f'(z)$ exists for any $z \in \Omega$.	
					}. Further, we have

\begin{gather}
	\begin{aligned}
		\left| B_0^{1 - z}B_1^z \right|^2 &= \left| B_0^{1 - z}\right|^2 \left| B_1^z \right|^2 = B_0^{1 - z}B_0^{1 - \overline{z}} B_1^z B_1^{\overline{z}} = \left( B_0^{1 -\Re z} \right)^2 \left( B_1^{\Re z} \right)^2 
	\end{aligned}
\end{gather}

	Consider $0 \leqslant \Re z \leqslant 1$ and $B_0 \geqslant 1$. Then  $B_0^{1 - \mathrm{Re}z} = \exp\left((1 - \Re z ) \log B_0\right) \geqslant 1$ and $B_0^{1 - \Re z } \geqslant B_0$ in the case $B_0 < 1$. A similar estimation of $B_1^{\Re z}$ leads to 

\begin{equation}
	\left| B_0^{1 - z}B_1^z \right| \geqslant \min\{1,B_0\}\min\{1,B_1\}
\end{equation}

	for all $z \in \overline{S}$. By this, $G(z)$ is bounded on $\overline{S}$ (by the boundedness of $F$). Let $M > 0$, such that $\left| G(z) \right| \leqslant M$ for $z \in \overline{S}$. Fix $n \in \mathbb{N}_{>0}$ and write $z := x + iy \in \overline{S}$. Since

\begin{gather}
	\begin{aligned}
	\left| G_n(z)\right|^2 &= \left| G(z) \right|^2 \left| e^{((x + iy)^2 - 1)/n} \right|^2\\
	& \leqslant M^2 e^{(x^2 + 2ixy -y^2 - 1)/n} e^{(x^2 - 2ixy -y^2 - 1)/n}\\
	&= M^2 \left( e^{-y^2/n} \right)^2 \left(e^{(x^2 - 1)/n}\right)^2\\
	&\leqslant M^2 \left(e^{-y^2/n}\right)^2\\
	&= M^2 \left( e^{-\left| y \right|^2/n} \right)^2
	\end{aligned}
\end{gather}

	we have $\lim_{y \to \pm \infty}\sup\{\left| G_n(z)\right| : x \in [0,1]\} = 0$ by the pinching-principle. Hence there exists some $C(n) > 0$, such that $\left| G_n(z) \right| \leqslant 1$ for all $\left| y \right| \geqslant C(n)$ and all $x \in [0,1]$. Consider the rectangle $R := [0,1] \times [-C(n),C(n)]$. Now $\left| G_n(z) \right| \leqslant 1$ on the lines $[0,1] \times \{\pm C(n)\}$ and since $\left| G(z) \right| = \left| F(z)\right|/B_0 \leqslant 1$, $\left| G(z) \right| = \left| F(z) \right|/B_1 \leqslant 1$ on the line $\{0\} \times [-C(n),C(n)]$ and $\{1\} \times [-C(n),C(n)]$ respectively by assumption, we have $\left| G_n(z) \right| \leqslant 1$ on $\partial S$. By the maximum modulus principle \footnote{
						Let $\Omega$ be a bounded region of the complex plane, $f$ be a complex-valued continuous function on $\overline{\Omega}$ which is holomorphic in $\Omega$. Then $\left| f(z) \right| \leqslant \sup\left\{ \left| f(z) \right| : z \in \partial \Omega\right\}$ for every $z \in \Omega$. See \cite[253]{rudin:rc_analysis:1987}.}
	we have $\left| G_n(z) \right| \leqslant 1$ on $R$ and thus $\left| G_n(z) \right| \leqslant 1$ on $\overline{S}$. Since inequalities are preserved by limits and the modulus is a continuous function, we have that $\left| G(z) \right| = \lim_{n \to \infty} \left| G_n(z) \right| \leqslant 1$ on $\overline{S}$. Taking $z := \theta + it$, where $0 \leqslant \theta \leqslant 1$ and $t \in \mathbb{R}$, we conclude $\left| F(z) \right| = \left| G(z) \right| \left| B_0^{1 - z}B_1^z\right| \leqslant B_0^{1 - \theta} B_1^{\theta}$, which completes the proof.
\end{proof}

\subsection{The Riesz-Thorin Interpolation Theorem}
Now we are able to proove the Riesz-Thorin Interpolation theorem without an interruption. To simplify notation, let $\Sigma_X$, $\Sigma_Y$ denote the set of all finitely simple functions on $X$ and $Y$ respectively.

\vspace{2mm}

\begin{mdframed}
	\begin{theorem}\emph{(Riesz-Thorin Interpolation Theorem)}
		Let $(X,\mu)$ be a measure space, $(Y,\nu)$ a semifinite measure space and $T$ be a linear operator defined on $\Sigma_X$ and taking values in the set of measurable functions on $Y$. Let $1 \leqslant p_0,p_1,q_0,q_1 \leqslant \infty$ and assume that

		\begin{equation}
			\left\|T(f)\right\|_{L^{q_0}} \leqslant M_0\left\|f\right\|_{L^{p_0}} \qquad \left\|T(f)\right\|_{L^{q_1}} \leqslant M_1\left\|f\right\|_{L^{p_1}}
		\end{equation}

		for all $f \in \Sigma_X$ and $M_0,M_1 < \infty$. Then for all $0 < \theta < 1$ we have

		\begin{equation}
			\left\|T(f)\right\|_{L^q} \leqslant M_0^{1 - \theta}M_1^\theta\left\|f\right\|_{L^p}
		\end{equation}

		for all $f \in \Sigma_X$, where

		\begin{equation}
			\frac{1}{p} = \frac{1 - \theta}{p_0} + \frac{\theta}{p_1} \qquad \frac{1}{q} = \frac{1 - \theta}{q_0} + \frac{\theta}{q_1}
		\end{equation}
		\label{thm:Riesz_Thorin}
	\end{theorem}
\end{mdframed}

\begin{proof}
Fix 
	
\begin{equation*}
	f :\equiv \sum_{j = 1}^n a_j e^{i\alpha_j}\chi_{A_j} \in \Sigma_X \qquad g :\equiv \sum_{k = 1}^m b_k e^{i\beta_k}\chi_{B_k} \in \Sigma_Y
\end{equation*}

	where $a_j, b_k > 0$ and $\alpha_j, \beta_k \in \mathbb{R}$ for every $j = 1,\hdots,n$, $k = 1,\hdots,m$. Define

\begin{equation*}
	P(z) := \frac{p}{p_0}(1 - z) + \frac{p}{p_1}z \qquad Q(z) := \frac{q'}{q'_0}(1 - z) + \frac{q'}{q'_1}z
\end{equation*}

	for $z \in \overline{S}$ (if $p,q' = \infty$ then also $p_0,p_1,q_0',q_1' = \infty $ and hence $P$, $Q$ are well defined). Further let
				
\begin{equation}
	f_z :\equiv \sum_{j = 1}^n a^{P(z)}_j e^{i\alpha_j}\chi_{A_j} \qquad g_z :\equiv  \sum_{k = 1}^m b^{Q(z)}_k e^{i\beta_k}\chi_{B_k}
	\label{def:fzgz}
\end{equation}
				
and 

\begin{equation}
	F(z) := \int_Y T(f_z)(y)g_z(y)d\nu(y)
\end{equation}

By the linearity of the operator $T$ we have

\begin{gather}
	F(z) = \sum_{j = 1}^n\sum_{k = 1}^m a^{P(z)}_j b_k^{Q(z)} e^{i\alpha_j} e^{i\beta_k} \int_YT(\chi_{A_j})(y)\chi_{B_k}(y)d\nu(y) 
	\label{def:F}
\end{gather}

and by H\"older's inequality \footnote{A proof can be found in \cite[223]{elstrodt:mass:2011}.}

\begin{gather}
	\begin{aligned}
		\left| \int_YT(\chi_{A_j})(y)\chi_{B_k}(y)d\nu(y) \right| &\leqslant \int_Y\left| T(\chi_{A_j})(y)\chi_{B_k}(y)\right|d\nu(y)\\
		&= \left\|T(\chi_{A_j})\chi_{B_k}\right\|_{L^1}\\
		&\leqslant \left\|T(\chi_{A_j})\right\|_{L^{q_0}} \left\|\chi_{B_k}\right\|_{L^{q_0'}}\\
		&\leqslant M_0\left\|\chi_{A_j}\right\|_{L^{p_0}} \left\|\chi_{B_k}\right\|_{L^{q_0'}}\\
		&\overset{p_0,q_0' \neq \infty}{=} M_0\mu\left(A_j\right)^{1/p_0} \nu\left(B_k\right)^{1/q_0'}\\ 
		&< \infty
	\end{aligned}
\end{gather}

for each $j = 1,\hdots,n$, $k = 1,\hdots,m$. In the case where either $p_0 = \infty$ or $q_0' = \infty$, consider that $\left\|\chi_{A_j} \right\|_{L^\infty}, \left\|\chi_{B_k}\right\|_{L^\infty} \leqslant 1 $. Thus the function $F$ is well-defined on $\overline{S}$. Let $t \in \mathbb{R}$. For $p,p_0 \neq \infty$

\begin{gather}
	\begin{aligned}
		\left\|f_{it}\right\|_{L^{p_0}} &= \left(\sum_{j = 1}^n \int_X \left| f_{it} \right|^{p_0} d\mu + \int_{X \setminus \bigcup_{j = 1}^n A_j} \left| f_{it} \right|^{p_0} d\mu\right)^{1/p_0}\\
		&= \left(\sum_{j = 1}^n \left| a_j^{P(it)} e^{i\alpha_j}\right|^{p_0}\int_X \chi_{A_j} d\mu\right)^{1/p_0}\\
		&= \left(\sum_{j = 1}^n a_j^{p_0\Re P(it)}\mu\left(A_j\right)\right)^{1/p_0}\\
		&= \left(\sum_{j = 1}^n a_j^p\mu\left(A_j\right)\right)^{p/\left(p_0p\right)}\\
		&= \left\|f\right\|_{L^p}^{p/p_0} 
	\end{aligned}
\end{gather}

holds. Let $p_0 = \infty$, $p \neq \infty$. Then either $\left\|f_{it}\right\|_{L^{\infty}} = 0$ or $\left\|f_{it}\right\|_{L^{\infty}} = 1$. In the former case $f \equiv 0$ $\mu$-a.e which implies $\mu\left( A_j \right) = 0$ for any $j = 1,\hdots,n$ and thus $\left\| f_{it}\right\|_{L^{\infty}} = 0$ and in the latter case $\left\| f_{it} \right\|_{L^{\infty}} = 1$ by the simple observation that $\left| a_j^{P(it)}\right| = a_j^{p/p_0} = 1$ and that there exists some index $j$, such that $\mu\left( A_j \right) \neq 0$. If $p = \infty$, observe that $P(z) = 1$ and thus $\left\| f_{it}\right\|_{L^{\infty}} = \left\| f\right\|_{L^{\infty}}$. By the same considerations we see that $\|g_{it}\|_{L^{q_0'}} = \|g\|_{L^{q'}}^{q'/q'_0}$ any legitime $q_0,q$. Hence

\begin{gather}
	\begin{aligned}
		\left| F(it) \right| &\leqslant \int_Y \left| T(f_{it})(y)g_{it}(y)\right| d\nu(y)\\
		&= \left\|T(f_{it}) g_{it}\right\|_{L^1}\\
		&\leqslant \left\|T(f_{it})\right\|_{L^{q_0}}\left\|g_{it}\right\|_{L^{q_0'}}\\
		&\leqslant M_0 \left\|f_{it}\right\|_{L^{p_0}} \left\|g_{it}\right\|_{L^{q_0'}}\\
		&= M_0 \left\|f\right\|_{L^p}^{p/p_0} \left\|g\right\|_{L^{q'}}^{q'/q'_0}\\
		&< \infty
	\end{aligned}
\end{gather}

by H\"older's inequality. In an analogous manner s we can estimate 
				
\begin{equation}
	\left\|f_{1 + it}\right\|_{L^{p_1}} = \left\|f\right\|_{L^p}^{p/p_1} \qquad \left\|g_{1 + it}\right\|_{L^{q_1'}} = \left\|g\right\|_{L^{q'}}^{q'/q_1'}
\end{equation}

and thus 
				
\begin{equation}
	\left| F(1 + it)\right| \leqslant M_1 \left\|f\right\|_{L^p}^{p/p_1}\left\|g\right\|_{L^{q'}}^{q'/q_1'}
\end{equation}	

Further 
		
\begin{gather*}
	\begin{aligned}
		\left| F(z)\right| &\leqslant \int_Y\left| T(f_z)(y)g_z(y)\right| d\nu(y) = \left\|T(f_z)g_z\right\|_{L^1} \leqslant \left\|T(f_z)\right\|_{L^{q_0}} \left\|g_z\right\|_{L^{q'_0}}\\
				&\leqslant M_0 \left\|f_z\right\|_{L^{p_0}} \left\|g_z\right\|_{L^{q'_0}} \overset{p_0,q_0' \neq \infty}{=} M_0 \left(\int_X \left| f_z \right|^{p_0}d\mu \right)^{1/p_0} \left(\int_Y \left| g_z \right|^{q'_0} d\nu\right)^{1/q'_0}\\
				&= M_0 \left( \sum\limits_{j = 1}^n a_j^{p_0\Re P(z)}\mu(A_j) \right)^{1/p_0} \left( \sum\limits_{k = 1}^m b_k^{q'_0\Re Q(z)} \nu(B_k) \right)^{1/q'_0}\\
				&= M_0 \left( \sum\limits_{j = 1}^n a_j^{p\left(1 - \Re z\right) + \left(pp_0\Re z\right)/p_1}\mu(A_j) \right)^{1/p_0} \left( \sum\limits_{k = 1}^m b_k^{q'\left(1 - \Re z\right) + \left(q'q'_0\Re z\right)/q_1'} \nu(B_k) \right)^{1/q'_0}\\
				&\leqslant M_0 \left( \sum\limits_{j = 1}^n a_j^{p + \left(pp_0\right)/p_1}\mu(A_j) \right)^{1/p_0} \left( \sum\limits_{k = 1}^m b_k^{q' + \left(q'q'_0\right)/q'_1} \nu(B_k) \right)^{1/q'_0}\\
				&= M_0 \left\|f\right\|_{L^{p + \left(pp_0\right)/p_1}}^{p/p_0 + p/p_1} \left\|g\right\|_{L^{q' + \left(q'q'_0\right)/q'_1}}^{q'/q_0' + q'/q_1'} =: C(f,g)
			\end{aligned}
		\end{gather*}
		
		
by H\"older's inequality and in the edge cases
		
\begin{gather*}
	\begin{aligned}
		&p_0 = \infty, q_0' \neq \infty: \qquad C(f,g) := M_0 \max_{j = 1,\hdots,n} a_j^{p/p_1} \|g\|_{L^{q' + (q'q'_0)/q'_1}}^{q'/q_0' + q'/q_1'}\\
		&p_0 \neq \infty, q_0' = \infty: \qquad C(f,g) :=  M_0 \|f\|_{L^{p + (pp_0)/p_1}}^{p/p_0 + p/p_1} \max_{k = 1,\hdots,m} b_k^{q'/q_1'}\\
		&p_0 = \infty, q_0' = \infty: \qquad C(f,g) := M_0 \max_{j = 1,\hdots,n} a_j^{p/p_1} \max_{k = 1,\hdots,m} b_k^{q'/q_1'}
	\end{aligned}
\end{gather*}
		
	Hence $F$ is bounded on $\overline{S}$. By writing
	
\begin{equation*}
	F(z) = \sum_{j = 1}^n\sum_{k = 1}^m e^{P(z)\log\left( a_j \right)} e^{Q(z)\log\left( b_k \right)} e^{i\alpha_j} e^{i\beta_k} \int_YT(\chi_{A_j})(y)\chi_{B_k}(y)d\nu(y) 		
\end{equation*}
	
it is immediate that $F$ is continuous on $\overline{S}$ and by 

\begin{multline*}
	F'(z) = \sum_{j = 1}^n\sum_{k = 1}^m a^{P(z)}_j\log \left( a_j \right) \left( \frac{p}{p_1} - \frac{p}{p_0} \right) b_k^{Q(z)}\log\left( b_j \right)\left( \frac{q'}{q'_1} - \frac{q'}{q'_0} \right) e^{i\alpha_j} e^{i\beta_k} \\\int_YT(\chi_{A_j})(y)\chi_{B_k}(y)d\nu(y) 	
\end{multline*}

$F$ is holomorphic in $S$. Therefore Hadamard's three lines lemma (\ref{lem:HTL}) yields

\begin{gather}
	\begin{aligned}
		\left| F(z) \right| &\leqslant \left( M_0  \left\|f\right\|_{L^p}^{p/p_0} \left\|g\right\|_{L^{q'}}^{q'/q'_0} \right)^{1 - \theta}\left(  M_1 \left\|f\right\|_{L^p}^{p/p_1}\left\|g\right\|_{L^{q'}}^{q'/q_1'} \right)^\theta\\
			&= M_0^{1 - \theta}M_1^\theta \left\|f\right\|_{L^p}\left\|g\right\|_{L^{q'}}
		\label{est:F}
	\end{aligned}
\end{gather}

	for $\Re z = \theta$. By $P(\theta) = Q(\theta) = 1$ and

\begin{gather}
	\begin{aligned}
		M_q\left( T(f) \right) &= \sup\left\{\left| \int_Y T(f)gd\nu\right| : g \in \Sigma_Y, \left\|g\right\|_{L^{q'}} = 1\right\}\\
		&=  \sup\left\{\left| F(\theta)\right| : g \in \Sigma_Y, \left\|g\right\|_{L^{q'}} = 1\right\}\\
		&\leqslant M_0^{1 - \theta}M_1^\theta \left\|f\right\|_{L^p}
		\label{id:F}
	\end{aligned}
\end{gather}

	we conclude $\left\| T(f)\right\|_{L^q} = M_q\left( T(f) \right)$ for any $f \in \Sigma_X$ using \cite[189]{folland:real_analysis:1999} (observe, that $T(f)g \in L^1$ by either one of the hypotheses on the linear operator $T$).
\end{proof}

\begin{remark}
	It is necessary to have $0 < \theta < 1$, since for example choosing $q_1 = 1$ and $q_0 > 1$ arbitrary leads for $\theta = 1$ to $q = 1$ but then the function $g$ can be choosen so, that the integral in the definition \textup{(\ref{def:F})} is $\infty$.
\end{remark}


\subsection{Young's inequality}
Using the Riesz-Thorin interpolation theorem, we can give an alternative proof of Young's inequality \cite[22--23]{grafakos:fourier:2014}.

\vspace{2mm}

\begin{mdframed}
	\begin{theorem}\emph{(Young's inequality)}
		Let $G$ be a locally compact group, which is a countable union of compact subsets, and let $\eta$ be a left invariant Haar measure. Let $1 \leqslant p,q,r \leqslant \infty$

		\begin{equation}
			\frac{1}{q} + 1 = \frac{1}{p} + \frac{1}{r}
		\end{equation}

		Then for all $f \in L^p(G,\eta)$ and all $g \in L^r(G,\eta)$ satisfying $\|g\|_{L^r} = \|\tilde{g}\|_{L^r}$ we have $f \ast g$ exists $\eta$-a.e. and satisfies

		\begin{equation}
			\|f \ast g\|_{L^q} \leqslant \|g\|_{L^r}\|f\|_{L^p}
		\end{equation}
	\end{theorem}
\end{mdframed}

\begin{proof}
	Fix $g \in L^r(G,\eta)$ and let $T(f) := f \ast g$ be defined on $L^1(G,\eta) + L^{r'}(G,\eta)$. Obviously, $T$ is a linear operator by the linearity of the integral. By Minkowski's integral inequality (see exercise \textbf{1.1.6} \cite[13]{grafakos:fourier:2014}) we get

	\begin{gather}
		\begin{aligned}
			\|T(f)\|_{L^r} &= \left(\int_G \left\vert \int_G f(y)g(y^{-1}x) d\eta(y)\right\vert^r d\eta(x)\right)^{1/r}\\
			&\leqslant \int_G \left( \int_G \vert f(y) \vert^r \vert g(y^{-1}x) \vert^r d\eta(x)\right)^{1/r} d\eta(y)\\
			&= \int_G \vert f(y) \vert \left( \int_G \vert g(y^{-1}x) \vert^r d\eta(y^{-1}x) \right)^{1/r} d\eta(y)\\
			&= \int_G \vert f(y) \vert \left( \int_G \vert g(z) \vert^r d\eta(z) \right)^{1/r} d\eta(y)\\
			&\leqslant \|f\|_{L^1} \|g\|_{L^r}
		\end{aligned} 
	\end{gather}

	for $f \in L^1(g,\mu)$ and $1 \leqslant p < \infty$ (since $(G,\eta)$ is $\sigma$-finite). The case $r = \infty$ follows from
	
	\begin{equation}
		\vert (f \ast g)(x) \vert = \left\vert \int_G f(y)g(y^{-1}x) d\eta(y)\right\vert \leqslant \int_G \vert f(y)\vert \vert g(y^{-1}x)\vert d\eta(y) \leqslant \|g\|_{L^\infty}\|f\|_{L^1}
	\end{equation}
	
	By stipulating $h(y) := g(y^{-1}x)$ we have 

	\begin{gather}
		\begin{aligned}
			\vert (f \ast g)(x) \vert &= \left\vert \int_G f(y)g(y^{-1}x) d\eta(y)\right\vert \leqslant \int_G \vert f(y)g(y^{-1}x)\vert d\eta(y)\\
			&= \|fh\|_{L^1} \leqslant \|f\|_{L^{r'}} \|h\|_{L^r} = \|f\|_{L^{r'}} \|\tilde{g}\|_{L^r} = \|g\|_{L^r} \|f\|_{L^{r'}}
		\end{aligned}
	\end{gather}

	for $r < \infty$ and $f \in L^{r'}(g,\eta)$, since

	\begin{equation*}
		\|h\|^r_{L^r} = \int_G \vert g(y^{-1}x)\vert^r d\eta(y) = \int_G \vert\tilde{g}(x^{-1}y)\vert d\eta(y) = \|\tilde{g}\|^r_{L^{r}}
	\end{equation*}

	The Riesz-Thorin interpolation theorem now yields for any $0 < \theta < 1$

	\begin{equation}
		\|f \ast g\|_{L^q} = \|T(f)\|_{L^q} \leqslant \|g\|_{L^r}^{1 - \theta}\|g\|_{L^r}^{\theta} \|f\|_{L^p} = \|g\|_{L^r}\|f\|_{L^p}
	\end{equation}

	where 

	\begin{equation*}
		\frac{1}{p} = \frac{1 - \theta}{1} + \frac{\theta}{r'} \qquad \frac{1}{q} = \frac{1 - \theta}{r} + \frac{\theta}{\infty}
	\end{equation*}

	and by 

	\begin{equation*}
		\frac{1}{p} = 1 - \frac{\theta}{r} \qquad \frac{1}{q} = \frac{1}{r} - \frac{\theta}{r}
	\end{equation*}

	we get 

	\begin{equation*}
			\frac{1}{q} + 1 = \frac{1}{p} + \frac{1}{r}		
	\end{equation*}
\end{proof}

\begin{remark}
	The proof would be much shorter if we just used Minkowski's inequality \textup{\cite[21--22]{grafakos:fourier:2014}} instead of Minkowski's integral inequality. However, the proof given here is an alternative version of the one given already for Minkowski's inequality.
\end{remark}

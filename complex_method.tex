\section{The Complex Method}
This theorem will unfortunately only be applicable to linear operators but will yield a more natural bound of the operator on the intermediate space. The proof will make strong use of complex variables technique. A major tool will be an application of the maximum modulus principle, known as \emph{Hadamard's three lines lemma}.

\subsection{Hadamard's Three Lines Lemma}

\begin{mdframed}
	\begin{lemma}{Hadamard's three lines lemma)}
		Let $F$ be an analytic function on the strip $S := \{z \in \mathbb{C}: 0 < \mathrm{Re}z < 1\}$, continuous and bounded on $\overline{S}$, such that $\vert F(z)\vert \leqslant B_0$ when $\mathrm{Re}z = 0$ and $\vert F(z) \vert \leqslant B_1$ when $\mathrm{Re}z = 1$, for some $0 < B_0,B_1 < \infty$. Then $\vert F(z) \vert \leqslant B_0^{1 - \theta}B_1^\theta$ when $\mathrm{Re}z = \theta$, for any $0 \leqslant \theta \leqslant 1$.
		\label{lemma:HTL}
	\end{lemma}
\end{mdframed}
				
\vspace{2mm}

\begin{proof}
	For $z \in \overline{S}$ define 

\begin{equation}
	G(z) := \frac{F(z)}{B_0^{1 - z}B_1^z} \qquad \forall n \in \mathbb{N}_{>0}: G_n(z) := G(z) e^{(z^2 - 1)/n} 
\end{equation}

Obviously, $G(z)$ and $G_n(z)$ are analytic functions on $S$ for $n \in \mathbb{N}_{>0}$\footnote{
						Recall, that a function $f$ is called \emph{analytic on $U$}, $U \subseteq \mathbb{C}$ open, if $f$ is analytic at every $z_0 \in U$, that is, there exists a power series $\sum_{n \in \mathbb{N}} a_n (z - z_0)^n$ and some $r > 0$, such that the series converges absolutely for $\vert z - z_0 \vert < r$, and such that for such $z$, we have $f(z) = \sum_{n \in \mathbb{N}} a_n (z - z_0)^n$ (as defined in \cite[68--69]{lang:complex_analysis:1993}). If $f$ and $g$ are analytic on $U \subseteq \mathbb{C}$, so are $f + g$, $f \cdot g$. Also $f/g$ is analytic on the open subset of $z \in U$ such that $g(z) \neq 0$. If $g:U \rightarrow V$ and $f: V \rightarrow C$ are analytic so is $f \circ g$. Further, if $f(z) = \sum_{n \in \mathbb{N}}a_n z^n$ is a power series with radius of convergence $r$, $f$ is analytic on $B_r(0)$ (for a proof see \cite[69--70]{lang:complex_analysis:1993}).	
					}. Further, we have

					\begin{gather}
						\begin{aligned}
							\vert B_0^{1 - z}B_1^z \vert^2 &= \vert B_0^{1 - z}\vert^2 \vert B_1^z \vert^2 = B_0^{1 - z}B_0^{1 - \overline{z}} B_1^z B_1^{\overline{z}} = \left( B_0^{1 -\mathrm{Re}z} \right)^2 \left( B_1^{\mathrm{Re}z} \right)^2 
						\end{aligned}
					\end{gather}

					Consider $0 \leqslant \mathrm{Re} z \leqslant 1$ and $B_0 \geqslant 1$. Then  $B_0^{1 - \mathrm{Re}z} = \exp\left((1 - \mathrm{Re}z ) \log B_0\right) \geqslant 1$ and $B_0^{1 - \mathrm{Re} z } \geqslant B_0$ in the case $B_0 < 1$. A similar estimation of $B_1^{\mathrm{Re}z}$ leads to 

					\begin{equation}
						\vert B_0^{1 - z}B_1^z \vert \geqslant \min\{1,B_0\}\min\{1,B_1\}
					\end{equation}

					for all $z \in \overline{S}$. By this, $G(z)$ is bounded on $\overline{S}$ (by the boundedness of $F$). Let $M > 0$, such that $\vert G(z) \vert \leqslant M$ for $z \in \overline{S}$. Fix $n \in \mathbb{N}_{>0}$ and write $z := x + iy \in \overline{S}$. Since

					\begin{gather}
						\begin{aligned}
						\vert G_n(z)\vert^2 &= \vert G(z) \vert^2 \vert e^{((x + iy)^2 - 1)/n} \vert^2\\
						& \leqslant M^2 e^{(x^2 + 2ixy -y^2 - 1)/n} e^{(x^2 - 2ixy -y^2 - 1)/n}\\
						&= M^2 \left( e^{-y^2/n} \right)^2 \left(e^{(x^2 - 1)/n}\right)^2\\
						&\leqslant M^2 \left(e^{-y^2/n}\right)^2\\
						&= M^2 \left( e^{-\vert y \vert^2/n} \right)^2
						\end{aligned}
					\end{gather}

					we have $\lim_{y \rightarrow \pm \infty}\sup\{\vert G_n(z)\vert : x \in [0,1]\} = 0$ by the pinching-principle. Hence there exists some $C(n) \in \mathbb{R}_{>0}$, such that $\vert G_n(z) \vert \leqslant 1$ for all $\vert y \vert \geqslant C(n)$ and all $x \in [0,1]$. Consider the rectangle $R := [0,1] \times [-C(n),C(n)]$. Now $\vert G_n(z) \vert \leqslant 1$ on the lines $[0,1] \times \{\pm C(n)\}$ and since $\vert G(z) \vert = \vert F(z)\vert/B_0 \leqslant 1$, $\vert G(z) \vert = \vert F(z) \vert/B_1 \leqslant 1$ on the line $\{0\} \times [-C(n),C(n)]$ and $\{1\} \times [-C(n),C(n)]$ respectively by assumption, we have $\vert G_n(z) \vert \leqslant 1$ on $\partial S$. By the maximum modulus principle \footnote{
						The theorem can be found in\cite[91--92]{lang:complex_analysis:1993}. I will reproduce it here.

						\begin{lemma}\emph{(Maximum Modulus Principle, global version)}
							Let $U \subseteq \mathbb{C}$ be a connected open set, and let $f$ be an analytic function on $U$. If $z_0 \in U$ is a maximum point for $\vert f \vert$, that is $\vert f(z_0) \vert \geqslant \vert f(z) \vert$ for all $z \in U$, then $f$ is constant on $U$.
						\end{lemma}

						For our purpose the following corollary is more appropriate.

						\begin{corollary}
							Let $U \subseteq \mathbb{C}$ be a connected open set and $f$ be a continuous function on $\overline{U}$, analytic and non-constant on $U$. If $z_0 \in \overline{U}$ is a maximum for $f$, that is $\vert f(z_0) \vert \geqslant \vert f(z) \vert$ for all $z \in \overline{U}$, then $z_0 \in \partial U$.
						\end{corollary}
					}
					we have $\vert G_n(z) \vert \leqslant 1$ on $R$ and thus $\vert G_n(z) \vert \leqslant 1$ on $\overline{S}$. Since inequalities are preserved by limits and the modulus is a continuous function, we have that $\vert G(z) \vert = \lim_{n \rightarrow \infty} \vert G_n(z) \vert \leqslant 1$ on $\overline{S}$. Taking $z := \theta + it$, where $0 \leqslant \theta \leqslant 1$ and $t \in \mathbb{R}$, we conclude $\vert F(z) \vert = \vert G(z) \vert \vert B_0^{1 - z}B_1^z\vert \leqslant B_0^{1 - \theta} B_1^{\theta}$, which completes the proof.\\
		\end{proof}

	\subsection{The Riesz-Thorin Interpolation Theorem}
Now we are able to proove the Riesz-Thorin Interpolation theorem without an interruption.

\vspace{2mm}

\begin{mdframed}
	\begin{theorem}\emph{(Riesz-Thorin Interpolation Theorem)}
		Let $(X,\mu)$ be a measure space, $(Y,\nu)$ a $\sigma$-finite measure space and $T$ be a linear operator defined on the set of all finitely simple functions on $X$ and taking values in the set of measurable functions on $Y$. Let $1 \leqslant p_0,p_1,q_0,q_1 \leqslant \infty$ and assume that

		\begin{equation}
			\|T(f)\|_{L^{q_0}} \leqslant M_0\|f\|_{L^{p_0}} \qquad \|T(f)\|_{L^{q_1}} \leqslant M_1\|f\|_{L^{p_1}}
		\end{equation}

		holds for all finitely simple functions $f$ on $X$ and $0 < M_0,M_1 < \infty$. Then for all $0 < \theta < 1$ we have

		\begin{equation}
			\|T(f)\|_{L^q} \leqslant M_0^{1 - \theta}M_1^\theta\|f\|_{L^p}
		\end{equation}

		for all finitely simple functions $f$ on $X$, where

		\begin{equation}
			\frac{1}{p} = \frac{1 - \theta}{p_0} + \frac{\theta}{p_1} \qquad \frac{1}{q} = \frac{1 - \theta}{q_0} + \frac{\theta}{q_1}
		\end{equation}
	\end{theorem}
\end{mdframed}

\begin{proof}
	The proof heavily relies on the fact, that the $L^p$ norm of a function can be obtained via duality for $1 \leqslant p \leqslant \infty$ (for $p = \infty$ the underlying space has to be $\sigma$-finite according to \cite[288--289]{elstrodt:mass:2011}) by 
	
	\begin{equation*}
		\|f\|_{L^p} = \sup \left\{ \left\vert \int_Y fgd\nu\right\vert : \|g\|_{L^{p'}} = 1\right\}
	\end{equation*}

with $p' := \frac{p}{p - 1}$ for $p \in ]1,\infty[$ and $p' := 1$ for $p = \infty$. Let $\mathcal{B}$ denote the domain of $\nu$ and define for notational simplification $\mathfrak{F} := \mathrm{span}_{\mathbb{C}}\{\chi_E: E \in \mathcal{B},\nu(E) < \infty\}$, the set of all finitely simple functions on $Y$\footnote{
		This is almost trivial. Consider $Y_1,Y_2 \in \mathcal{B}$ with $\nu(Y_1),\nu(Y_2) < \infty$ and $Y_1 \cap Y_2 \neq \emptyset$. Then $f \equiv z_1\chi_{Y_1} + z_2\chi_{Y_2} \in \mathfrak{F}$ for $z_1,z_2 \in \mathbb{C}$. We see, that $f \equiv z_1 \chi_{Y_1\setminus Y_2} + z_2 \chi_{Y_2\setminus Y_1} + (z_1 + z_2)\chi_{Y_1 \cap Y_2} \in \mathfrak{F}$ where the latter function is a finitely simple one since $\nu(Y_1 \cup Y_2) \leqslant \nu(Y_1) + \nu(Y_2) < \infty$ and $Y_1\setminus Y_2,Y_2 \setminus Y_1, Y_1 \cap Y_2 \subseteq Y_1 \cup Y_2$.	
	}. Since $\mathfrak{F}$ is dense in $L^p$ for every $0 < p < \infty$\footnote{
		In \cite[242]{elstrodt:mass:2011} a proof can be found, that $\mathfrak{F}$ is dense in $\mathcal{L}^p$ for $0< p < \infty$. Now the canonical map $\pi: \mathcal{L}^p \rightarrow L^p/\mathcal{N}$ is continuous. Hence we may use the following lemma.

		\begin{lemma}
			Let $X$ and $Y$ be topological spaces, $f: X \rightarrow Y$ and $A \subseteq X$ dense in $X$. Then $f(A)$ is dense in $Y$. 
		\end{lemma}

		\begin{proof}
			By \cite[104]{munkres:topology:2000} we have $Y = f(X) = f(\overline{A})  \subseteq \overline{f(A)} \subseteq Y$.
		\end{proof}
		
		
	}, we may use the corollary found in \cite[76]{bourbaki:general_topology:1995}
	
	\begin{corollary}\emph{(Principle of extension of identities)}
		Let $f,g$ be two continuous mappings of a topological space $X$ into a Hausdorff space $Y$. If $f(x) = g(x)$ at all points of a dense subset of $X$, then $f \equiv g$.
	\end{corollary}

	to see, that also

	\begin{equation*}
		\|f\|_{L^p} = \sup \left\{ \left\vert \int_Y fgd\mu\right\vert : g \in \mathfrak{F},\|g\|_{L^{p'}} = 1\right\}
	\end{equation*}

	First we will deal with the case \underline{$q > 1$}. Fix $f :\equiv \sum_{k = 1}^n a_k e^{i\alpha_k}\chi_{X_k}$, where $n \in \mathbb{N}_{>0}$,$a_k > 0$, $\alpha_k \in [0,2\pi[$, $X_i \cap X_j = \emptyset$ for $i,j = 1,\hdots,n$ and $\mu(X_k) < \infty$ for every $k = 1,\hdots,n$. Further let $g :\equiv \sum_{k = 1}^m b_k e^{i\beta_k}\chi_{Y_k} \in \mathfrak{F}$, where $m \in \mathbb{N}_{>0}$,$b_k > 0$ and $\beta_k \in [0,2\pi[$. Define

				\begin{equation*}
					P(z) := \frac{p}{p_0}(1 - z) + \frac{p}{p_1}z \qquad Q(z) := \frac{q'}{q'_0}(1 - z) + \frac{q'}{q'_1}z
				\end{equation*}

				for $z \in \overline{S}$ (in the case $p = \infty$ we get also $p_0 = p_1 = \infty$ and hence by stipulating $\frac{\infty}{\infty}:= 1$ the function $P$ is well-defined). Further let
				
				\begin{equation}
					f_z :\equiv \sum_{k = 1}^n a^{P(z)}_k e^{i\alpha_k}\chi_{X_k} \qquad g_z :\equiv  \sum_{k = 1}^m b^{Q(z)}_k e^{i\beta_k}\chi_{Y_k}
					\label{def:fzgz}
				\end{equation}
				
				and 

				\begin{equation}
					F(z) := \int_Y T(f_z)(y)g_z(y)d\nu(y)
				\end{equation}

				By the linearity of the operator $T$ we have

				\begin{gather}
					F(z) = \sum_{j = 1}^n\sum_{k = 1}^m a^{P(z)}_j b_j^{Q(z)} e^{i\alpha_j} e^{i\beta_k} \int_YT(\chi_{X_j})(y)\chi_{Y_k}(y)d\nu(y) 
					\label{def:F}
				\end{gather}

				and by using H\"older's inequality \footnote{A proof can be found in \cite[223]{elstrodt:mass:2011}.}

				\begin{gather}
					\begin{aligned}
						\left\vert \int_YT(\chi_{X_j})(y)\chi_{Y_k}(y)d\nu(y) \right\vert &\leqslant \int_Y\vert T(\chi_{X_j})(y)\vert \chi_{Y_k}(y)d\nu(y)\\
						&= \|T(\chi_{X_j})\chi_{Y_k}\|_{L^1}\\
						&\leqslant \|T(\chi_{X_j})\|_{L^{q_0}} \|\chi_{Y_k}\|_{L^{q_0'}}\\
						&\leqslant M_0\|\chi_{X_j}\|_{L^{p_0}} \|\chi_{Y_k}\|_{L^{q_0'}}\\
						&\overset{q_0' \neq \infty}{=} M_0\|\chi_{X_j}\|_{L^{p_0}} \mu(Y_k)^{1/q_0'}\\ 
						&< \infty
					\end{aligned}
				\end{gather}

				for each $j = 1,\hdots,n$, $k = 1,\hdots,m$. In the case $q_0' = \infty$ we simply have $\|\chi_{Y_k}\|_{L^\infty} \leqslant 1 $. Thus the function $F$ is well-defined on $\overline{S}$. Now

				\begin{gather}
					\begin{aligned}
						\|f_{it}\|_{L^{p_0}} &= \left(\sum_{k = 1}^n \int_X \vert f_{it} \vert^{p_0} d\mu + \int_{X \setminus \bigcup_{k = 1}^n X_k} \vert f_{it} \vert^{p_0} d\mu\right)^{1/p_0}\\
						&= \left(\sum_{k = 1}^n \vert a_k^{P(it)} e^{i\alpha_k}\vert^{p_0}\int_X \chi_{X_k} d\mu\right)^{1/p_0}\\
						&= \left(\sum_{k = 1}^n a_k^{p_0\mathrm{Re}P(it)}\mu(X_k)\right)^{1/p_0}\\
						&= \left(\sum_{k = 1}^n a_{k}^p\mu(X_k)\right)^{p/(p_0p)}\\
						&= \|f\|_{L^p}^{p/p_0} 
					\end{aligned}
				\end{gather}

				for $p,p_0 \neq \infty$. Let us consider $p_0 = \infty$, $p \neq \infty$. Then either $\|f_{it}\|_{L^{\infty}} = 0$ or $\|f_{it}\|_{L^{\infty}} = 1$. Since $\|\cdot\|_{L^p}$ is a norm for $1 \leqslant p \leqslant \infty$ (see \cite[231]{elstrodt:mass:2011}), we have $f = 0$ $\mu$-a.e. if $\|f_{it}\|_{L^{\infty}} = 0$. Since $f$ is finitely simple, we may conclude $f \equiv \sum_{k = 1}^n a_k e^{i\alpha_k}\chi_{X_k}$, where $\mu(X_k) = 0$ for $k = 1,\hdots,n$. But then $\|f_{it}\|_{L^{\infty}} = \inf\{B > 0 : \mu(\{\vert f_{it} \vert > B\}) = 0\} =  \inf\{B > 0 : \mu(\{1 > B\}) = 0\}= 0$ since $\vert a_k^{P(it)}\vert = \lim_{p_0 \rightarrow \infty} a_k^{p/p_0} = 1$. In the other case we simply have $\|f_{it}\|_{L^{\infty}} = 1$ since there exists at least one subset $X_k$ such that $\mu(X_k) \neq 0$. Now consider $p = \infty$. Then $p_0 = p_1 = \infty$. Thus $P(it) = 1$ and so $f_z \equiv f$ and the equation holds trivially. By the same considerations we see that $\|g_{it}\|_{L^{q_0'}} = \|g\|_{L^{q'}}^{q'/q'_0}$ for $q_0 \in [1,\infty]$ (set $\infty' := 1 $). Hence

				\begin{gather}
					\begin{aligned}
						\vert F(it) \vert &\leqslant \int_Y \vert T(f_{it})(y)g_{it}(y)\vert d\nu(y)\\
						&= \|T(f_{it}) g_{it}\|_{L^1}\\
						&\leqslant \|T(f_{it})\|_{L^{q_0}}\|g_{it}\|_{L^{q_0'}}\\
						&\leqslant M_0 \|f_{it}\|_{L^{p_0}} \|g_{it}\|_{L^{q_0'}}\\
						&= M_0 \|f\|_{L^p}^{p/p_0} \|g\|_{L^{q'}}^{q'/q'_0}\\
						&< \infty
					\end{aligned}
				\end{gather}

				by H\"older's inequality. By similar calculations we get 
				
				\begin{equation}
					\|f_{1 + it}\|_{L^{p_1}} = \|f\|_{L^p}^{p/p_1} \qquad \|g_{1 + it}\|_{L^{q_1'}} = \|g\|_{L^{q'}}^{q'/q_1'}
				\end{equation}

				and thus 
				
				\begin{equation}
					\vert F(1 + it)\vert \leqslant M_1 \|f\|_{L^p}^{p/p_1}\|g\|_{L^{q'}}^{q'/q_1'}
				\end{equation}	

				Further 
		
		\begin{gather*}
			\begin{aligned}
				\vert F(z)\vert &\leqslant \int_Y\vert T(f_z)(y)g_z(y)\vert d\nu(y)\\
				&= \|T(f_z)g_z\|_{L^1}\\
				&\leqslant \|T(f_z)\|_{L^{q_0}} \|g_z\|_{L^{q'_0}}\\
				&\leqslant M_0 \|f_z\|_{L^{p_0}} \|g_z\|_{L^{q'_0}}\\
				&\overset{p_0,q_0' \neq \infty}{=} M_0 \left(\int_X \vert f_z \vert^{p_0} d\mu\right)^{1/p_0} \left(\int_Y \vert g_z \vert^{q'_0} d\nu\right)^{1/q'_0}\\
				&= M_0 \left( \sum\limits_{j = 1}^n a_j^{p_0\mathrm{Re}P(z)}\mu(X_j) \right)^{1/p_0} \left( \sum\limits_{k = 1}^m b_k^{q'_0\mathrm{Re}Q(z)} \nu(Y_k) \right)^{1/q'_0}\\
				&= M_0 \left( \sum\limits_{j = 1}^n a_j^{p(1 - \mathrm{Re}z) + (pp_0\mathrm{Re}z)/p_1}\mu(X_j) \right)^{1/p_0} \left( \sum\limits_{k = 1}^m b_k^{q'(1 - \mathrm{Re}z) + (q'q'_0\mathrm{Re} z)/q_1'} \nu(Y_k) \right)^{1/q'_0}\\
				&\leqslant M_0 \left( \sum\limits_{j = 1}^n a_j^{p + (pp_0)/p_1}\mu(X_j) \right)^{1/p_0} \left( \sum\limits_{k = 1}^m b_k^{q' + (q'q'_0)/q'_1} \nu(Y_k) \right)^{1/q'_0}\\
				&= M_0 \|f\|_{L^{p + (pp_0)/p_1}}^{p/p_0 + p/p_1} \|g\|_{L^{q' + (q'q'_0)/q'_1}}^{q'/q_0' + q'/q_1'}\\
				&=: C(f,g)
			\end{aligned}
		\end{gather*}
		
		
		by H\"older's inequality and in the edge cases
		
		\begin{gather*}
			\begin{aligned}
				&p_0 = \infty, q_0' \neq \infty: \qquad C(f,g) := M_0 \max_{j = 1,\hdots,n} a_j^{p/p_1} \|g\|_{L^{q' + (q'q'_0)/q'_1}}^{q'/q_0' + q'/q_1'}\\
				&p_0 \neq \infty, q_0' = \infty: \qquad C(f,g) :=  M_0 \|f\|_{L^{p + (pp_0)/p_1}}^{p/p_0 + p/p_1} \max_{k = 1,\hdots,m} b_k^{q'/q_1'}\\
				&p_0 = \infty, q_0' = \infty: \qquad C(f,g) := M_0 \max_{j = 1,\hdots,n} a_j^{p/p_1} \max_{k = 1,\hdots,m} b_k^{q'/q_1'}
			\end{aligned}
		\end{gather*}
		
		
		Hence $F$ is bounded on $\overline{S}$. It is obvious, that $F$ is analytic on $S$ and continuous on $\overline{S}$ (as the sum, product,quotient,composition of analytic/continuous functions). Therefore we can apply Hadamard's three lines lemma to get

		\begin{gather}
			\begin{aligned}
				\vert F(z) \vert &\leqslant \left( M_0  \|f\|_{L^p}^{p/p_0} \|g\|_{L^{q'}}^{q'/q'_0} \right)^{1 - \theta}\left(  M_1 \|f\|_{L^p}^{p/p_1}\|g\|_{L^{q'}}^{q'/q_1'} \right)^\theta\\
			&= M_0^{1 - \theta}M_1^\theta \|f\|_{L^p}\|g\|_{L^{q'}}
			\label{est:F}
		\end{aligned}
		\end{gather}

		for $\mathrm{Re}z = \theta$ where $0 \leqslant \theta \leqslant 1$. Further observe $P(\theta) = Q(\theta) = 1$ for $0 < \theta < 1$ and thus 
		
		\begin{gather}
			\begin{aligned}
				\|T(f)\|_{L^q} &= \sup\left\{\left\vert \int_Y T(f)gd\nu\right\vert : g \in \mathfrak{F}, \|g\|_{L^{q'}} = 1\right\}\\
				&=  \sup\left\{\left\vert F(\theta)\right\vert : g \in \mathfrak{F}, \|g\|_{L^{q'}} = 1\right\}\\
				&\leqslant M_0^{1 - \theta}M_1^\theta \|f\|_{L^p}
				\label{id:F}
			\end{aligned}
		\end{gather}

		Now assume \underline{$q = 1$}. Then $q_0 = q_1 = 1$. Let $g :\equiv \sum_{k = 1}^m b_k e^{i\beta_k}\chi_{Y_k}$ be a simple function (this means, that $\nu(Y_k) = \infty$ is possible for some $Y_k$) with $\|g\|_{L^{\infty}} = 1$. Define $f_z$ as above and

		\begin{equation}
			F(z) := \int_Y T(f_z)(y)g(y)d\nu(y)	
		\end{equation}

		Using the linearity property of $T$ we see again, that $F(z)$ is well-defined on $\overline{S}$. Analogously we get 

		\begin{equation}
			\vert F(it)\vert \leqslant M_0 \|f\|_{L^p}^{p/p_0} \qquad \vert F(1 + it) \vert \leqslant M_1 \|f\|_{L^p}^{p/p_1}
		\end{equation}

		Again, $F$ is bounded on $\overline{S}$ by 

		\begin{gather*}
			\begin{aligned}
				&p_0 \neq \infty, q_0' = \infty: \qquad C(f,g) :=  M_0 \|f\|_{L^{p + (pp_0)/p_1}}^{p/p_0 + p/p_1} \max_{k = 1,\hdots,m} b_k^{q'/q_1'}\\
				&p_0 = \infty, q_0' = \infty: \qquad C(f,g) := M_0 \max_{j = 1,\hdots,n} a_j^{p/p_1} \max_{k = 1,\hdots,m} b_k^{q'/q_1'}
			\end{aligned}
		\end{gather*}

		Hadamard's three lines lemma therefore yields

		\begin{equation}
			\vert F(z)\vert \leqslant \left(M_0 \|f\|_{L^p}^{p/p_0}\right)^{1 - \theta} \left( M_1 \|f\|_{L^p}^{p/p_1}\right)^\theta =  M_0^{1 - \theta} M_1^\theta \|f\|_{L^p}
		\end{equation}

		and observing $P(\theta) = 1$ for $0 < \theta < 1$ yields

		\begin{gather}
			\begin{aligned}
				\| T(f) \|_{L^1} &= \sup \left\{\left\vert \int_Y T(f)gd\nu \right\vert : g \text{ simple}, \|g\|_{L^{\infty}} = 1 \right\}\\
				&= \sup \{\vert F(\theta) \vert : g \text{ simple}, \|g\|_{L^{\infty}} = 1 \}\\
				&\leqslant  M_0^{1 - \theta} M_1^\theta \|f\|_{L^p}
			\end{aligned}
		\end{gather}

		This is justified by the fact, that the simple functions are dense in $L^{\infty}$ (for a proof see \cite[100]{cohn:measure_theory:2013}).	
\end{proof}

\begin{remark}
	As you can see in the proof of the case $q = 1$, it is necessary to have $0 < \theta < 1$. Since for example choosing $q_1 = 1$ and $q_0 > 1$ arbitrary leads for $\theta = 1$ to $q = 1$ but then the function $g$ can be choosen so, that the integral in the definition \textup{(\ref{def:F})} diverges.
\end{remark}

\begin{remark}
	The proof initially given by Grafakos differs in the case study of $q$. I argued in a different way, because I used the density of the simple functions and finitely simple functions whereas he used the theorem given here \textup{\cite[189]{folland:real_analysis:1999}}, which makes the distinction of the cases $q > 1$ and $q = 1$ unnecessary. \end{remark}

\subsection{Young's inequality}
Using the Riesz-Thorin interpolation theorem, we can give an alternative proof of Young's inequality \cite[22--23]{grafakos:fourier:2014}.

\vspace{2mm}

\begin{mdframed}
	\begin{theorem}\emph{(Young's inequality)}
		Let $G$ be a locally compact group, which is a countable union of compact subsets, and let $\eta$ be a left invariant Haar measure. Let $1 \leqslant p,q,r \leqslant \infty$

		\begin{equation}
			\frac{1}{q} + 1 = \frac{1}{p} + \frac{1}{r}
		\end{equation}

		Then for all $f \in L^p(G,\eta)$ and all $g \in L^r(G,\eta)$ satisfying $\|g\|_{L^r} = \|\tilde{g}\|_{L^r}$ we have $f \ast g$ exists $\eta$-a.e. and satisfies

		\begin{equation}
			\|f \ast g\|_{L^q} \leqslant \|g\|_{L^r}\|f\|_{L^p}
		\end{equation}
	\end{theorem}
\end{mdframed}

\begin{proof}
	Fix $g \in L^r(G,\eta)$ and let $T(f) := f \ast g$ be defined on $L^1(G,\eta) + L^{r'}(G,\eta)$. Obviously, $T$ is a linear operator by the linearity of the integral. By Minkowski's integral inequality (see exercise \textbf{1.1.6} \cite[13]{grafakos:fourier:2014}) we get

	\begin{gather}
		\begin{aligned}
			\|T(f)\|_{L^r} &= \left(\int_G \left\vert \int_G f(y)g(y^{-1}x) d\eta(y)\right\vert^r d\eta(x)\right)^{1/r}\\
			&\leqslant \int_G \left( \int_G \vert f(y) \vert^r \vert g(y^{-1}x) \vert^r d\eta(x)\right)^{1/r} d\eta(y)\\
			&= \int_G \vert f(y) \vert \left( \int_G \vert g(y^{-1}x) \vert^r d\eta(y^{-1}x) \right)^{1/r} d\eta(y)\\
			&= \int_G \vert f(y) \vert \left( \int_G \vert g(z) \vert^r d\eta(z) \right)^{1/r} d\eta(y)\\
			&\leqslant \|f\|_{L^1} \|g\|_{L^r}
		\end{aligned} 
	\end{gather}

	for $f \in L^1(g,\mu)$ and $1 \leqslant p < \infty$ (since $(G,\eta)$ is $\sigma$-finite). The case $r = \infty$ follows from
	
	\begin{equation}
		\vert (f \ast g)(x) \vert = \left\vert \int_G f(y)g(y^{-1}x) d\eta(y)\right\vert \leqslant \int_G \vert f(y)\vert \vert g(y^{-1}x)\vert d\eta(y) \leqslant \|g\|_{L^\infty}\|f\|_{L^1}
	\end{equation}
	
	By stipulating $h(y) := g(y^{-1}x)$ we have 

	\begin{gather}
		\begin{aligned}
			\vert (f \ast g)(x) \vert &= \left\vert \int_G f(y)g(y^{-1}x) d\eta(y)\right\vert \leqslant \int_G \vert f(y)g(y^{-1}x)\vert d\eta(y)\\
			&= \|fh\|_{L^1} \leqslant \|f\|_{L^{r'}} \|h\|_{L^r} = \|f\|_{L^{r'}} \|\tilde{g}\|_{L^r} = \|g\|_{L^r} \|f\|_{L^{r'}}
		\end{aligned}
	\end{gather}

	for $r < \infty$ and $f \in L^{r'}(g,\eta)$, since

	\begin{equation*}
		\|h\|^r_{L^r} = \int_G \vert g(y^{-1}x)\vert^r d\eta(y) = \int_G \vert\tilde{g}(x^{-1}y)\vert d\eta(y) = \|\tilde{g}\|^r_{L^{r}}
	\end{equation*}

	The Riesz-Thorin interpolation theorem now yields for any $0 < \theta < 1$

	\begin{equation}
		\|f \ast g\|_{L^q} = \|T(f)\|_{L^q} \leqslant \|g\|_{L^r}^{1 - \theta}\|g\|_{L^r}^{\theta} \|f\|_{L^p} = \|g\|_{L^r}\|f\|_{L^p}
	\end{equation}

	where 

	\begin{equation*}
		\frac{1}{p} = \frac{1 - \theta}{1} + \frac{\theta}{r'} \qquad \frac{1}{q} = \frac{1 - \theta}{r} + \frac{\theta}{\infty}
	\end{equation*}

	and by 

	\begin{equation*}
		\frac{1}{p} = 1 - \frac{\theta}{r} \qquad \frac{1}{q} = \frac{1}{r} - \frac{\theta}{r}
	\end{equation*}

	we get 

	\begin{equation*}
			\frac{1}{q} + 1 = \frac{1}{p} + \frac{1}{r}		
	\end{equation*}
\end{proof}

\begin{remark}
	The proof would be much shorter if we just used Minkowski's inequality \textup{\cite[21--22]{grafakos:fourier:2014}} instead of Minkowski's integral inequality. However, the proof given here is an alternative version of the one given already for Minkowski's inequality.
\end{remark}

\section{The Real Method}
\subsection{The Marcinkiewicz Interpolation Theorem}
The name originates from the real variables technique used for prooving the theorem.
\vspace{2mm}
\begin{mdframed}
	\begin{theorem}\emph{(The Marcinkiewicz Interpolation Theorem)}
		Let $(X,\mu)$ be a $\sigma$-finite measure space, $(Y,\nu)$ another measure space and $0 < p_0 < p_1 \leqslant \infty$. Further let $T$ be a sublinear operator defined on
		
		\begin{equation*}
			L^{p_0} + L^{p_1} := \left\{ f_0 + f_1 : f_0 \in L^{p_0}(X,\mu), f_1 \in L^{p_1}(X,\mu) \right\}
		\end{equation*}
		
		and taking values in the space of measurable functions on $Y$. Assume that there exist $A_0,A_1 < \infty$ such that

		\begin{align}
			&\forall f \in L^{p_0}(X,\mu)~\|T(f)\|_{L^{p_0,\infty}} \leqslant A_0 \|f\|_{L^{p_0}}\label{hyp:fp_0}\\
			&\forall f \in L^{p_1}(X,\mu)~\|T(f)\|_{L^{p_1,\infty}} \leqslant A_1 \|f\|_{L^{p_1}}\label{hyp:fp_1}
		\end{align}

		Then for all $p_0 < p < p_1$ and for all $f \in L^p(X,\mu)$ we have the estimate

		\begin{equation}
			\|T(f)\|_{L^p} \leqslant A \|f\|_{L^p}
		\end{equation}

		where

		\begin{equation}
			A := 2\left( \frac{p}{p - p_0} + \frac{p}{p_1 - p} \right)^{1/p}A_0^{\frac{\frac{1}{p} - \frac{1}{p_1}}{\frac{1}{p_0}-\frac{1}{p_1}}}A_1^{\frac{\frac{1}{p_0}-\frac{1}{p}}{\frac{1}{p_0}-\frac{1}{p_1}}}
			\label{eq:constant}
		\end{equation}
	\end{theorem}
\end{mdframed}

\begin{proof}
	The proof is subdivided into two main parts, which are further subdivided. In detail, we have the following partitioning:

	\begin{enumerate}[label = \textbf{(\roman*.)}]
		\item \underline{$p_1 < \infty$.}
			\begin{enumerate}[label = \textbf{\alph*.}]
				\item Split $f$ using cut-off functions.
				\item Estimate the distribution function $d_{T(f)}$.
				\item Estimate $\|T(f)\|_{L^p}^p$.
			\end{enumerate}
		\item \underline{$p_1 = \infty$.}
			\begin{enumerate}[label = \textbf{\alph*.}]
				\item Show that $\mu(\{\vert T(f_1)\vert > \alpha/2\}) = 0$.
				\item Estimate the distribution function $d_{T(f_0)}$.
				\item Estimate $\|T(f)\|_{L^p}^p$.
			\end{enumerate}
	\end{enumerate}

	\begin{enumerate}[label = \textbf{(\roman*.)}]
		\item 
			\begin{enumerate}[label = \textbf{\alph*.}]
				\item Let us first consider the case \underline{$p_1 < \infty$}. Fix $f \in L^p(X,\mu)$, $\alpha > 0$ and $\delta > 0$ ($\delta$ will be determined later). We split $f$ using so-called \emph{cut-off} functions, by stipulating $f \equiv f_0(\cdot;\alpha,\delta) + f_1(\cdot;\alpha,\delta)$, where $f_0(\cdot;\alpha,\delta)$ is the \emph{unbounded part of $f$} and $f_1(\cdot;\alpha,\delta)$ is the \emph{bounded part of $f$}, defined by

	\begin{gather}
		\begin{aligned}
			f_0(x;\alpha,\delta) &:= \begin{cases}
				f(x), & \vert f(x) \vert > \delta \alpha,\\
				0, & \vert f(x)\vert \leqslant \delta \alpha.
			\end{cases}\\
			f_1(x;\alpha,\delta) &:= \begin{cases}
				f(x), & \vert f(x) \vert \leqslant \delta \alpha,\\
				0, & \vert f(x)\vert > \delta \alpha.
			\end{cases}
		\end{aligned}
		\label{eq:cut_off}
	\end{gather}

	for $x \in X$. To facilitate reading I will omit the dependency of $f_0(\cdot;\alpha,\delta)$ and $f_1(\cdot;\alpha,\delta)$ upon the parameters $\alpha$ and $\delta$ in what follows and simply write $f_0$, $f_1$ respectively. Since $p_0 < p$ we have 

	\begin{gather}
		\begin{aligned}
			\|f_0\|^{p_0}_{L^{p_0}} &= \int_{X} \vert f_0\vert^{p_0} d\mu =\int_{X} \vert f \vert^{p_0} \cdot \chi_{\left\{\vert f\vert > \delta\alpha \right\}} d\mu \overset{(\dagger)}{=} \int_{\left\{\vert f \vert > \delta\alpha \right\}} \vert f \vert^{p_0}d\mu\\ 
			&= \int_{\left\{\vert f\vert > \delta\alpha \right\}} \vert f \vert^p \vert f \vert^{p_0 - p} d\mu = \int_{\left\{\vert f\vert > \delta\alpha \right\}} \frac{\vert f \vert^p}{\vert f \vert^{p - p_0}} d\mu\\
			&\leqslant \frac{1}{(\delta\alpha)^{p - p_0}} \int_{\left\{\vert f\vert > \delta\alpha \right\}} \vert f \vert^p d\mu = (\delta\alpha)^{p_0 - p} \int_{X} \vert f \vert^p \cdot \chi_{\left\{\vert f\vert > \delta\alpha \right\}} d\mu\\
			& \leqslant (\delta\alpha)^{p_0 - p} \int_{X} \vert f \vert^p d\mu = (\delta\alpha)^{p_0 - p} \|f\|^p_{L^p} < \infty
		\end{aligned}
		\label{est:f0}
	\end{gather}

	Thus $f_0 \in L^{p_0}(X,\mu)$. Analogously it can be checked, that $\|f_1\|^{p_1}_{L^{p_1}} \leqslant (\delta\alpha)^{p_1 - p}\|f\|_{L^p}^p$ and so $f_1 \in L^{p_1}(X,\mu)$. Therefore $f \equiv f_0 + f_1 \in L^{p_0} + L^{p_1}$.\\
	
	\emph{Proof of the equality $(\dagger)$.} Assume $\mu$ is defined on the $\sigma$-algebra $\mathcal{A}$. We have to proove that $\{\vert f \vert > \delta\alpha\} \in \mathcal{A}$\footnote{
		For $Y \in \mathcal{A}$ the $\mu$-integral of $f: X \rightarrow \mathbb{C}$ over $Y$ is defined to be $\displaystyle \int_Y fd\mu := \int_X f \cdot \chi_Y d\mu$. For more details see \cite[135--136]{elstrodt:mass:2011}.}.
		Since $f$ is complex-valued, we may write $f \equiv \mathrm{Re} f + i\mathrm{Im}f$ and thus $\vert f\vert^2 \equiv \mathrm{Re}^2 f + \mathrm{Im}^2f$. Since $f$ is measurable by hypothesis this implies that $ \mathrm{Re} f$ and $\mathrm{Im}f$ are measurable\footnote{For a proof see \cite[106]{elstrodt:mass:2011}}. Further for measurable real-valued functions $f,g: (X,\mathcal{A}) \rightarrow (\overline{\mathbb{R}},\overline{\mathfrak{B}})$\footnote{$\overline{\mathfrak{B}} := \sigma(\overline{\mathbb{R}})$ and $\overline{\mathfrak{B}} = \{B \cup E : B \in \mathfrak{B}, E \subseteq \{\pm \infty\}\}$.} 
		the functions $f + g$ and $f \cdot g$ are measurable\footnote{For a proof see \cite[107]{elstrodt:mass:2011}.}
		and thus $\vert f \vert^2$ is measurable. Hence $\{ \mathrm{Re}^2 f + \mathrm{Im}^2f > \lambda\} \in \mathcal{A}$\footnote{For a proof see \cite[105--106]{elstrodt:mass:2011}} for any $\lambda \in \mathbb{R}$. So especially for $\lambda := (\delta\alpha)^2$ we have $\{\vert f \vert > \delta\alpha\} \in \mathcal{A}$\footnote{This follows from the fact that $x < y$ if and only if $x^n < y^n$ for $n \in \mathbb{N}_{>0}$ and some real numbers $x,y > 0$ (see \cite[119]{zorich:analysis_I:2004}).}.
	In a similar manner it can also be prooven that $\{\vert f\vert \leqslant \delta \alpha\} \in \mathcal{A}$. Let us next proove a useful lemma.

	\begin{lemma}
		Let $A \in \powerset(X)$ and $\chi_A: (X,\mathcal{A}) \rightarrow (\mathbb{C},\mathfrak{B}^2)$ be the \emph{characteristic function of the set $A$}. Then $\chi_A$ is measurable if and only if $A$ is measurable.
			\label{lem:charfun}
	\end{lemma}
		
	\begin{proof}
		Assume $\chi_A$ is measurable. Then $\mathrm{Re}\chi_A$ and $\mathrm{Im}\chi_A$ are measurable. Especially for $0 < \lambda < 1$ we have that $\{\mathrm{Re}\chi_A > \lambda\} = A \in \mathcal{A}$. Conversly, assume $A$ is measurable. For $\lambda < 0$ we have $\{\mathrm{Re}\chi_A > \lambda\} = X \in \mathcal{A}$, $\lambda \in [0,1[$, $\{\mathrm{Re}\chi_A > \lambda\} = A \in \mathcal{A}$ and $\{\mathrm{Re}\chi_A > \lambda\} = \emptyset \in \mathcal{A}$ for $\lambda \geqslant 1$. Since $\mathrm{Im}\chi_A \equiv 0$ we have $\{\mathrm{Im}\chi_A > \lambda \} = X \in \mathcal{A}$ if $\lambda < 0$ and $\{\mathrm{Im}\chi_A > \lambda\} = \emptyset \in \mathcal{A}$ if $\lambda \geqslant 0$.   
	\end{proof}

	By Lemma \ref{lem:charfun} and the fact that $f \cdot g$ is measurable for two measurable functions $f,g: (X,\mathcal{A}) \rightarrow (\mathbb{C},\mathfrak{B}^2)$\footcite[107]{elstrodt:mass:2011}, $f_0$ and $f_1$ are measurable since $f_0 \equiv f \cdot \chi_{\{\vert f \vert > \delta \alpha\}}$ and $f_1 \equiv f \cdot \chi_{\{\vert f \vert \leqslant \delta \alpha\}}$.\\

	One subtility is left to clear: the $\mu$-integrability of either $\vert f_1\vert^{p_0}$ or $\vert f_1 \vert^{p_1}$ requires that $\vert f_0 \vert^{p_0}$ and $\vert f_1 \vert^{p_1}$ are measurable functions. By the fact that any continuous map $g: (X,d_X) \rightarrow (Y,d_Y)$ between metric spaces is Borel-measurable (see \cite[86]{elstrodt:mass:2011}) and that the composition of measurable functions is again measurable (see \cite[87]{elstrodt:mass:2011}), the measurability of either $f_0$ or $f_1$ follows by $\vert f_0 \vert^{p_0} \equiv \cdot^{p_0} \circ \vert f \cdot \chi_{\{\vert f\vert > \delta\alpha\}}\vert$ and $\vert f_1 \vert^{p_1} \equiv \cdot^{p_1} \circ \vert f \cdot \chi_{\{\vert f \vert \leqslant \delta \alpha\}}\vert$ by stipulating $\cdot^{p}: (\mathbb{R}_{\geqslant 0},\vert \cdot \vert) \rightarrow (\mathbb{C},\vert \cdot \vert)$, $x^{p} := \exp(p \log(x))$ for $p > 0$ and $x \in \mathbb{R}_{> 0}$ and $x^p := 0$ if $x = 0$.

	\item Since $T$ is a sublinear operator we have $\vert T(f) \vert = \vert T(f_0 + f_1) \vert \leqslant \vert T(f_0) \vert + \vert T(f_1)\vert$. Thus for any $y \in Y$ with $\vert T(f)(y) \vert > \alpha$ we therefore have either $\vert T(f_0)(y) \vert > \alpha/2$ or $\vert T(f_1)(y) \vert > \alpha/2$ 
		\footnote{Without loss of generality assume $\vert T(f_0)(y) \vert \leqslant \vert T(f_1)(y) \vert $. Then we have $\alpha < \vert T(f)(y)\vert \leqslant \vert T(f_0)(y) \vert + \vert T(f_1)(y)\vert \leqslant 2\vert T(f_1)(y)\vert$ (this is possible since $\mathbb{R}$ is an ordered field).}
		. Hence

\begin{equation*}
	\{\vert T(f)\vert > \alpha \} \subseteq \{\vert T(f_0) \vert > \alpha/2 \} \cup \{\vert T(f_1) \vert > \alpha/2 \}
\end{equation*}

and so by the monotonicity and subadditivity property of the measure $\mu$ we have

\begin{gather}
	\begin{aligned}
	d_{T(f)}(\alpha) &= \mu(\{\vert T(f)\vert > \alpha\})\\
	&\leqslant \mu(\{\vert T(f_0)\vert > \alpha/2 \} \cup \{\vert T(f_1)\vert > \alpha/2 \})\\
	&\leqslant \mu(\{\vert T(f_0) \vert > \alpha/2 \}) + \mu(\{\vert T(f_1)\vert > \alpha/2 \})\\
	&= d_{T(f_0)}(\alpha/2) + d_{T(f_1)}(\alpha/2)
	\label{est:T}
	\end{aligned}
\end{gather}

Now by hypothesis (\ref{hyp:fp_0}) we can estimate $d_{T(f_0)}(\alpha/2)$ as follows

\begin{gather}
	\begin{aligned}
		d_{T(f_0)}(\alpha/2) &= \left(\frac{\alpha/2}{\alpha/2}\right)^{p_0} d_{T(f_0)}(\alpha/2)\\
		&\leqslant \left(\frac{1}{\alpha/2}\right)^{p_0} \left[\sup\left\{ \gamma d_{T(f_0)}(\gamma)^{1/p_0}: \gamma > 0\right\}\right]^{p_0}\\
	 & = \left(\frac{1}{\alpha/2}\right)^{p_0} \|T(f_0)\|^{p_0}_{L^{p_0,\infty}}\\
	 & \leqslant \left(\frac{A_0}{\alpha/2}\right)^{p_0} \|f_0\|^{p_0}_{L^{p_0}}
	\label{est:p_0}
	\end{aligned}
\end{gather}

Analogously, we get by hypothesis (\ref{hyp:fp_1}) the estimate $d_{T(f_1)}(\alpha/2) \leqslant \left(\frac{A_1}{\alpha/2}\right)^{p_1} \|f_1\|^{p_1}_{L^{p_1}}\label{est:p_1}$.

	\item By

		\begin{gather}
			\begin{aligned}
				\int_0^{\frac{1}{\delta}\vert f\vert}\alpha^{p-p_0-1} d\lambda = 
				\begin{cases}
					\frac{1}{p-p_0}\frac{1}{\delta^{p-p_0}}\vert f \vert^{p - p_0}, & p \geqslant p_0 + 1\\
					\lim\limits_{\omega \rightarrow 0^+} \int_\omega^{\frac{1}{\delta}\vert f\vert}\alpha^{p-p_0-1} d\lambda\\
					= \lim\limits_{\omega \rightarrow 0^+}\left[\frac{1}{p-p_0}\alpha^{p - p_0}\big\vert_\omega^{\frac{1}{\delta}\vert f\vert}\right]\\
					= \frac{1}{p-p_0}\left[\frac{1}{\delta^{p-p_0}}\vert f \vert^{p - p_0} - \lim\limits_{\omega \rightarrow 0^+} \omega^{p-p_0}\right]\\
					= \frac{1}{p-p_0}\frac{1}{\delta^{p-p_0}} \vert f\vert^{p - p_0}, & p_0 < p < p_0 + 1
				\end{cases}
			\end{aligned}
		\end{gather}

		and

		\begin{gather}
			\begin{aligned}
				\int_{\frac{1}{\delta}\vert f\vert}^{\infty}\alpha^{p-p_1-1} d\lambda &= \lim_{\omega \rightarrow \infty} \left[ \frac{1}{p - p_1} \alpha^{p - p_1}\right]^\omega_{\frac{1}{\delta}\vert f\vert}\\
				&= \frac{1}{p - p_1} \left[  \lim_{\omega \rightarrow \infty} \omega^{p - p_1} - \frac{1}{\delta^{p - p_1}} \vert f\vert^{p - p_1}\right]\\
				&= \frac{1}{p_1 - p}\frac{1}{\delta^{p-p_1}} \vert f \vert^{p - p_1}
			\end{aligned}
		\end{gather}

		and the representation $\displaystyle \|f\|^p_{L^p} = p \int_0^{\infty} \alpha^{p-1}d_f(\alpha) d\lambda$ for $ 0 < p < \infty$ we get

		\begin{gather}
			\begin{aligned}
				\|T(f)\|^p_{L^p(Y,\mathcal{B},\nu)} = & p\int_0^{+\infty}\alpha^{p-1}d_{T(f)} d\lambda\\
				\leqslant & p(2A_0)^{p_0}\int_0^{\infty}\alpha^{p-p_0-1} \int_{\{\vert f \vert > \delta \alpha\}} \vert f\vert^{p_0}d\mu d\lambda\\
				& + p(2A_1)^{p_1}\int_0^{+\infty}\alpha^{p-p_1-1} \int_{\{\vert f \vert \leqslant \delta \alpha\}} \vert f \vert^{p_1}d\mu d\lambda\\
				= & p(2A_0)^{p_0}\int_{\{\vert f \vert > 0\}} \vert f \vert^{p_0} \int_0^{\frac{1}{\delta}\vert f\vert}\alpha^{p-p_0-1} d\lambda d\mu\\
				& + p(2A_0)^{p_0}\int_{\{\vert f \vert = 0\}} \vert f \vert^{p_0} \int_0^{\frac{1}{\delta}\vert f\vert}\alpha^{p-p_0-1} d\lambda d\mu\\
				& + p(2A_1)^{p_1}\int_X \vert f\vert^{p_1} \int_{\frac{1}{\delta}\vert f\vert}^{+\infty}\alpha^{p - p_1 - 1} d\lambda d\mu\\
				= & p(2A_0)^{p_0}\int_X \vert f \vert^{p_0} \int_0^{\frac{1}{\delta}\vert f\vert}\alpha^{p-p_0-1} d\lambda d\mu\\
				& + p(2A_1)^{p_1}\int_X \vert f\vert^{p_1} \int_{\frac{1}{\delta}\vert f\vert}^{+\infty}\alpha^{p - p_1 - 1} d\lambda d\mu\\
				= & \frac{p(2A_0)^{p_0}}{p-p_0}\frac{1}{\delta^{p-p_0}}\int_X \vert f \vert^{p_0}\vert f \vert^{p-p_0} d\mu\\
				& + \frac{p(2A_1)^{p_1}}{p_1-p}\frac{1}{\delta^{p-p_1}}\int_X \vert f \vert^{p_1} \vert f\vert^{p-p_1}d\mu\\
				= & p\left( \frac{(2A_0)^{p_0}}{p - p_0}\frac{1}{\delta^{p - p_0}} + \frac{(2A_1)^{p_1}}{p_1 - p}\delta^{p_1 - p} \right)\|f\|_{L^p}^p
			\end{aligned}
			\label{est:Tfp}
		\end{gather}

		We pick $\delta > 0$ such that $(2A_0)^{p_0}\delta^{p_0 - p} = (2A_1)^{p_1}\delta^{p_1 - p}$. Solving for $\delta$ yields 

		\begin{equation}
			\delta = \frac{1}{2} \left( \frac{A_0}{A_1}\right)^{p_1/(p_1 - p_0)}
		\end{equation}

		Substituting this in estimate (\ref{est:Tfp}) leads to

		\begin{gather}
			\begin{aligned}
				\|T(f)\|_{L^p}^p &\leqslant p\left( \frac{(2A_0)^{p_0}}{p - p_0}\frac{2^{p - p_0}A_1^\frac{p_1(p-p_0)}{p_1-p_0}}{A_0^\frac{p_0(p-p_0)}{p_1 - p_0}} + \frac{(2A_1)^{p_1}}{p_1 - p} \frac{A_0^\frac{p_0(p_1 - p)}{p_1 - p_0}}{2^{p_1 - p}A_1^\frac{p_1(p_1 - p)}{p_1 - p_0}} \right)\|f\|_{L^p}^p\\
				&=  2^pp\left( \frac{A_0^\frac{p_0(p_1 - p)}{p_1 - p_0}A_1^\frac{p_1(p-p_0)}{p_1-p_0}}{p - p_0} + \frac{A_0^\frac{p_0(p_1 - p)}{p_1 - p_0}A_1^\frac{p_1(p - p_0)}{p_1 - p_0}}{p_1- p} \right)\|f\|_{L^p}^p\\
			\end{aligned}
		\end{gather}

		And taking the $p$-th power further

		\begin{gather}
			\begin{aligned}
				\|T(f)\|_{L^p} &\leqslant 2\left( \frac{p}{p - p_0} + \frac{p}{p_1- p} \right)^{1/p} A_0^\frac{p_0(p_1 - p)}{p(p_1 - p_0)}A_1^\frac{p_1(p - p_0)}{p(p_1 - p_0)}\|f\|_{L^p}\\
				&= 2\left( \frac{p}{p - p_0} + \frac{p}{p_1- p} \right)^{1/p} A_0^{\frac{p_0(p_1 - p)}{p(p_1 - p_0)}\frac{p_1}{p_1}}A_1^{\frac{p_1(p - p_0)}{p(p_1 - p_0)}\frac{p_0}{p_0}}\|f\|_{L^p}\\
				&= 2\left( \frac{p}{p - p_0} + \frac{p}{p_1- p} \right)^{1/p} A_0^{\frac{\frac{p_1 - p}{pp_1}}{\frac{p_1 - p_0}{p_0p_1}}}A_1^{\frac{\frac{p - p_0}{p_0p}}{\frac{p_1 - p_0}{p_0p_1}}}\|f\|_{L^p}\\
				&= 2\left( \frac{p}{p - p_0} + \frac{p}{p_1- p} \right)^{1/p} A_0^{\frac{\frac{1}{p} - \frac{1}{p_1}}{\frac{1}{p_0} - \frac{1}{p_1}}}A_1^{\frac{\frac{1}{p_0} - \frac{1}{p}}{\frac{1}{p_0} - \frac{1}{p_1}}}\|f\|_{L^p}
			\end{aligned}
		\end{gather}
	\end{enumerate}
	
\item
	\begin{enumerate}[label = \textbf{\alph*.}]
		\item Assume \underline{$p_1 = \infty$}. We again use the cut-off functions defined in (\ref{eq:cut_off}) to decompose $f$.  Since $\{\vert f_1\vert > \delta\alpha \} = \emptyset$, we have 

\begin{equation*}
	\|T(f_1)\|_{L^\infty} \leqslant A_1 \|f_1\|_{L^\infty} = A_1 \inf \left\{B > 0: \mu(\{\vert f_1 \vert > B\}) = 0 \right\} \leqslant A_1\delta\alpha = \alpha/2
\end{equation*}

Provided we stipulate $\delta := 1/(2A_1)$. Therefore the set $\{\vert T(f_1) \vert > \alpha/2\}$ has measure zero (this is immediate since $\|T(f_1)\|_{L^\infty} =  \inf \left\{B > 0: \mu(\{\vert T(f_1) \vert > B\}) = 0 \right\} \leqslant \alpha/2 $ and any subset of a set with measure zero has itself measure zero). Thus similar to part \textbf{b.} of \textbf{(i.)} we get $d_{T(f)}(\alpha) \leqslant d_{T(f_0)}(\alpha/2)$.

	\item Hypothesis (\ref{hyp:fp_0}) yields the estimate $\displaystyle d_{T(f_0)}(\alpha/2) \leqslant \left(\frac{A_0}{\alpha/2}\right)^{p_0} \int_{\{2A_1\vert f \vert > \alpha\}} \vert f \vert^{p_0}d\mu$.

	\item Thus by \textbf{a.} and \textbf{b.}

	\begin{gather}
		\begin{aligned}
			\|T(f)\|_{L^p}^p &= p \int_0^{\infty} \alpha^{p-1}d_{T(f)} d\lambda\\
			&\leqslant p (2A_0)^{p_0} \int_0^{\infty} \alpha^{p-p_0-1} \int_{\{2A_1\vert f \vert > \alpha\}} \vert f \vert^{p_0}d\mu d\lambda\\
			&= p(2A_0)^{p_0} \int_X \vert f\vert^{p_0} \int_0^{2A_1\vert f \vert} \alpha^{p - p_0 - 1}d\lambda d\mu\\
			&= \frac{2^ppA_0^{p_0}A_1^{p - p_0}}{p - p_0} \int_X \vert f\vert^{p} d\mu\\
			&= \frac{2^ppA_0^{p_0}A_1^{p - p_0}}{p - p_0} \|f\|_{L^p}^p
			\label{est:p_1_infty}
		\end{aligned}
	\end{gather}

	That the constant $2^ppA_0^{p_0}A_1^{p - p_0}/(p - p_0)$ found in (\ref{est:p_1_infty}) is the $p$-th power of the one stated in the theorem can be seen by passing the constant (\ref{eq:constant}) to the limit $p_1 \rightarrow \infty$:

	\begin{gather*}
		\begin{aligned}
			\lim\limits_{p_1 \rightarrow \infty} A =& \lim\limits_{p_1 \rightarrow \infty}	\left[2\left( \frac{p}{p - p_0} + \frac{p}{p_1 - p} \right)^{1/p}A_0^{\frac{\frac{1}{p} - \frac{1}{p_1}}{\frac{1}{p_0}-\frac{1}{p_1}}}A_1^{\frac{\frac{1}{p_0}-\frac{1}{p}}{\frac{1}{p_0}-\frac{1}{p_1}}}\right]\\
			=& 2\exp\left[ \frac{1}{p} \log\left(\frac{p}{p - p_0} + \lim\limits_{p_1 \rightarrow \infty} \frac{1}{p_1}\frac{p}{1 - p\lim\limits_{p_1 \rightarrow \infty}\frac{1}{p_1}}\right)\right]\\
			& \cdot \lim\limits_{p_1 \rightarrow \infty}A_0^{\frac{\frac{1}{p} - \frac{1}{p_1}}{\frac{1}{p_0}-\frac{1}{p_1}}}\cdot\lim\limits_{p_1 \rightarrow \infty}A_1^{\frac{\frac{1}{p_0}-\frac{1}{p}}{\frac{1}{p_0}-\frac{1}{p_1}}}\\
		=& 2\left( \frac{p}{p - p_0} \right)^{1/p} \exp\left[\displaystyle\frac{\frac{1}{p} - \lim\limits_{p_1 \rightarrow + \infty}\frac{1}{p_1}}{\frac{1}{p_0}-\lim\limits_{p_1 \rightarrow \infty}\frac{1}{p_1}}\log(A_0)\right]\\
		& \cdot \exp\left[\frac{\frac{1}{p_0}-\frac{1}{p}}{\frac{1}{p_0}-\lim\limits_{p_1 \rightarrow \infty}\frac{1}{p_1}}\log(A_1)\right]\\
		=& 2\left( \frac{p}{p - p_0} \right)^{1/p} A_0^{\frac{p_0}{p}} A_1^{1 - \frac{p_0}{p}}
		\end{aligned}
	\end{gather*}
	\end{enumerate}

\end{enumerate}
\end{proof}
